\documentclass[11pt, oneside, reqno]{article}
\usepackage{amssymb, amsthm, amsmath, amsfonts}
%\parindent=0pt

\newcommand\bs{\char '134 }
\newcommand{\Gc}{\mathcal{G}}
\newcommand{\Lc}{\mathcal{L}}
\newcommand{\Ic}{\mathcal{I}}
\newcommand{\Kc}{\mathcal{K}}
\newcommand{\Ec}{\mathcal{E}}
\newcommand{\Ne}{\mathbb{N}}
\newcommand{\Rc}{\mathcal{R}}
\newcommand{\KR}{\Kc\Rc}


\begin{document}

\title{Automatic application of interdependent abstraction rules}
\author{Houssam Abbas}
\date{\today}
\maketitle

%\newpage
%\tableofcontents        
%\newpage

\section{Introduction}
Given a model of a system, a requirement $\varphi$, and a set of abstraction rules, we seek an automatic way to obtain the most abstract model of the system that is compatible with $\varphi$.
A model is \emph{compatible} with a spec if the truth value of the spec can be evaluated on any run of the model.
In particular, this requires that the atomic propositions of $\varphi$ refer to signals that are produced by the model.

\section{Representing abstraction rules}
We are given a finite directed graph $G=(V,E)$ representing the human heart. 
Each edge is a conduction path, and each vertex is a node.
Each vertex $v$ is decorated with an extended timed automaton $A_v$.
The expressions of invariants and guards of the automaton use a set of parameters $P_v$. 
These parameters can be reset at transitions of the automaton.
%Collectively, we refer to $(A_v,P_v,\Ic_v)$ as the dynamics $D_v$ of the vertex $v$.
Similarly, each edge $e$ is associated with an extended timed automaton $A_e$.
All vertex automata have one of 2 structures, and all path automata have the same structure. 
The differences between them are in the values of the parameters.

Let $I(G)$ be the set of all induced subgraphs of $G$, and $\Gc$ be the set of all finite directed graphs.
In what follows, all subgraphs are induced by $G$ unless stated otherwise.
Given subgraph $K$, $P_K$ is the parameter vector of all vertices and edges of $K$: $P_K = (P_x)_{x \in E(K) \cup V(K)}$.
Also, $\Ec(K) = V(K) \cup E(K)$.
For the rest of this document, fix the graph $G$.

A \emph{rule} $R$ is a partial function on the set $\Gc$ of finite directed subgraphs:
\[R: \Gc \rightarrow \Gc\]
The argument of $R$ might be all of $G$.
The rule has the following effect
\begin{itemize}
	\item $|V(R(K))| \leq |V(K)|$,
	\item $|E(R(K))| \leq |E(K)|$,
	\item $V(R(K)) \subset V(K)$,
	\item For every $x \in V(K) \cap V(R(K)) \bigcup E(K) \cap E(R(K))$, $R(A_x)$ is an abstraction of $A_x$, in the sense that $\Lc(R(A_x))$ is a superset of $\Lc(A_x)$.
\end{itemize}
\emph{Informally}, the rule $R$ operates by one of four ways: deleting a vertex and all edges incident on it, deleting an edge, merging two vertices together, fusing two edges if the vertex they share has degree 2.
With each of these effects, the rule will also modify the \emph{parameter values} of the automata decorating some of the elements of the subgraph to which it applies.
abstracting (in the sense of behavior inclusion) the automaton decorating a vertex or edge, over-approximating the intervals decorating a vertex or edge.
A rule does NOT modify the structure of the automata.
Let $\Rc$ be the set of all rules we are concerned with.

\subsection{Rule applicability}
\label{ruleApplicability}
Each rule $R$ is associated with a subgraph $H_R = (V_R,E_R)$ and formula $\phi_R$ carrying on the parameters of $H_R$.
A rule is defined on a subgraph $K$ iff $K$ is isomorphic to $H_R$ and its parameters satisfy $\phi_R$
\begin{equation}
\label{eq:applicableR}
R(K) \textrm{ is defined iff } K \equiv H_R \land P_K \models \phi_R
\end{equation}
E.g., Rule $R_1$ applies only on linear graphs with 3 vertices and whose parameters $ERP$ obey $ERP(v_1) < ERP(v_2)$ and $ERP(v_2) > ERP(v_3)$.\footnote{A cleaner, but perhaps less intuitive, way of doing this is by defining $R$'s domain to be the set of equivalence classes of $G$ induced by the isomorphism-to-$H_R$ relation. 
	That is, $R: I(G)/H_R  \rightarrow \Gc$ where $I(G)/H_R = \{C_1,C_2\}$ is the set of equivalence classes $C_1 = \{K \in I(G) \;|\; K \equiv H_R \}$ and $C_2 = \{K \in I(G) \;|\; K \not \equiv H_R\}$.
}

We assume that $\phi_R$ is given by linear predicates over $P$
\begin{equation}
\label{eq:formOfPhi}
\exists B_i, C_i\; s.t. P_K \models \phi_R \textrm{ iff } B_i P_K \leq C_i
\end{equation}

If a given subgraph is not of the form on which $R$ is applicable, then $R$ is not defined on it.
In what follows, we let $\Kc$ be the set of all induced subgraphs of $G$ that are isomorphic to some $H_{R}, R \in \Rc$ : 
$\Kc = \{K \in I(G) \; s.t. \; \exists R \in \Rc . K \equiv H_R\}$.

In a given $H_R$ (treated as the equivalence class of all subgraphs isomorphic to it), we may uniquely identify its vertices and the edges by numbering them in an increasing sequence.
We call the integer associated with a vertex or edge the \emph{role} played by that element in $H_R$.
E.g. given the linear graph $v e v' e' v''$, 
role($v$) = 0, role($v'$) = 1, role($v''$) = 2, role($e$) = 3, role($e'$) = 4.
When we determine that $K$ is isomorphic to $H_R$, the vertices and edges of $K$ are assigned the same roles as their isomorphic counterparts in $H_R$.
We call the elements of a subgraph $H_R$ \emph{co-actors} in the rule $R$, since they share the same stage $H_R$.

Given a rule $R_i$, let $\Kc(i) \subset \Kc$ be the set of induced subgraphs to which $R$ is applicable, i.e. 
\[\Kc(i) = \{K \subset \Kc \;|\; K \equiv H_{R_i}\}\]
In what follows we make heavy use of the set 
\[\Kc\Rc = \{(i,s) \;|\; s \in \Kc(i)\}\]

\subsection{Rule application}
Only one rule can be applied at a time. 

In a given application, a rule is applied to only one subgraph.

\subsection{Rule effect}
\label{ruleEffect}
When a rule is applied on a subgraph $K$, its effect on a vertex $v \in V(K)$ depends on the role played by $v$ in $K$. 
E.g. in $H_R = v e v' e' v''$ and $K = v_1 e v_2 e' v_3$, assume that $R$ deletes the middle vertex $v'$, so $R(H_R) =  v e^+ v''$ (where $e^+$ is obtained by fusing $e$ and $e'$ together).
To know how $R$ affects $v_2$ we need to know what its role is.
Because role($v_2$) = role($v'$), we know that $v_2$ will be deleted. 
On the other hand, because role($v_1$) = role($v$) and $v$ is untouched, we know $v_1$ is untouched.

Given a graph $G = (V,E)$, a finite set of rules $\Rc = \{R_1,\ldots,R_m\}$, we have a choice of the order in which to apply the rules on $G$, and at each step (= each rule application), a choice of the induced subgraph on which to apply it.
Thus when we speak of applying a rule, we must give the rule $R_i$, the time step $j$ at which it is applied, and the particular subgraph $K$ to which it is being applied.

The boolean variable $a_{isj}$ tracks that: it is 1 iff rule $R_i$ is applied at step $j$ to subgraph $K_s \in \Kc$.
Then the set $\{a_{isj}(x), x \in V \cup E, i \in [|\Rc|], j \in [N], k \in [||V|+|E|]\}$ characterizes completely the sequence of rule applications and the graphs to which they were applied.
In other words, by reading the sequence of $a_{isj}$'s, we can tell what rule was applied when to what subgraph.
%
%When examining the effect of a rule application on a given element $x \in V \cup E$, we must give the rule $R_i$, the subgraph $K_s$ to which it applies, and the role played by $x$ in that subgraph.
%Given a vertex $v$, the triplet $(k,i,s)$ says that $v$ plays role $k$ for rule $R_i$ in subgraph (or `stage') $K_s$.
%A vertex may have many such triplets since it may be shared by many subgraphs and play different roles in them for different rules.
%Same goes for edges.
%Then we define the function $\rho: V \cup E \rightarrow 2^{\Ne \times \Ne \times \Gc}$ by 
%\begin{equation}
%\rho(x) = \{(k,i,s) \; s.t. \; x \textrm{ plays role $k$ for rule $R_i$ in subgraph }K_s\}
%\end{equation}
%In what follows we write
%\[\rho_1(x) = \{k \in \Ne \; s.t. \; \exists (k,i,s) \in \rho(x)\}\]
%\[\rho_2(x) = \{i \in \Ne \; s.t. \; \exists (k,i,s) \in \rho(x)\}\]
%\[\rho_3(x) = \{s \in \Ne \; s.t. \; \exists (k,i,s) \in \rho(x)\}\]

\section{Disabling of rules}
\label{disablingOfRules}
When a rule is applied, it changes the graph, and this affects the applicability of the rules to the resulting graph. 
For example, if a rule deletes a vertex, then that vertex can not have any rule applied to it from that point onwards. 
As an example, consider $H_{R_i} = v_1e_1v_2e_2v_3e_3v_4$, and the rule fuses $e_2$ to $e_3$ so $R(H_{R_i}) = v_1e_1v_2e^+v_4$.
Then the effect of this rule is that $e_2$ and $e_3$ can no longer play any role from that point onwards (in any rule in any subgraph) since they've disappeared. 
Same for $v_3$ which linked them.
If roles($e_2,e_3,v_2$) = $4,5,6$ then we define $<i> = \{4,5,6\}$ to be the set of roles that are disabled forever by an application of this rule $R_i$.

Note that we don't try to model what role might be played by the new edge $e^+$, since this would complicate things tremendously.
Note also that instead of introducing a new edge $e^+$ we could've just modeled this as the survival of $e_2$, i.e. $R(H_{R_i}) = v_1e_1v_2e_2v_4$. 

\subsection{Disabling by structure change}
When a rule is applied on a particular subgraph $K_s$, it disables forever its vertices and edges whose roles are listed in $<i>$.
These elements $x$ also played other roles for other rules in other subgraphs.
The conservative thing to do then is to assume that those other subgraphs no longer exist as structures, and thus should be disabled or `destroyed'.
\footnote{This is not the most accurate thing to do since the new edge $e^+$ (or the surviving edge $e_2$) might save the situation but that's too comlicated.}
Therefore we define the set of subgraphs destroyed by the application of rule $R_i$ to subgraph $K_s$ to be: $\forall R_i \in \Rc, K_s \in \Kc(i)$
\begin{equation}
\label{eq:Di}
 D(i,s) := \textrm{set of subgraphs destroyed by the application (i,s)}
\end{equation}
%
%\begin{equation}
%\label{eq:Di}
%\forall R_i \in \Rc, D(i) := \{s \in \rho_3(x) \;|\; \text{roles}(x) \in <i>\}
%\end{equation}

The disabling action of the rules can now be stated as
\begin{equation}
\label{eq:disabling}
\forall i,j,s \; a_{isj} =1 \implies a_{i's'j'} = 0 \;\forall \; i', j'>j,s' \in D(i,s)
\end{equation}

\subsection{Parameter updates}
Another way in which a rule might affect the application of future rules is by its effect on the automata parameters.
Let $P_x[0]$ be the initial value of the parameter vector $P_x$ (i.e., before any rules application).
Let $P_x[j]$ be its value at time $j$ after $j$ rule applications.
So the rule to be applied at time $j$ uses the value $P_x[j]$ to determine its applicability.

In general, a rule application might update the parameters of an element $x$.
We postulate that 
\begin{enumerate}
	\item [[Locality of Parameter Update]] (LoPU) When a rule is applied on a subgraph $K$ containing $x$, the parameter $P_x$ is updated using the values of $x$'s co-actors only. 
	That is, the parameters $P_y$ of $y \in V(K)\cup E(K)$ only.
	\label{LoPU}
\end{enumerate}

The form of the update depends on the rule and the role of the element in that rule.
So in the equation that follows, we index the update function $f_{ik}$ by the rule $R_i$ and the role $k=role(x,R_i)$.
Since only the rule applied at the previous step should play a part in the update, we pre-multiply $f_{ik}$ by $a_{i,j-1,s}$ which tells us which rule was applied.
Since the assumption \ref{LoPU} says that only parameters from co-actors affect $P_x$, only the co-actors appear as arguments to $f_{ik}$.
Note that $x$ itself therefore appears in the argument.
The general form of the udpate is
\begin{eqnarray}
\label{eq:paramUpdate}
&&\forall x, P_x[0] = P_{x,0} \textrm{ (initalization)}
\\
&& \forall x, j>0
\nonumber
\\
&& P_x[j] = P_x[j-1] + 
\nonumber
\\
&& \sum_{i,s \in \Kc(i)}a_{i,s,j-1} (-P_x[j-1] +  f_{i,role(x,i)}(P_y[j-1],y \in V(K_s)\cup E(K_s))
\nonumber
\end{eqnarray}

Function $f_{i,k}$ will take the form either min, max or sum.

The following axioms we assume for rule applications:
\begin{enumerate}
	\item [[Sanity]] If $G$ is physiological (i.e. is a physiologically meaningful model of the heart), then $R(G)$ is physiological. This is a sanity requirement on our rules.
	\label{axiomSanity}
	\item [[Permanence]] Once a stage $K$ is destroyed by the application of a rule $R_i$, it is destroyed forever.
	Thus, if a vertex is modified in such a way that a rule $R_i$ can't be applied to it \emph{in role $k$} at some step, no future rule applications will make $R_i$ applicable to it again.
	\label{axiomPermanence}
	\item [[Time-Invariance (TI)]] The effect of a rule on the structure of a graph does not depend on when the rule was applied.
	E.g., if a rule merges two vertices, then this effect is the same regardless of whether the rule was applied at step $j$ or step $j'$.
	\label{axiomTI}
	\item [[Space-Invariance (SI)]] The effect of a rule on a vertex does not depend on the specific subgraph on which the rule is applied. 
	Rather it only depends on the role.
	E.g., if a rule merges a vertex with one of its neighbors, it does so regardless of the subgraph it is currently being applied to. 
	(We could have achieved the same effect by defining the rule directly as a function of equivalence classes induced by isomorphism to $H_R$).
	\label{axiomSI}
\end{enumerate}

\section{Expressing model appropriateness for the requirement}

Recall that we seek a model that is compatible with the requirement $\phi$.
This implies that the truth value of every atomic proposition appearing in the spec $\phi$ can be evaluated on any run of the model.
Let $AP_\phi$ be the set of atomic propositions appearing in $\phi$.
In general, the set $AP_\phi$ will contain propositions involving the edges and vertices of the model $G$ and their dynamics. 
Therefore, for compatibility, the final model should preserve the edges and vertices appearing in $AP_\phi$. 
Let $V_\phi$ and $E_\phi$ be those vertices and edges.
They are preserved if no rule involves them in any role.
This is expressed as
\begin{equation}
\label{eq:Vphi}
\forall j, \forall i, \forall s \; s.t. \; K_s \cap (V_\phi \cup E_\phi)\neq \emptyset, a_{isj}=0
\end{equation}


\section{A measure of abstractness}
We must define a measure of `abstractness'.
Based on the fact that a rule decreases the number of vertices and edges in the subgraph to which it is being applied, the abstraction power of a rule is partially defined as $|V(K)| - |V(R_i(K))| + |E(K)| - |E(R_i(K))| := \delta_{i}^{\sigma}$.
We introduce an axiom on the abstracting power of a rule:
\begin{enumerate}
	\item [[Constant Abstraction (CA)]] A rule $R$ has fixed structure abstraction power.
	Namely, for all $(R,K) \in \KR$, 
	\[|V(K)| - |V(R(K))| + |E(K)| - |E(R(K))| = constant \; \delta_{R}^{\sigma}\]
\end{enumerate}

To model the changes to automata of elements produced by applying $R_i$ at time $j$ to graph $K_s$, we use 
\[\sum_{x \in K_s}\|P_x[j]-P_x[j-1]\| = \delta_{i,s}^p[j]\]
The extent by which the parameters is increased or decreased is a measure of abstraction. 
Here, $\|\cdot\|$ is some norm.


Note that the parameter gain $\delta_{i,s}^p$ is time-dependent since the parameters evolve with time.
We can now define the abstraction power of a rule $R_i$ as the sum of the two: $\delta_{i}=  \delta_{i,s}^{p}[j]+\delta_{i}^{\sigma}$.
Finally, at time $j$, the abstraction gain $\delta[j]$ is given by
\begin{equation}
\delta[j] = \sum_{(i,s) \in \KR}a_{isj}(\delta_{i}^{\sigma} + \delta_{i,s}^p[j]) = \sum_{(i,s) \in \KR}a_{isj}(\delta_{i}^{\sigma} + \sum_{x \in K_s}\|P_x[j]-P_x[j-1]\|)
\end{equation}
(Only one of the summands sums to 1, the others to 0, since only one rule is applied in a given step $j$).


\section{Interdependence between rules}
We can now express the interdependence between the rules. 

The data of the problem is:
\begin{itemize}
	\item the graph $G = (V,E)$.
	\item The rules $\Rc = \{R_1,\ldots,R_m\}$ and associated subgraphs $H_{R_i}$
	\item For each $H_{R}$, the list of integers uniquely identifying its vertices and edges, roles($H_R$)
	\item The initial values of the parameter vectors $P_x[0]$
	\item The matrices $B_i,C_i$ for each rule $R_i$
	\item The preserved vertices and edges $V_\phi$ and $E_\phi$
	\item the induced subgraphs $\Kc = \{K_1,\ldots,K_S\}$.
	\item for each $K_s$, the subgraph $H_R$ it is isomorphic to.
	\item for each $R_i$, the list of subgraphs it applies to $\Kc(i)$
	\item for each pair $R_i,K_s$, the set $D(i,s)$ of subgraphs destroyed by an application of $R_i$ to $K_s$	
	\item for each $R_i$, $\pi(i)$ = list of rules preceding $R_i$.
	\item The constants $\delta_{R}$.
	\item $f_{ik}$ functions for parameter updates
\end{itemize}
%\item $\rho(x)$ for each $x\in V\cup E$. $\rho(x)$ is a finite set of integer triplets $(k,i,s)$ where $s \leq |\Kc|$, $k \leq |V(K_s)\cup E(K_s)|$, $i \leq |\Rc|$. This can be induced from the above data.

The decision variables are
\begin{itemize}
	\item The binaries $a_{isj}$
\end{itemize}

1) First, we need to express that only one rule is applied at a step, and to one stage:
\begin{equation}
\label{eq:oneruleinstepArithm}
\forall j, \; \sum_{i,s}a_{isj}  \leq 1\; \forall j,k,v
\end{equation}

6) We need to express that a rule is applied at every step
\begin{equation}
\label{eq:oneruleeverystepArithm}
\forall j\; \sum_{i,s}a_{isj} \geq 1 
\end{equation}

Note that these can be combined into:
\begin{equation}
\label{eq:exactlyoneArithm}
\forall j\; \sum_{i,s \in \Kc(i)}a_{isj} = 1 
\end{equation}

2) A rule is applied only on a subgraph subject to it:
\begin{equation}
\label{eq:ruleongraphArithm}
\forall j,i,s \notin \Kc(i): a_{isj} = 0
\end{equation}

8) A rule is applied to a graph only if its parameters satisfy the rule condition. We reformulate and then arithmetize \eqref{eq:applicableR} as follows
\begin{eqnarray}
\label{eq:applicableRArithm}
&& \forall j,i,K \equiv H_{R_i}, R_i(K) \textrm{ is defined only if } P_K[j] \models \phi_{R_i}
\nonumber
\\
\Leftrightarrow && \forall j,i,s, a_{isj}=1 \implies B_iP_{K_s}[j]-C_i \leq 0 
\nonumber 
\\
\Leftrightarrow && \forall j,i,s, a_{isj} (B_iP_{K_s}[j]-C_i) \leq 0 
\end{eqnarray}

4) Rule effect: We need to express rule disabling by structure change, so we arithmetize \eqref{eq:disabling}.
\begin{equation}
\label{eq:disablingStructArithm}
\forall i,j,s \; a_{isj}\left(\sum_{i',j'>j,s' \in D(i)}a_{i's'j'}\right) =0
\end{equation}

9) Rule effect: Parameter updates were already arithmetized above in \eqref{eq:paramUpdate}. 

7) We need to express rule precedence, namely that some rules $R_i$ require that other rules from a set $\pi(i)$ be applied first:
\begin{eqnarray}
\label{eq:precedenceArithm}
&& \forall i,j,s, a_{isj}=1 \implies a_{i'j's'}=1 \textrm{ for some } j'<j,s',\forall i'\in \pi(i)
\nonumber
\\
\Leftrightarrow && \forall i,j,s, a_{isj}=1 \implies \sum_{j'<j,s'}\left(\Pi_{i'\in \pi(i)}a_{i'j's'}\right) \geq 1 
\nonumber
\\
\Leftrightarrow && \forall i \sum_{j,s}a_{isj} \geq 1 \implies \sum_{j'<j,s'}\left(\Pi_{i'\in \pi(i)}a_{i'j's'}\right) \geq 1 
\nonumber
\\
\Leftrightarrow && \forall i, \sum_{j,s}\left(a_{isj} + \sum_{j'<j,s'}\left(\Pi_{i'\in \pi(i)}a_{i'j's'}\right)\right) \neq 1 
\end{eqnarray}

10) We need to express model appropriateness, so we repeat \eqref{eq:Vphi}
\begin{equation}
\label{eq:VphiArithm}
\forall j, \forall i, \forall s \; s.t. \; K_s \cap (V_\phi \cup E_\phi)\neq \emptyset, a_{isj}=0
\end{equation}
Note we don't need to arithmetize it since its elements are static and known beforehand.

\section{The most abstract model that is compatible with the requirement}

Finally, we define the optimization whose solution gives the most abstract model subject to preserving the sets $V_\phi$ and $E_\phi$.
The optimization is given by the following integer linear program over the decision variables

\begin{eqnarray}
\max_{a_{isj}} && \sum_{j \leq N} \delta[j] =\sum_{j \leq N} \left(\sum_{i,s}a_{isj}\delta_{R_i}\right)
\\
\textrm{s.t. }&& \eqref{eq:oneruleinstepArithm}-\eqref{eq:VphiArithm}
\end{eqnarray}


\end{document}