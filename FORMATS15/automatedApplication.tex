\subsection{Automatic rule application}
\label{automatedApplication}

Because we start from a set of initial models, and each one may be abstracted using a number of abstraction rules, we have a choice of which rules to apply to which models, and the order in which to apply them.
Depending on which rule is applied when, we end up with different abstract models.
\emph{We seek a rule application order which yields the most abstract model that is still appropriate for the requirement $\phi$}.
In this section we propose a measure of abstractness, and sketch how an optimal order of application can be found as the solution of a Mixed-Integer Nonlinear Program (MINLP).
Due to space limitations the following discussion omits certain details which can be found in the technical report. 

First, we observe that when a rule is applied to a labeled graph $G=(V,E,A)$, it either decreases the number of vertices, or the number of edges, or it enlarges the invariant and/or guard sets of some of the automata (by manipulating the parameters as done, e.g., by rule R4).
Let $\{R_i\}_{i=1}^n$ be the set of abstraction rules, $\{G_s\}_{s=1}^m$ be the initial set of labeled graph heart models, and assume a pre-fixed maximum number of rule applications $N \geq 1$.
We define a measure of the abstraction power $\alpha(R_i,G_s)$ of a rule $R_i$ applied to graph $G_s$ to be
\[\alpha(R_i,G_s) = |V(G_s)| - |V(R_i(G_s))| + |E(G_s)| - |E(R_i(G_s))| + \sum_{x \in \Ec(R_i(G_s))}|\theta^x - R_i(\theta^x)|\] 
Intuitively, a rule $R$ is more abstract than rule $R'$ if it removes more elements or enlarges the parameter ranges more.
Then if we apply rules $R_{i_1}, R_{i_2},\ldots, R_{i_N}$ in that order to the set $\{G_i\}$, the abstractness of the final model is given by given by
$\alpha = \sum_j \alpha(R_{i_j},G_{i_j})$
where $G_{i_j}$ is the graph to which $R_{i_j}$ was applied.

Rule application must respect the following constraints, all of which can be encoded as the constraints of a MINLP:
\begin{enumerate}
	\item Only one rule can be applied at a time.
	\item A rule can only be applied to a (sub)graph with a given structure, and whose parameters respect certain conditions.
	\item When a rule is applied to a subgraph, it may disable future rule applications to this subgraph, either because a) it changed the subgraph's structure, or b) it updated the subgraphs's parameters.
\end{enumerate}
We define binary variables $a_{isj}$ such that $a_{isj} = 1$ iff at the $j^{th}$ time step, we apply rule $R_i$ to graph $G_s$.
The first constraint, for example, can be encoded as
\[\forall j=1,\ldots, N, \sum_{i,s}a_{isj} \leq 1\]

Proceeding in this manner, we formulate a MINLP whose solution gives an optimal order of rule application, in the sense of maximizing $\alpha$.
Then model-checking can be performed on this most abstract model returned by the optimization.