\section{Composing STORMED systems}
\label{sec:compositionality}
The results in this section and the next apply to STORMED systems in general, including those with time-unbounded operation.
We write $[m] = \{1,\ldots,m\}$.

We show that the parallel composition of SHS is still a SHS.
Recall that $\theta_{\ell}(t;x)$ is the flow starting at $(\ell,x)$.
Given hybrid systems $\Sys_1,\ldots,\Sys_m$, their parallel composition $\Sys = \Sys_1 || \ldots ||\Sys_m$ is defined in the usual way:
$\Sys.\stSet = \Pi_i \stSet_i$,
$\Sys.\modeSet = \Pi_i \modeSet_i$,
$\Sys.\hsSet_0 =\Pi_i \hsSet_{0_i}$,
$Inv(\ell) = \Pi_{i}Inv_i(\ell_i)$,
$\theta_{\ell}(x,t)= [\theta_{\ell_1}^1(x_1,t)(t),\ldots,\theta_{\ell_m}^m(x_m,t)(t) ]^T$.
The system jumps if any of its subsystems jumps, so its guard sets are of the form 
$D_1\times\ldots \times D_m$ where for at least one $i$, $D_i$ is a guard of $\Sys_i$, and for the rest $D_j =\stSet_j$.
When a guard of a subsystem is satisfied, the state of that subsystem is reset according to its reset map.
Formally, the guards are made disjoint to avoid non-determinism.

In general $\Sys$ is not separable: indeed for any candidate value of $d_{min}$, one could find a transition $(i,j)$ of $\Sys$ due to, say, a jump of $\Sys_1$, s.t. at that moment $x_2$ is closer than $d_{min}$ to its own guard. 
This causes $\Sys$ to jump $j \rightarrow k$ without having traveled the requisite minimum distance, thus violating the separability of $\reset_{ij}(\guard_{ij})$ and $\guard_{jk}$.
Therefore we need to impose an extra condition on minimum separability \emph{across} sub-systems.
This extra condition is satisfied by $\Sys_{ICD}||\Sys_{CA}$ as can be seen by inspection.
%To obtain separability, we augment the state of $\Sys$ with the variable $r$ that keeps track of the current mode.
%Formally, given a hybrid system $\Sys$, define $\widehat{\Sys}$ with state space $\widehat{\stSet} = \Sys.\hsSet$.
%For $\hat{x}  = (r ,x)\in \hat{\stSet}$, $r (t)$ tracks the current mode.
%Its flow is given by $r(t) = \mode$ if the current mode of $\Sys$ is $\mode $.
%It is easy to verify the semantics are not modified by this augmentation, and that we have (proof in appendix):
%\begin{lemma}
%	\label{lemma:separable}
%Let $\SHS = (\Sys, \ldots)$ be a STORMED hybrid system with guard separation $d_{min}$.
%	Let $\widehat{\Sys}$ be the corresponding augmented system.
%	Then $\widehat{\Sys}$ has separable guards and the separation is at least 
%	$\sqrt{\min (1,d_{min}^2)}$
%\end{lemma}
\begin{thm}
	\label{thm:SHS composition}		
	Let $\SHS_i = (\Sys_i, \Ac, \phi_i,b_i^-,b_i^+, d_{min,i}, \varepsilon_i, \zeta_i)$, $i=1,\ldots,m$ be deterministic SHS 
	defined using the same underlying o-minimal structure, 
	and where each state space $\stSet_i$ is bounded by $B_{X_i}$.
	Assume that the following \textbf{Collection Separability} condition holds: 
	for all $i,j \leq m $ there exists $d_{min}^{ij}>0$ s.t.
	$x_i \in G^i_e \implies d(x_j,G^j_{e'}))>d_{min}^{ij}$ for all $e'\in E_i$ where $G^k_e$ is a guard of $\SHS_k$ on edge $e$ and $E_k$ is the edge set of $\SHS_k$.
	\\
	Define parallel composition $\SHS = (\Sys, \Ac, \phi,b^-,b^+, d_{min}, \varepsilon, \zeta)$ where
	$\Sys = \Sys_1 || \ldots ||\Sys_m$,	
	$\phi = (\phi_1,\ldots,\phi_m)^T \in \Re^{mn}$,
	$b_i^- = \inf_{x \in \stSet} \phi\cdot x$,
	$b_i^+ = \sup_{x \in \stSet} \phi\cdot x$,
	$\varepsilon = \min(\min_i \varepsilon_i, \min_i \frac{\zeta_i}{B_{X_i}})$,
	\[d_{min} = \min_{I\subset [m]} (\min_{i\in I}d_{min}^i, \min_{i\in I ,j \in [m]\setminus I }d_{min}^{ij})\]	
	$\zeta = \min_i \zeta_i$.
Then $\SHS$ is STORMED.
\end{thm}

\begin{prf}
\textbf{(S)} In $\Sys$, let $y=(y_1,\ldots,y_m)=\reset_e((x_1,\ldots,x_m))$ and assume without loss of generality that it was $\Sys_1$ that caused the jump.
Thus $y_j=x_j,j >1$.
Write $e=(\mode,\mode ')$.
By hypothesis $d(y_j,G^j_{e_j})>d_{min}^{1j}$ for all $j>1,e_j\in E_j$, 
and by separability of $\Sys_1$ $d(y_1,G^1_{e_1}) > d_{min}^1$ so $d(y,G_{\mode', \mode ''}) > \min (d_{min}^1, \min_{j>1}d_{min}^{1j}) $ for any guard leading out of $\mode'$, and we have separability.
The argument can be repeated for any set $I \subset [m]$ of systems jumping simultaneously.
\\
\textbf{(T)}: let $\mode = (\mode_1,\ldots,\mode_m)$ and $x = (x_1,\ldots,x_m)$.
The $\widehat{\Sys}$ flow $\widehat{\theta}_\mode(t;\hat{x}) = (\mode(t),\theta_\mode(t;x))$ is TISG because the component flows $\theta^i_{\mode_i}(t;x_i)$ and $\mode(t) \equiv \mode$ are TISG.
\\
\textbf{(O)} The cartesian product of definable sets is definable, so the system $\widehat{\Sys}$ is o-minimal.
\\
\textbf{(RM)} First we show that resets of $\Sys$ are monotonic, then that the resets of $\widehat{\Sys}$ are monotonic.
Let $p,q \in \modeSet$ be two modes of $\Sys$, $p\neq q$.

\underline{Case 1: $\Sys$ jumps $p \rightarrow p$}.
So any subsystem $\Sys_i$ either jumped $p_i \rightarrow p_i$ or didn't jump at all.
If $x^+ = x^-$, then (RM) is satisfied.
Else, define $\phi \defeq (\phi_1,\ldots,\phi_m) \in \Re^{n\cdot m}$, where $\phi_i$ is the $\phi$ vector of system $\Sys_i$.
Then $\phi \cdot (x^+ - x^-) = \sum_{i\in K}\phi_i\cdot(x_i^+ - x_i^-)$, 
where $K \subset [m]$ is the set of indices of sub-systems that jumped with $x_i^- \neq x_i^+$.
Note that $K$ depends on $x^-,x^+$.
%
For all $x^-,x^+$ pairs (and so for all $K$) 
$\sum_{i\in K} \zeta_i \geq \min_{i \in [m]}\zeta_i \defeq \zeta > 0$.
So by (RM) for each $\Sys_i$,
\begin{equation*}
\label{eq:parallel rm1}
\phi \cdot (x^+ - x^-)\geq  \sum_{i\in K}\phi_i\cdot(x_i^+ - x_i^-) \geq \sum_{i\in K}\zeta_i \geq \zeta > 0
\end{equation*}
Thus (RM) is satisfied.

\underline{Case 2: $\Sys$ jumps $p \rightarrow q$}.
At least one syb-system $\Sys_i$ jumped $p_i \rightarrow q_i \neq p_i$.
Then 
$\phi \cdot (x^+ - x^-) = \sum_{i\in [m]}\phi_i\cdot(x_i^+ - x_i^-) = \sum_{i\in K}\phi_i\cdot(x_i^+ - x_i^-)$,
where $K = K_= \cup K_{\neq} \subset [m]$ and 
$K_=$ is the index set of subsystems that jumped $p_i \rightarrow p_i$ with $x_i^+ \neq x_i^-$, 
and $K_{\neq}$ is the index set of subsystems that jumped $p_i \rightarrow q_i \neq p_i$ with $x_i^+ \neq x_i^-$.
Subsystems that didn't jump or jumped without changing their state don't contribute to the sum.
Note that $K_=,K_{\neq}$ depend on $x^-,x^+$.
So we have
$\phi \cdot (x^+ - x^-) \geq \sum_{i\in K_{\neq}}\varepsilon_i ||x_i^+-x_i^-|| + \sum_{i\in K_=} \zeta_i$.

For all $X_i$, $||x_i^+-x_i^-|| \leq B_{X_i}$ for all $x_i^-,x_i^+ \in X_i$.
Therefore $\zeta_i \frac{||x_i^+-x_i^-||}{B_{X_i}} \leq \zeta_i$ for all $i\in K$.
So 
\begin{eqnarray*}
&\;&\sum_{i\in K_{\neq}} (\min_{i\in [m]}\varepsilon_i) ||x_i^+-x_i^-|| + \sum_{i\in K_=} \frac{\zeta_i}{B_{X_i}}||x_i^+-x_i^-||
\\
&\geq& \sum_{i\in K_{\neq}} (\min_{i\in [m]}\varepsilon_i) ||x_i^+-x_i^-|| + \sum_{i\in K_=} (\min_{i\in [m]}\frac{\zeta_i}{B_{X_i}})||x_i^+-x_i^-||
\end{eqnarray*}
Let $\varepsilon \defeq \min(\min_i \varepsilon_i, \min_i \frac{\zeta_i}{B_{X_i}})$.
Then 
\begin{equation*}
\phi \cdot (x^+ - x^-) \geq \sum_{i\in K} \varepsilon ||x_i^+-x_i^-|| \geq \varepsilon ||x^+-x^-||
\end{equation*}
So $\Sys$ has monotonic resets.

What about $\widehat{\Sys}$?
Consider again the case where $\Sys$ jumps $p\rightarrow q\neq p$, the other case being trivial.
Fix an arbitrary $\phi_d \leq - \varepsilon$.
Let $\hx = (\stPt,\mode)$ be the state of $\widehat{\Sys}$, and let $\hat{\phi} = (\phi,\phi_d)$ with $\phi= (\phi_1,\ldots,\phi_m)$ as defined above.
Then $\hat{\phi} \cdot (\hx^+ - \hx^-) = \phi \cdot (x^+ - x^-) + \phi_d \cdot (q-p) \geq \varepsilon ||x_i^+-x_i^-|| + \phi_d (1-|\modeSet|) \geq  \varepsilon ||x_i^+-x_i^-|| + \varepsilon (|\modeSet|-1) \geq  \varepsilon ||x_i^+-x_i^-|| + \varepsilon |q-p|$ and the resets of $\widehat{\Sys}$ are monotonic.

The flows of $\widehat{\Sys}$ are also monotonic along $\hat{\phi} = (\phi,\phi_d)$.
Indeed for any $q \in \modeSet$,
$\hat{\phi}\cdot (\hat{\theta}_q(t+\tau;x) - \hat{\theta}_q(t;x)) = \phi\cdot (\theta_q(t+\tau;x) - \theta_q(t;x) ) + \phi_d \cdot (q-q) = \sum_{i=1}^{m}\phi_i \cdot (\theta^i_{q_i}(t+\tau;x_i) - \theta^i_{q_i}(t;x_i)) \geq \varepsilon_i ||(\theta^i_{q_i}(t+\tau;x_i) - \theta^i_{q_i}(t;x_i))|| \geq \varepsilon ||(\theta_{q}(t+\tau;x) - \theta_{q}(t;x))|| = \varepsilon ||\hat{\theta}_q(t+\tau;x) - \hat{\theta}_q(t;x)||$

\textbf{(ED)} By Prop. \ref{prop:ED}.
	\end{prf}