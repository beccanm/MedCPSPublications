\section{Arrhythmia detection}
\label{sec:discriminators}
Disturbances of the heart's normal rhythm are known as \emph{arrhythmias}.
\emph{\ac{VT}} is an example of an arrhythmia originating in the ventricles, in which the ventricles spontaneously beat at a very high rate.
If the \ac{VT} is sustained, or degenerates into \ac{VF}, it is fatal within seconds.
An abnormally fast heart rate that originates in the atria is referred to as a \emph{\ac{SVT}}.
This is a diseased but non-fatal condition.
In what follows, we will refer to sustained \ac{VT} and \ac{VF} together as \ac{VT}.
\emph{The ICD's main task is to discriminate \ac{VT} from \ac{SVT} and deliver therapy to the former}.

Most \ac{VT}/\ac{SVT} detection algorithms found in ICDs today are composed of individual \emph{discriminators}. 
A discriminator is a software function whose task is to decide whether the current arrhythmia is SVT or VT. 
No one discriminator can fully distinguish between SVT and VT.
%: the two may have similar rates but different EGM morphology, or similar morphology and different rates, or similar morphology and rate but different onset characteristics.
Thus a detection algorithm is often a decision tree built using a number of discriminators \emph{running in parallel}.
The detection algorithm of Boston Scientific is shown in Fig. \ref{fig:bsc detection} \cite{compass}.
%\acp{VT} and \acp{SVT} can share similar characteristics and might even occur simultaneously, so an \ac{SVT} is often mis-detected as a \ac{VT} by the \ac{ICD}. 
%This is problematic because \ac{VT} therapy consists of low and high energy electric shocks of 30-40 Joules ($\sim$800V) delivered directly to the heart, which causes severe stress to the patient, and might even be linked to increased morbidity \cite{shock_mortality}.
%Therefore, one of the biggest challenges for ICDs is to discriminate between fatal \acp{VT} that require a shock, and non-fatal \acp{SVT} that should not be shocked.% \cite{Ellenbogen11_Pacingbook}.
%The detection algorithms of various manufacturers' ICDs share common discriminators: for example, they all have a Rate discriminator, some version of a Morphology discriminator, a Stability discriminator, etc. \cite{compass}\cite{med-book},\cite{SJM13},\cite{Biotronik13_Lumax}.
%While the details of these software functions differ, their outlines and goals are similar. 
We have modeled each discriminator in this detection algorithm as a STORMED hybrid system.
The algorithm itself is then a hybrid system.
\textbf{The ICD system is thus 
$\mathbf{\Sys_{ICD} = \Sys_{Sense}||\Sys_{Detection-Algo}}$ where $\mathbf{\Sys_{Detection-Algo}}$ is the parallel composition of the discriminator systems}.
In what follows, we present three of these discriminators we modeled, which are found in most ICDs and model them as hybrid systems, and prove they are STORMED.
%We also give one example of how these discriminators and others are brought together in a decision tree to make up a detection algorithm.
\begin{figure}[t]
	\centering
	\includegraphics[scale=0.4]{figures/BS_det_nosig}
	\vspace{-10pt}
	\caption{Boston Scientific's detection algorithm}
	\vspace{-10pt}
	\label{fig:bsc detection}
\end{figure}

\subsection{Three Consecutive Fast Intervals}
\label{sec:tcfi}
\begin{figure}[t]
	\centering
	\includegraphics[scale=0.3]{figures/TCFI}
		\vspace{-10pt}
	\caption{Three Consecutive Fast Intervals $\Sys_{TCFI}$}
	\label{fig:tcfi}
\end{figure}
Our first module simply detects whether three consecutive fast intervals have occurred, where `fast' means the interval length, measured between 2 consecutive peaks on the \ac{EGM} signal, is shorter than some pre-set amount.
See Fig. \ref{fig:tcfi}.
States $t$ and $t_p$ are clocks as before.
The vector $L_3$ is three-dimensional, and stores the values of the last three intervals.
The event VEvent? is shorthand for the transition $y(t) \geq Th$ being taken by the $\Sys_{Sense}$ automaton.
In other words, it indicates a ventricular event.
Then $L_3$ gets reset to $L_3^+ = (z_1,z_2,z_3)^+ \defeq \text{Circulate}(L_3,t-t_p)$ where
\begin{equation}
L_3^+ = 
\left(\begin{matrix}
z_2\\z_3\\t-t_{p}\\
\end{matrix}
\right)
=
\left(\begin{matrix}
0 & 1 & 0\\0& 0& 1\\0& 0 &0
\end{matrix}
\right) L_3 + 
\left(\begin{matrix}
0\\0\\t-t_p
\end{matrix}
\right)
\end{equation}
%
\begin{lemma}
	$\Sys_{TCFI}$ is STORMED.
\end{lemma}
\begin{prf}
We show that the reset are monotonic - the other properties are easily checked.
For reset monotonicity, we invoke the fact that there is a minimum beat-to-beat separation: heartbeats can't follow one another with vanishingly small delays. 
In other words, there exists $m>0$ such that $t - t_p^- > m$.
Similarly, there's a maximum delay between two heartbeats, call it $B$.
Now, we seek a vector $\phi \in \Re^5$ s.t. 
\begin{equation}
\label{eq:tcfi resets}
\phi \cdot \left(\begin{matrix}
t-t\\t - t_p\\L_3^+ - L_3\\
\end{matrix}
\right) = \phi_p(t-t_p) +\phi_{L_3} \cdot \underbrace{\left(\begin{matrix}
z_2-z_1\\z_3-z_2\\t-t_p - z_3\\
\end{matrix}
\right)}_{\delta} \stackrel{Want}{\geq} \zeta > 0
\end{equation} 
Now $|\delta|$ is upper bounded by $\sqrt{3\cdot (2B)^2}$ since each element is the difference of intervals shorter than $B$.
Also, $t-t_p^- > m > 0$.
So choose $\phi_{L_3} = (\phi_{z,1},\phi_{z,2},\phi_{z,3})>0$ element-wise.
\eqref{eq:tcfi resets} is satisfied if the following stronger inequality is satisfied, which can be achieved by an appropriate choice of $\phi_{z,i}$: \;
$\phi_p m \geq \zeta + \sqrt{12B^2}\sum_1^3\phi_{z,i}$
\end{prf}

\subsection{Vector Timing Correlation}
\label{sec:VTC}
\begin{figure}[t]
	\centering
	\vspace{-15pt}
	\includegraphics[scale=0.3]{figures/VTCEGMCompare}
	\caption{\small \acp{EGM} of different origin have different morphologies.}
	\label{fig:egmmorphology}
	\vspace{-10pt}
\end{figure}
%\caption{\small \acp{EGM} of different origin have different morphologies, while \acp{EGM} of same origin have very similar morphologies.}
It has been clinically observed that a depolarization wave originating in the ventricles (as produced during \ac{VT} for example) will in general produce a different \ac{EGM} morphology than a wave originating in the atria (as produced during \ac{SVT}) \cite{compass}.
See Fig. \ref{fig:egmmorphology}.
%
A morphology discriminator measures the correlation between the morphology of the current \ac{EGM} and that of a stored \emph{template} \ac{EGM} acquired during normal sinus rhythm.
If the correlation is above a pre-set threshold for a minimum number of beats, then this is an indication that the current arrhythmia is supraventricular in origin.
Otherwise, it might be of ventricular origin.

Boston Scientific's implementation of a morphology discriminator is called Vector and Timing Correlation (VTC).
VTC first samples 8 \emph{fiducial} points $\egm_i,i=1,\ldots,8$ on the current \ac{EGM} $\egm$ at pre-defined time instants.
Let $\egm_{m,i}$ be the corresponding points on the template \ac{EGM}.
A simple 0-shift correlation $\rho_{new}$ is calculated between the two sequences. 
If 3 out of the last 10 calculated correlation values exceed the threshold, then \ac{SVT} is decided and therapy is withheld.

The system of Fig. \ref{fig:HVTC} implements the VTC discriminator.
As before, $t$ is a local clock.
$\mu$ accumulates the values of the current \ac{EGM}, $\alpha$ accumulates the product $\egm_i \egm_{m,i}$, 
$\beta$ accumulates $\egm_i^2$.
State $w$ is an auxiliary state we need to establish the STORMED property.
$\vec{\nu}$ is a 10D binary vector: $\nu_i = -1$ if the $i^{th}$ correlation value fell below the threshold, and is $+1$ otherwise.
$L_3$ is the state of $\Sys_{TCFI}$: the guard condition $L_3 \leq th$ indicates that all its entries have values less than the tachycardia threshold, which is when $\Sys_{VTC}$ starts computing.
$WindowEnds$ indicates the `end' of an \ac{EGM}, measured as a window around the peak sensed by $\Sys_{Sense}$.  
%
\begin{figure}[t]
\centering
\includegraphics[scale=0.325]{figures/VTC1v2}
\vspace{-10pt}
\caption{VTC calculation. $iT_s$ is the sampling time for the $i${th} fiducial point, $i=1,\ldots,8$. $R2_{1},\ldots,R2_{8}$ are the corresponding resets. For clarity of the figure, 8 transitions are represented on the same edge.}
\vspace{-10pt}
\label{fig:HVTC}
\end{figure}
%
\begin{lemma}
	\label{lemma:vtc}
	$\Sys_{VTC}$ is STORMED.
	\end{lemma}

\subsection{Stability discrimination}
\label{sec:stability}
%\begin{figure}[t]
%	\centering
%	\includegraphics[scale=0.35]{figures/EGMStability}
%	\caption{Examples of a unstable rhythm (top) and stable rhythm (bottom).}
%	\label{fig:stable unstable}
%\end{figure}
\emph{Stability} refers to the variability of the peak-to-peak cycle length.
A rhythm with large variability (above a pre-defined threshold) is said to be \emph{unstable}, and is called stable otherwise.
The Stability discriminator is used to distinguish between atrial fibrillation, which is usually unstable, and \ac{VT}, which is usually stable.
%Atrial fibrillation, which is usually unstable, might induce a high ventricular rate .
%A \ac{VT}, on the other hand, is usually stable.
%Therefore this is a useful discriminator.

The Stability discriminator shown in Fig. \ref{fig:Hstab} simply calculates the variance of the cycle length over a fixed period called a Duration (measured in seconds).
Let $DL \geq 0$ be the Duration length.
\yhl{The events $DurationBegins?$ and $DurationEnds?$ indicate the transitions} of a simple system that measures the lapse of one Duration (not shown here).
State $t$ is a clock, $L_1$ accumulates the sum of interval lengths (and will be used to compute the average length), 
$L_2$ accumulates the squares of interval lengths,
and $\kappa$ is a counter that counts the number of accumulated beats.
$\sigma_2$ is assigned the value of the variance given by $\frac{1}{\kappa}[L_2 - L_1^2/\kappa]$
\begin{figure}[t]
	\centering
	\includegraphics[scale=0.3]{figures/stability1v2}
	\vspace{-10pt}
	\caption{Stability discriminator.}
	\vspace{-10pt}
	\label{fig:Hstab}
\end{figure}

\begin{lemma}
	\label{lemma:stability}
	$\Sys_{Stab}$ is STORMED.	
\end{lemma}

Now that each system was shown to be STORMED, it remains to establish that their parallel composition is STORMED.
This result does not hold in general - Thm.~\ref{thm:SHS composition} gives conditions under which parallel composition respects the STORMED property.
Intuitively, we require that whenever a sub-collection of the systems jumps, the remaining systems that did not jump are separated from all of their respective guards by a uniform distance.
This is a requirement that can be shown to hold for our systems by modeling various minimal delays in the systems' operation. 
%For example, when a $VEvent?$ is issued by $\Sys_{Sense}$, $\Sys_{VTC}$ does not jump and will wait at least until the sampling time of the next fiducial point to make a transition.
%Or, when an atrial cell fires (in $\Sys_{CA}$), we model a minimal delay between it and all other cells that do not fire simultaneously.
We may now state:
\begin{thm}
Consider the collection of systems $\Sys_{CA}$, $\Sys_{ICD} = \Sys_{Sense} || \Sys_{Detection-Algo}$ where the latter is the parallel composition of the discriminator systems.
This collection satisfies the hypotheses of Thm. \ref{thm:SHS composition} (Section \ref{sec:compositionality}) and therefore the parallel system  $\Sys_{CA} || \Sys_{ICD}$ is STORMED and has a finite bisimulation.
\end{thm}


