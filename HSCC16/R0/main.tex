% THIS IS SIGPROC-SP.TEX - VERSION 3.1
% WORKS WITH V3.2SP OF ACM_PROC_ARTICLE-SP.CLS
% APRIL 2009
%
% It is an example file showing how to use the 'acm_proc_article-sp.cls' V3.2SP
% LaTeX2e document class file for Conference Proceedings submissions.
% ----------------------------------------------------------------------------------------------------------------
% This .tex file (and associated .cls V3.2SP) *DOES NOT* produce:
%       1) The Permission Statement
%       2) The Conference (location) Info information
%       3) The Copyright Line with ACM data
%       4) Page numbering
% ---------------------------------------------------------------------------------------------------------------
% It is an example which *does* use the .bib file (from which the .bbl file
% is produced).
% REMEMBER HOWEVER: After having produced the .bbl file,
% and prior to final submission,
% you need to 'insert'  your .bbl file into your source .tex file so as to provide
% ONE 'self-contained' source file.
%
% Questions regarding SIGS should be sent to
% Adrienne Griscti ---> griscti@acm.org
%
% Questions/suggestions regarding the guidelines, .tex and .cls files, etc. to
% Gerald Murray ---> murray@hq.acm.org
%
% For tracking purposes - this is V3.1SP - APRIL 2009


\documentclass{acm_proc_article-sp} 

\usepackage{fainekos-macros}
\usepackage{color}
\usepackage[english]{babel}
\usepackage{acronym}
%\usepackage{amsmath}
%\usepackage{graphicx}
%\usepackage[colorinlistoftodos,obeyFinal]{todonotes}
%\usepackage{wrapfig}
%\usepackage{makeidx}  % allows for indexgeneration
%\usepackage{bibspacing}
%\usepackage{fancyvrb}
%\usepackage{amsfonts}
\usepackage{url}
\usepackage[ruled]{algorithm}
\usepackage{algpseudocode}
\usepackage{paralist}

%\setlength{\bibspacing}{\baselineskip}
%\usepackage[tight,footnotesize]{subfigure}

\newcommand{\tikzcircle}[2][black,fill=black]{\tikz[baseline=-0.5ex]\draw[#1,radius=#2] (0,0) circle ;}
\newcommand{\slabel}{\sigma}
\newcommand{\tlabel}{\slabel_\chi}
\newcommand{\ulabel}{\slabel_u}
\newcommand{\labelSet}{\Sigma}
\newcommand{\ulabelSet}{\labelSet_u}
\newcommand{\bslabel}{\bar{\slabel}}
\newcommand{\tlabelSet}{\labelSet_\chi}
\newcommand{\btlabel}{\bar{\tlabel}}
\newcommand{\trans}[1]{\xrightarrow{#1}}
\newcommand{\out}[1]{\left\langle #1\right\rangle}
\newcommand{\Omapeps}{\Oc_\varepsilon}
\newcommand{\hsOut}{f}
\newcommand{\CDtau}{\textbf{CD}_\tau(\Sys_1 , \Sys_2)}
\newcommand{\simu}{\mathcal{S}}
\newcommand{\bisimu}{\mathcal{B}}
\newcommand{\Ft}{\Fc_t}
\newcommand{\Fd}{\Fc_d}
\newcommand{\partition}{\mathcal{P}}
\newcommand{\SHS}{\Sigma}
\newcommand{\simue}{\simu^\varepsilon}
\newcommand{\hx}{\hat{\stPt}}
\newcommand{\Cc}{\mathcal{C}}
\newcommand{\egm}{s}
\acrodef{NSR}{Normal Sinus Rhythm}
\acrodef{AP}{Action Potential}
\acrodef{ICD}{Implantable Cardioverter Defibrillator}
\acrodef{SVT}{SupraVentricular Tachycardia}
\acrodef{VT}{Ventricular Tachycardia}
\acrodef{VF}{Ventricular Fibrillation}
\acrodef{SVT}{SupraVentricular Tachycardia}
\acrodef{EGM}{electrogram}
\acrodef{CA}{cellular automata}

\newboolean{REPORT}
\setboolean{REPORT}{FALSE}
\linespread{0.97}

\begin{document}

\title{Model Checking Implantable Cardioverter Defibrillators}

\author{
% 1st. author
\alignauthor
Houssam Abbas, Kuk Jin Jang, Zhihao Jiang, Rahul Mangharam\\
       \affaddr{Department of Electrical and Systems Engineering}\\
       \affaddr{University of Pennsylvania, Philadelphia, PA, USA}\\
       \email{\{habbas,  jangkj, zhihaoj, rahulm\}@seas.upenn.edu}
}
% Just remember to make sure that the TOTAL number of authors
% is the number that will appear on the first page PLUS the
% number that will appear in the \additionalauthors section.

\maketitle
\begin{abstract}
Ventricular Fibrillation is a disorganized electrical excitation of the heart that results in inadequate blood flow to the body.
It usually ends in death within a few seconds.
The most common way to treat the symptoms of fibrillation is to implant a medical device, known as an \emph{Implantable Cardioverter Defibrillator} (ICD), in the patient's body.
Model-based verification can play a crucial role in ICD development.
In this paper, we build a hybrid system model of the human heart+ICD closed-loop system, and show that it admits a finite bisimulation by showing it to be a STORMED hybrid system.
In general, it may not be possible to compute the bisimulation.
We show that approximate reachability can yield a finite \emph{simulation} for STORMED systems, which improves on the existing verification procedure.
In the process, we show that certain compositions respect the STORMED property.
Thus it is possible to model check important formal properties of ICDs in a closed loop with the heart, such as delayed therapy, missed therapy, or inappropriately administered therapy. 
The results of this paper are theoretical, since no model checkers exist for STORMED systems. 
In future work we will implement a procedure for model checking the heart+ICD loop.
\end{abstract}

%Ventricular Fibrillation is a disorganized electrical excitation of the heart that results in inadequate blood flow to the body.
%It usually ends in death within a few seconds.
%The most common way to treat the symptoms of fibrillation is to implant a medical device, known as an Implantable Cardioverter Defibrillator (ICD), in the patient's body.
%Model-based verification can play a crucial role in ICD development.
%In this paper, we build a hybrid system model of the human heart+ICD closed-loop system, and show that it admits a finite bisimulation by showing it to be a STORMED hybrid system.
%In general, it may not be possible to compute the bisimulation.
%We show that approximate reachability can yield a finite \emph{simulation} for STORMED systems, which improves on the existing verification procedure.
%In the process, we show that certain compositions respect the STORMED property.
%Thus it is possible to model check important formal properties of ICDs in a closed loop with the heart, such as delayed therapy, missed therapy, or inappropriately administered therapy. 
%The results of this paper are theoretical, since no model checkers exist for STORMED systems. 
%In future work we will implement a procedure for model checking the heart+ICD loop.
%% A category with the (minimum) three required fields
%\category{H.4}{Information Systems Applications}{Miscellaneous}
%%A category including the fourth, optional field follows...
%\category{D.2.8}{Software Engineering}{Metrics}[complexity measures, performance measures]

%% A category with the (minimum) three required fields
%\category{H.5.1}{Computing methodologies}{Modeling and simulation}[Model development and analysis]
%%A category including the fourth, optional field follows...
%\category{L.2}{Applied computing}{Life and medical sciences}
%\terms{Medical devices, life-critical CPS, STORMED systems, heart modeling, ICDs}

%\section{Hit these points}

The spread of heart disease and its ocnsequences in terms of mortality

How common are ICDs (nb implants per month in us and world)

market size

current practice for design and verification

heart+device as a hybrid system, based on cellular automata augmented with differential equations in the cells

in this paper, we show that $H_{CA}$ falls in a decidable class: stormed systems, and that approximate methods can be used to build a finite simulation.

our contributions:\\
theoretical\\
- show that the parallel composition of stormed is still stormed
\\
- show that approximate algorithms can produce a finite simulation (previous results use exact algo to produce a finite bisumulation)
\\
- the approximate reachability algo is more broadly applicable than manual polynomial approximations required in Vladimerou. especially since we have MANY models that might yield different polynomial approximations.
\\
application:\\
- show that the common components of most ICDs can be modeled as stormed
- show that a popular and practically validated heart model is stormed
- and there have been no examples of "real" stormed systems. This is the first such example.

there are no model checkers for stormed systems:
QE is too costly. 
the reachability-based algo is conceptual.
next research is to develop MCs for SHS.

SpaceEx approximations are o-minimal

why not just reachability? because want to verify LTL properties.

Examples of LTL properties of interest

the fact that we have timers in the state means we can indirectly verify some MTL properties: give examples

by Pfaffian extensions, can do any resolution of VTC.

Continuous extension of VTC: o-minimal by Pfaffian

the components we present are common to BSC, SJM, Biotronic and some MEdtronic so these results are broadly applicable.


---------------------------------------------------
0. Intro

The spread of heart disease and its ocnsequences in terms of mortality

How common are ICDs (nb implants per month in us and world)

market size

current practice for design and verification

related work: brihaye thesis, 
event b for cellular automata, hao's timed automata, PDE models of flavio and natalia, kwiatowska parameter synthesis, verification of tolerant systems (prabhakar), discrete abstraction of HS paper by pappas et al,
Kwiatowska's Invariant Verification of Nonlinear Hybrid Automata Networks of Cardiac Cells,
Grosu et al approimate bisim for sodium channel dynamics,
Mery's Formal VHM cellular automata model,
Pappas and girard's TAC09 paper

I. Theory

define transition system

define simulation nad bisimu for TS

give generic reachability algo for computing a bisimulation

define HS

define transition system of HS

define simulation and bisimulation for HS as TS

define o-minimal structures

define SS

show that using apx reachability on continuous systems yields a simulation

show that this simulation is finite

show that spaceex approximatino sets are o-minimal 

If initial partition is given by $x,y \in P$ iff $x \models p \leftrightarrow y \models p$, this respects formulas

show that parallel composition of SS is a SS


II. Modeling

basic cardiac EP 

cellular automata model of the heart with diff equations: 
	
	- give model
	
	- validated by EPs working with it to learn and by in vivo validation
	
	- show STORMED

APD restitution can also be part, see tech report

VT and SVT discrimination

Overview of detection algorithm

Three Consecutive Fast Intervals: show STORMED

VTC discrete: show STORMED

VTC continuous: show STORMED

Stability: show STORMED

We're skipping others

III Examples of properties





\section{Introduction}%Problems regards to discrimination in IC
\label{sec:intro}
	\begin{figure}[t]
		\centering
		\vspace{-10pt}
		\includegraphics[scale=0.25]{figures/figTransResearchSpectrum.pdf}
		\vspace{-10pt}
		\caption{\small Bringing a device to market.
			Clinical trials are the last step before a new device's market approval.
			Model-based clinical trials will provide insight during planning and execution of clinical trials, leading to reduction in costs and increasing the chance of a successful trial.}
		\vspace{-10pt}
		\label{fig:spectrum}
	\end{figure}
	
The lives of millions of patients around the world depend on medical devices.
In the domain of cardiac devices, for example,
10,000 people in the U.S. receive an Implantable Cardioverter Defibrillator (a heart rhythm adjustment device) every month \cite{asktheicd}.
Clinical trials have presented evidence that patients implanted with ICDs have a mortality rate reduced by up to 31\% \cite{maditrit}.
Unfortunately, ICDs suffer from a high rate of \emph{inappropriate therapy}, which takes the form of unnecessary electric shocks or pulse sequences delivered to the heart.
Inappropriate therapy increases patient stress, reduces their quality of life, and is linked to increased morbidity \cite{shock_mortality}.
Depending on the particular ICD and its settings, the rates of inappropriate therapy range from 46\% to 62\% of all delivered therapy episodes \cite{GoldABBTB11_RIGHTresults}.
\mynote{SD}{This percentage is too high. Inappropriate therapies are typically not more than 20\%.\newline
	HA: this was calculated from RIGHT results. Another figure we could quote is Reported rates of inappropriate therapy range from 11 to 41\% (cited in RIGHT 2006).
	}
The closed loop formed by the organ (heart) and device (ICD) is an example of a life-critical Cyber-Physical System (CPS): the device's software is the cyber component, and the physiological phenomenon (e.g., cardiac rate and electrical activity) is the physical component.
%\headline{The technology in some of these devices combines hardware and software, each of which must be rigorously verified to be efficacious and safe.}
%\headline{In this paper we are concerned with \emph{closed-loop devices}.}
%Such a device is in a feedback loop with the organ(s) it effects (see Fig.\ref{fig:pacemaker}): an ICD for example monitors the heart rate, and delivers therapy to maintain an adequate rate.
%Another example is the artificial pancreas, which monitors blood glucose levels and delivers insulin to maintain safe glucose levels.

\emph{After the verification and testing effort is completed}, regulatory agencies like the F.D.A. require that the safety and efficacy of new devices be demonstrated in a \emph{\ac{CT}} (Fig.~\ref{fig:spectrum}).
%\footnote{In this paper, we always mean a randomized controlled trial, which is the type of trial described here.}
In a trial, a group of patients that are treated with the new device (this is the `intervention group') are compared to a group of patients who are treated with the current standard of care (e.g., a different device currently on the market; this is the `control group').
The objective is to see whether the different devices result in significantly different effects on the patients.
Clinical trials are major endeavors, involving physicians, patients, statisticians, clinical centers, companies and regulators, sometimes in several countries.
Late-phase trials can run for several years, and cost millions of dollars.
For example, a 2002 trial for stents lasted 2 years, enrolled 800 patients and cost \$10 to \$12 million and lasted 24 months \cite{Kaplan04_Cost}.
Trials also pose an inherent risk to the patients in the intervention group by exposing them to an unproven device.
Thus it is crucial that they be well planned, and rigorously executed.

In reality, any trial runs the risk of errors during its planning and execution stages, which can invalidate the results of the trial.
In this paper, we pose and propose an answer to the following question: \emph{how can modeling of CPS assist in the planning and execution of a clinical trial, so as to increase the chances of a successful trial}?
%The advent of computer models for various physiological functions defines a new convergence point for computer science and engineering with medicine.

Specifically, in this paper,
we demonstrate how  computer models can be used for early, affordable and reproducible testing of a clinical trial's premises and assumptions.
Model-based empirical validation of the premises reduces the risk of conducting a trial that fails to demonstrate the desired effect (typically, an improvement of new intervention over the control). 
We used the Rhythm ID Going Head to Head Trial (RIGHT) \cite{GoldABBTB11_RIGHTresults}, which lasted five years and sought to compare the diagnostic algorithms used by two ICDs for correctly diagnosing potentially fatal tachycardias (abnormally fast heart rhythms).
Our contributions are as follows:
\begin{itemize}
	\item We defined a heart model capable of producing several types of tachycardias, in terms of both timing characteristics and morphology of the electrical signals (the \emph{\acp{EGM}})
	(Section \ref{sec:heart modeling}).
	This model allows us to generate thousands of arrhythmia exemplars which serve as our virtual trial patients.
	\item For modeling purposes, we annotated segments of over 100 \ac{EGM} records of real patients to extract the tachycardias that they suffered from at time of recording.
	These electrograms are essential to simulate our models.
	\item Using the openly available literature, we implemented the tachycardia detection algorithms of two major ICD manufacturers: Boston Scientific and Medtronic.
	(See Section \ref{sec:device models}).
	\item We developed an experimental setup to validate our implementation of the Boston Scientific algorithm against a real ICD (Section \ref{sec:validation}).
	\mynote{SD}{wants "against a real ICD" deleted..}
	
	The connected heart model and ICD form the CPS under study.
	\item With the above elements in place, we generated a synthetic cohort, consisting of $11,000+$ heart models displaying a wide range of tachycardias.
	These models are then connected to the 2 ICD implementations and arrhythmia detection results are recorded.
	This allows us to estimate their relative efficacy on an condition-by-condition basis
	(Section \ref{sec:results}.)
\end{itemize}

We call our approach \ac{MBCT}, or \ac{MBCT}.
An \ac{MBCT} is a trial whose subjects are computer models of the (heart, device) closed-loop system.
By generating large cohorts, we can answer several questions contributing to the planning and execution of a clinical trial.

\paragraph{Modeling as regulatory grade evidence}
\label{sec:related work}
Most medical models today are aimed at either better understanding the phenomenon under study \cite{vfiborganization_Tusscher07} or at device debugging and verification \cite{VHM_proc}. 
There is only one case in which a computer model has been used to intervene in the regulatory process of medical devices, namely the T1 Diabetes Model (T1DM) of UVA/PADOVA \cite{T1DM}.
T1DM models glucose kinetics in hypoglycemia, and has been accepted by the FDA as a substitute for animal trials.
The T1DM has a fixed virtual cohort with 300 patients.
Its objective is to test the efficacy of new glucose control algorithms by simulating them on the virtual cohort.
While our models can be used in this way, our objective here is to target specific clinical trials steps and improve how they are conducted.
This dictates the experimental setup and the cohort generation considerations.

The Avicenna consortium \cite{Avicenna} lays out a vision for `In-Silico Clinical Trials' similar to our approach.
However, the emphasis in Avicenna is on individualized patient models, as they propose to customize the model to each patient enrolled in a trial.
In the present work, we propose a usage of MBCT \emph{prior} to recruitment.
Thus our models need not be fitted to a given patient's data, which might be impossible, invasive, or burdensome for the conduct of the trial.


\begin{figure}[tb]
	\vspace{-10pt}
	\includegraphics[scale=0.4]{figures/figICD.pdf}
	\vspace{-10pt}
	\smallcaption{ICD connected to the heart. The atrial, ventricular, and shock electrogram signals are measured by the device, which uses them to diagnose the current state of the heart and determine whether therapy is required.}
	\vspace{-10pt}
	\label{fig:icd}
\end{figure}



\begin{figure*}[tb]
	\vspace{-10pt}
	\centering
	\includegraphics[scale=0.3]{figures/figMBCToverview}
	\vspace{-10pt}
	\caption{Overview of an \ac{MBCT}. 1) \ac{EGM} recordings of real patients are adjudicated to create a table of \ac{EGM} morphologies of various tachycardias. 2) Subsets of these morphologies are combined with a timing model to create a synthetic heart model. 3) Through variation of the parameters of the model, an entire synthetic cohort is generated and simulated to produce synthetic \ac{EGM} signals. 4) Various device evaluation experiments can be executed with this synthetic cohort.}
	\vspace{-10pt}
	\label{fig:mbct overview}
\end{figure*}


\section{Hybrid systems and simulations}
\label{sec:preliminaries}

This section presents fairly standard definitions on 
%transition systems, 
hybrid systems and their simulations \cite{AlurHLP00ieee}.
It also defines STORMED hybrid systems, which admit finite bisimulations \cite{VladimerouPVD08_STORMED}.

\subsection{Transition and hybrid systems}
\label{sec:transition systems}

%{\large Partitions.} Given a set $Q$, a \emph{partition} $\partition = \{P_1,\ldots,P_k\}$ of $Q$ is a set of disjoint subsets of $Q$ whose union equals $Q$. 
%Let $\equiv_\partition$ be the associated equivalence relation.
%Partition $\partition'$ \emph{refines} $\partition$ if every block $P'$ of $\partition'$ is a subset of some block $P$ of $\partition$; we write this as $\partition' \subset \partition$.
%
\begin{defn}
	\label{defn:transition system}
	A \emph{transition system} $T = (Q,\labelSet,\trans{},Q_0)$ consists of a set of states $Q$, a set of events $\labelSet$ , a transition relation $\trans{} \subset Q \times \labelSet \times Q$, a set of initial states $Q_0$. 
	We write $q \trans{\slabel}q'$ to denote a transition element $(q,q') \in \trans{}$.
	Given $P\subset Q$, we define $Post_\slabel(P) \defeq \{q'\;|\;\exists q\in P. q \trans{\slabel}q'\}$
	%
	Given an equivalence relation $\sim$ on $Q$, the \emph{quotient system} $T/\sim$ is
	$T/\sim = (Q/\sim, \{*\}, \trans{}_\sim, Q_0/\sim)$
	where $[q] \trans{*}_\sim [q']$ iff $q \trans{\slabel} q'$ for some $\slabel \in \labelSet$.
	Here $[q]$ is the equivalence class of $q$ and $Q/\sim$ is the set of equivalence classes of $\sim$.
\end{defn}

\begin{defn}
	\label{defn:simulation}	
	Given two transition systems $T_1$ and $T_2$ with the same state space $Q$,
	a \emph{simulation} relation from $T_1$ to $T_2$ is a relation $\simu \subset Q \times Q$ such that 
	for all $(q_1,q_2) \in \simu$, if $q_1 \trans{\slabel}_1 q_1'$, there exists a $q_2' \in Q$ s.t. $q_2 \trans{\slabel}_2 q_2'$ and $(q_1',q_2') \in \simu$.
	A \emph{bisimulation relation} between $T_1$ and $T_2$ is both a simulation relation from $T_1$ to $T_2$ and from $T_2$ to $T_1$.
\end{defn}
%Given a partition $\partition$ of $Q$, the \emph{natural bisimulation} between $T$ and $T/\partition$ is $\Bc_P = \{(q,P) \in Q \times \partition \;|\; q \in P\}$.	
%Conversely, a bisimulation $\Bc$  of $T$ defines a partition $\partition_\Bc$ of $Q$ where $q,q'$ are in the same block of the partition iff $(q,q') \in \Bc$.
%
The bisimulation $\bisimu$ is said to \emph{respect} $\sim$ if $(q,q') \in \bisimu \implies q \sim q'$.
%
The following algorithm, if it terminates, yields a finite bisimulation for $T$ that respects the given equivalence relation~\cite{AlurHLP00ieee}.
Moreover, it is the \emph{coarsest} bisimulation (with respect to inclusion) that respects $\sim$.
\begin{algorithm}[t]
		\caption{Computing a bismimulation respecting $\sim$}
		\label{algo:bisimulation}
		\begin{algorithmic}
			\Require Transition system $T = (Q,\labelSet,\trans{},Q_0)$, equivalence relation $\sim$.
			\State Set $\simu = Q/\sim$			
			\While{$\exists P,P' \in \simu$ and $\slabel \in \labelSet$ s.t. $\emptyset \neq P' \cap Post_\slabel(P) \neq P'$}
				\State Set $\simu = \simu \setminus \{P'\} \cup \{P' \cap Post_\slabel(P) , P' \setminus Post_\slabel(P) \}$
			\EndWhile	
			\State Return $\simu$
		\end{algorithmic}
\end{algorithm}
Given a set of atomic propositions $AP$, if $\sim$ is s.t. $q \sim q'$ iff both states satisfy exactly the same set of atomic propositions, then model checking CTL$^*$ properties can be done on the finite bisimulation instead of the possibly infinite $T$.
%The \emph{coarsest} bisimulation $\bisimu$ that refines a partition $\partition$ is one such that there is no other bisimulation $\bisimu'$ satisfying $\partition_\bisimu \subset \partition_{\bisimu'} \subset \partition$.

%Given a subset $P$ of $Q$, we define its $\slabel$-successor as $Post_\slabel(P) = \{q \in Q | \exists p \in P. q \trans{\slabel} p\}$.
%In other words, $Post_\slabel(P)$ is the set of states forward-reachable from $P$ via the event $\slabel$.


%
%Note that the resulting bisimulation is the \emph{coarsest} bisimulation (i.e., with the least number of blocks in its induced partition) that refines $\partition$.
%\subsection{Hybrid systems}
%\label{sec:hybridSystems}
\begin{defn}
	\label{defn:hybrid system}	
	A \emph{hybrid automaton} is a tuple \[\Sys = (\stSet,\modeSet,\hsSet_0,\{f_\mode\}, Inv,E, \{\reset_{ij}\}_{(i,j)\in E}, \{\guard_{ij}\}_{(i,j)\in E})\] where 
		 $\stSet \subset \Re^n$ is the continuous state space equipped with the Euclidian norm $\|\cdot\|$, 
		$\modeSet \subset \Ne$ is a finite set of modes,
		 $\hsSet_0 \subset \stSet \times \modeSet$ is an initial set,
		 $\{f_\mode\}_{\mode \in \modeSet}$ determine the continuous evolutions with unique solutions,
		 $Inv: \modeSet \rightarrow 2^\stSet$ defines the invariants for every mode,
		 $E \subset \modeSet^2$ is a set of discrete transitions,
		 \yhl{$\guard_{ij} \subset \stSet$ is guard set for the transitions (so $\Sys$ transitions $i \rightarrow j$ when $\stPt \in \guard_{ij}$),
		 $\reset_{ij}: \stSet \rightarrow \stSet$ is an edge-specific reset function.}
		 \\
		 Set $\hsSet = \modeSet\times \stSet$.
		 Given $(\mode,\stPt_0) \in \hsSet$, the \emph{flow} $\theta_{\mode}(;\stPt_0):\Re_+ \rightarrow \Re^n$ is the solution to the IVP $\dot{x}(t) = f_\mode (x(t))$, $\stPt(0)=\stPt_0$.
\end{defn}
%
The associated transition system is $T_\Sys = (\hsSet,  E \cup \{\tau\},\trans{},\hsSet_0)$ 
with $\trans{} = (\bigcup_{e \in E} \trans{e}) \cup \trans{\tau}$ 
where $(i,\stPt) \trans{e} (j,y)$ iff $e = (i,j), \stPt \in \guard_{ij}, y = \reset_{ij}(\stPt)$ and $(i,\stPt) \trans{\tau} (j,y)$ iff $i = j$ and there exists 
a flow $\theta_i(\cdot;x)$ of $\Sys$ and $t\geq 0$ s.t. $\theta_i(t;x)=y$ and $\forall t' \leq t$, $\theta_i(t';x) \in Inv(i)$.
%Note that the transition $\trans{\tau}$ abstracts away time, i.e. it doesn't preserve information about the duration of continuous flow.
For a set $P \subset \hsSet$,$P_{|\stSet}$ denotes its projection onto $\stSet$, 
and $P_{|\modeSet}$ its projection onto $\modeSet$. 
\begin{defn}
	\label{defn:reachability operators}[Reachability]
	Let $\Sys$ be a hybrid system with hybrid state space $\hsSet$, 	 
	$I = [0,b) \subset [0,+\infty)$ be a (possibly unbounded) interval, 
	$t \in I$, 
	and $\epsilon >0$.
	The \emph{$\epsilon$-approximate continuous reachability operator}, 
	$\Rc^{\epsilon}_t : 2^\hsSet \rightarrow 2^\hsSet$ is given by
	\begin{eqnarray*}
		\Rc^{\epsilon}_t(P) = \{(i,\stPt) \in \stSet | \exists x_0 \in P_{|\stSet}, t \geq 0. 
		||\theta_i(t;x_0) - \stPt|| \leq \epsilon\} 
	\end{eqnarray*}
%	\begin{eqnarray*}
%	&&\Rc^{\epsilon}_t(P) = \{(i,\stPt) \in \stSet | \exists z: [0,t] \rightarrow \Re^n . z(0) \in P_{|\stSet},
%	\\
%	&&||z(t) - \stPt|| \leq \epsilon,
%	\forall t' \in [0,t]\; \dot{z}(t') = f_i(z(t')), z(t') \in Inv(i) \} \nonumber
%	\end{eqnarray*}
	where $P = \{i\}\times W$, $W \subset Inv(i)$.
	%
	Define also $\Rc^{\epsilon}_I(P) = \cup_{t\in I} \Rc^{\epsilon}_t(P)$.
	%
	The (exact) \emph{discrete reachability operator} is:
	\begin{eqnarray*}
	\Rc_{d}(P) &=& \cup_{j: (i,j) \in E} \reset_{ij}(P \cap G_{ij})
	\end{eqnarray*}
\end{defn}
%
For a hybrid system, $Post_\slabel$ computes the forward reach sets, and is implemented by $\Rc^0_{[0,\infty)}$ and $\Rc_d$. 
Algorithm \ref{algo:bisimulation}, applied to $T_\Sys$, implements the following iteration, 
in which 
$\Fc_t(\partition)$ is the coarsest bisimulation with respect to $\trans{\tau}$\footnote{I.e., $\Ft$ only considers the continuous transition relation. Namely, it is a bisimulation of $T_\Sys^c \defeq (Q/\sim,\{*\},\trans{\tau},Q_0/\sim)$.} 
respecting the partition $\partition$, 
and 
$\Fc_d(\partition) \defeq \{(h_1,h_2)  \;|\; (h_1 \trans{e} h_1') \implies (\exists e' \in E, h_2' \; . h_2 \trans{e'} h_2' \land h_1' \equiv_\partition h_2') \} \cap \partition$ \cite{VladimerouPVD08_STORMED}:
\begin{eqnarray}
W_0 = \Ft(Q/\sim), \quad\forall i\geq 0,\; W_{i+1} = \Ft(\Fd(W_i))
\end{eqnarray}
This iteration (equivalently, Alg.~\ref{algo:bisimulation}) does not necessarily terminate for hybrid systems because the reach set might intersect a given block of $Q/\sim$ an infinite number of times (see \cite{LaFerrierePS00_Ominimal} for an example).
The class of systems introduced in the next section has the property that Algorithm \ref{algo:bisimulation} does terminate for it and returns a finite $\simu$.

%By iterating the above two reachability operators, we get:
%\begin{defn}\cite{VladimerouPVD08_STORMED}
%	\label{defn:iterated reachability operators}	
%	Let $\Sys$ be a hybrid system with hybrid state space $\hsSet$, 	 
%	$I = [0,b) \subset [0,+\infty)$ be a (possibly unbounded) interval, 
%	$t \in I$, 
%	and $\epsilon >0$.
%	Given a partition $\partition$ of $\hsSet$, 
%	$\Fc_t(\partition)$ is the coarsest bisimulation with respect to $\trans{\tau}$ respecting $\partition$, and 
%	$\Fc_d(\partition) = \{(h_1,h_2)  \;|\; (h_1 \trans{d} h_1') \implies (\exists h_2' \; . h_2 \trans{d} h_2' \land h_1' \equiv_\partition h_2') \} \cap \partition$.
%\end{defn}

%\begin{algorithm}
%	\caption{Bisimulation for continuous transitions}
%	\label{algo:Ft}
%	\begin{algorithmic}
%		\Require Hybrid system $\Sys$, a partition $\partition$ of $\hsSet$.
%		\State Set $\simu = \partition$			
%		\While{$\exists P,P' \in \simu$ s.t. $\emptyset \neq P' \cap \Rc_I^0(P) \neq P'$}
%		\State Set $\simu = \simu \setminus \{P'\} \cup \{P' \cap \Rc_I^0(P) , P' \setminus \Rc_I^0(P) \}$
%		\EndWhile	
%		\State Return $\simu$
%	\end{algorithmic}
%\end{algorithm}


\subsection{O-minimality and STORMED systems}
\label{sec:ominimality}
We give a very brief introduction to o-minimal structures.
A more detailed introduction can be found in \cite{LaFerrierePS00_Ominimal} and references therein.
We are interested in sets and functions in $\Re^n$ that enjoy certain finiteness properties, called order-minimal sets (o-minimal).
These are defined inside \emph{structures} $\Ac = (\Re,<, +,-,\cdot,\exp,\ldots)$.
The subsets $Y \subset \Re^n$ we are interested in are those that are \emph{definable} using first-order formulas $\formula$: $Y = \{(a_1,\ldots,a_n) \in \Re^n \;|\;  \formula(a_1,\ldots,a_n)\}$.
(First-order formulas use the boolean connectives and the quantifiers $\exists,\forall$).
The atomic propositions from which the formulas are recursively built allow only the operations of the structure $\Ac$ on the real variables and constants, and the relations of $\Ac$ and equality.
For example $2x-3.6y < 3z$ and $x=y$ are valid atomic propositions of the structure $\Lc_\Re=(\Re,<, +,-,\cdot)$, while $cosh(x) < 3z$ is not because $cosh$ is not in the structure.
These structures are already sufficient to describe a set of dynamics rich enough for our purposes and for various classes of linear systems.
%\textbf{Formal definitions}.
A \emph{language} is a set of relations, functions and constants.
E.g. $\Lc_\Re = (<,+,-, \cdot,\Re)$ is the language where the only relation is $<$, the functions are $+,-,\cdot$, and the constants are taken from $\Re$,
and  $\Lc_{exp} = (<,+,-, \cdot,\exp, \Re)$ adds the exponential symbol to it.
Let $\Vc = \{x,y,z,x_1,x_2,\ldots\}$ be a set of variables.
A \emph{term} of a language is either a variable, a constant, or $F(\theta_1,\ldots,\theta_m)$ where $F$ is an $m$-ary function expressible in the language and the $\theta_i$ are terms.
E.g. $2x-3.6y$ is a term of $\Lc_\Re$.
The \emph{atomic formulas} of the language are of the form $\theta_1=\theta_2$ or $R(\theta_1,\ldots,\theta_m)$ where $R$ is an $m$-ary relation on the terms $\theta_i$.
E.g., $2x-3.6y < z\cdot z$ is a an atomic formula in $\Lc_\Re$ which uses the terms $2x-3.5y$ and $z \cdot z$.
The set of (first-order) \emph{formulas} is defined recursively as follows: every atomic formula is a formula, and if $\formula_1,\formula_2$ are formulas, then so are $\formula_1 \land \formula_2, \neg \formula_1, \forall x: \formula_1, \exists x:\formula_1$.
Here the symbols are the usual boolean connectives ( conjunction $\land$ and negation $\neg$), and quantifiers (there exists $\exists$ and for all $\forall$).
A \emph{sentence} in the language is a formula where all variables are inside the scope of a quantifier, e.g. $\exists y . (x+y>3)$ is not a sentence.

So far, a language has been defined in a purely syntactic manner.
A \emph{model} of a language is a set $S$ along with an interpretation of the relations, functions and constants of the language.
We are interested in this paper in the model of $\Lc_{exp}$ where the set $S = \Re$, and the symbols $<,+,-, \cdot,exp$ have their usual meaning (less than, addition, etc).
A set $Y \subset \Re^n$ is \emph{definable} in $\Lc_\Re$ if there exists a formula of the model such that $Y$ can be expressed as $Y = \{(a_1,\ldots,a_n) \in \Re^n \;|\; \formula(x_1,\ldots,a_n)\}$.
E.g., the formula $x^2-1=0$ defines the set $\{-1,+1\}$.
Finally, a \emph{theory} is a subset of sentences in the model $\Lc_\Re$, i.e. it is a collection of sentences that are true in $\Lc_\Re$.
\begin{defn}
	\label{defn:ominimal struct}	
	A theory of $(\Re,\ldots)$ is \emph{o-minimal} if the only definable subsets of $\Re$ are finite unions of points and (possibly unbounded) intervals.	
	A function $f:x \mapsto f(x)$ is o-minimal if its graph $\{(x,y) \;|\; y=f(x)\}$ is a definable set.
\end{defn}
We use the terms o-minimal and definable interchangeably, and they refer to $\Lc_{\exp}= (\Re,<, +,-,\cdot,\exp)$ which is known to be o-minimal.
%
The dot product between $x,y\in \Re^n$ is denoted $x \cdot y$, and $d(Y,S)=\inf \{ \|y-s\| \;|\; (y,s) \in Y\times S \}$.
\begin{defn}\cite{VladimerouPVD08_STORMED}.
	\label{defn:stormed system}	
	A \emph{STORMED hybrid system} (SHS) $\SHS$ is a tuple $(\Sys,\Ac, \phi,b_-,b_+, d_{min}, \epsilon,\zeta)$ where $\Sys$ is a hybrid automaton, $\Ac$ is an o-minimal structure, $d_{min}, \epsilon, \zeta$ are positive reals, $b_-,b_+ \in \Re$ and $\phi \in \stSet$ such that:
	\\
	\textbf{(S)} The system is $d_{min}$-separable, meaning that for any $e=(\mode, \mode ')\in E$ and $\mode ''\neq \mode'$,$d(\reset_e(\guard_{(\mode ,\mode ')}), \guard_{(\mode' ,\mode '')})>d_{min}$
	\footnote{\yhl{The original definition of separability \cite{VladimerouPVD08_STORMED} required the guards themselves to be separated, which is insufficient to guarantee that if $\Sys$ flows, it flows a uniform minimum distance along $\phi$. Indeed assume the guards are separated. If $x \in \guard_{(\mode,\mode')}$ and $y = \reset_{(\mode,\mode')}(x)$, it can be that $y \in \guard_{(\mode',\mode'')}$ and thus a jump happens, even though $\guard_{(\mode,\mode')}$ and $\guard_{(\mode',\mode'')}$ are separated. Therefore we need $d(y,\guard_{\mode',\mode''}) > d_{min}$ for all $y \in \reset_e(\guard_e)$, which is the condition we use in Def.~\ref{defn:stormed system}. The properties of SHS, in particular the existence of finite bisimulation, are therefore preserved by this change.}}
	\\
	\textbf{(T)} The flows (i.e., the solutions of the ODEs) are Time-Independent with the Semi-Group property (TISG), meaning that for any $\mode \in \modeSet, \stPt \in \stSet$, the flow $\theta_\mode$ starting at $(\mode , x)$ satisfies: 1) $\theta_{\mode}(0;x) = x$, 2) for every $t,t' \geq 0$, $\theta_{\mode}(t+t';x) = \theta_{\mode}(t'; \theta_\mode(t;x))$
	\\
	\textbf{(O)} All the sets and functions of $\Sys$ are definable in the o-minimal structure $\Ac$
	\\
	\textbf{(RM)} The resets and flows are monotonic with respect to the same vector $\phi$, meaning that \\
	1) (Flow monotonicity) for all $\mode \in \modeSet$, $x \in \stSet$ and $t,\tau \geq 0$, $\phi \cdot (\theta_\mode (t+\tau;x) - \theta_\mode (t;x) ) \geq \epsilon ||\theta_\mode (t+\tau;x) - \theta_\mode (t;x) ||$, 
	and \\
	2) (Reset monotonicity) for any edge $(\mode,\mode') \in E$ and any $x^-,x^+ \in \stSet$ s.t. $x^+ = R_{\mode ,\mode'}(x^-)$, 
	\begin{compactenum}
		\item if $\mode = \mode'$, then either $x^-=x^+$ or $\phi \cdot (x^+-x^-)\geq \zeta$
		\item if $\mode \neq \mode'$, then $\phi\cdot (x^+-x^-) \geq \epsilon ||x^+-x^-||$
	\end{compactenum}

	\textbf{(ED)} Ends are Delimited: for all $e \in E$ we have $\phi \cdot x \in (b_- , b_+)$ for all $x \in G_{e}$
\end{defn}
Intuitively, the above conditions imply the trajectories of the system always move a minimum distance along $\phi$ whether flowing or jumping, which guarantees that no area of the state space will be visited infinitely often. 
This is at the root of the finiteness properties of STORMED systems.
%
The following result justifies the interest in STORMED systems: they admit finite bisimulations.
\begin{thm}\cite{VladimerouPVD08_STORMED}
	\label{thm:stormed finite bisimu}	
	Let $\Sys$ be a STORMED hybrid system, and let $\partition$ be an o-minimal partition of its hybrid state space. 
	Then $\Sys$ admits a finite bisimulation that respects $\partition$.
\end{thm}
%We will need the following result.
%\begin{lemma}\cite{VladimerouPVD08_STORMED}
%	\label{lemma:finite steps}
%	Given a SHS $\SHS$, the number of discrete transitions of any execution of $\SHS$ is uniformly upper bounded.
%\end{lemma}
We need the following result in what follows.
\begin{prop}
	\label{prop:ED}
	If the state space $\stSet$ of a hybrid automaton $\Sys$ is bounded, then its guards have delimited ends.
\end{prop}
\begin{prf}
	For all guard sets $G$ and all $x \in G$, $||\phi \cdot x || \leq ||\phi|| \cdot ||x|| \leq ||\phi||.\max\{||x||, x\in \stSet \} < \infty$.
\end{prf}

%\section{Statement of results}
\label{sec:stmt}
state main results here in compact form
\section{Heart model}
\label{sec:heartcellularautomata}
For the verification of \acp{ICD},
we adopt the \acf{CA}-based heart model developed in \cite{Spector11_Emergence},\cite{CorreaEtAl11_EGMFractionation}.
%Like \cite{BartocciCBESG09_HIOAmodeling}, it is based on \acf{CA} and is augmented with differential equations in each cell to describe the evolution of the transmembrane voltage with time.
This model lies in-between high spatial fidelity but slow to compute PDE-based whole heart models  \cite{vfiborganization_Tusscher07}, and low spatial fidelity but very fast-to-compute automata-based models \cite{TECS}.
PDE-based models are not currently amenable to formal verification, both theoretically and practically.
\yhl{Models based on ionic currents \cite{Islam1MGSG14_CompositionalityCells} might be more accurate but are likely to be more computationally expensive.}
Timed automata models can not simulate the electrograms needed for \ac{ICD} verification.
\ac{CA}-based models are appealing due to their intuitive correspondence with the heart's anatomy and function and their relative computational simplicity.
\ac{CA}-based models were used in \cite{Mery},\cite{BartocciCBESG09_HIOAmodeling} and \cite{Chen14_Quantitative}.
This paper's model also has the important advantage of forming the basis of software used to train electrophysiologists, and allows interactive simulation of surgical procedures like ablation \cite{visibleep}.
\yhl{In particular, it can simulate fibrillation and other tachycardias.}%

\textbf{This paper's automata:}All hybrid automata in this paper have the whole state space as invariants and transitions are urgent (taken immediately when the guard is enabled).
We also observe that, as will be seen in Section \ref{sec:discriminators},
i) the \ac{ICD} will always reach a decision of VT or SVT in finite time, 
ii) at which point it resets its \yhl{controlled (software) variables} so new values are computed for the next arrhythmia episode.
So while the heart can beat indefinitely, for the purposes of \ac{ICD} verification, 
there's a uniform upper bound on the length of time of any execution.
Let $D \geq 0$ be this duration ($D$ is on the order of 30sec depending on device settings).
Also, the \ac{EGM} voltage signal $\egm$ has upper and lower bounds $\overline{\egm}$ and $\underline{\egm}$.
\yhl{Therefore, every mode of every automaton in what follows has a transition to mode End shown in Fig.~\ref{fig:endMode}.
We don't show these transitions in the automata figures to avoid congestion.
}
\begin{figure}[t]
\centering
\includegraphics[scale=0.4]{figures/endMode}
\caption{When the ICD makes a VT/SVT decision, all systems transition to mode End.}
\label{fig:endMode}
\end{figure}

%Past this time, the \ac{ICD} algorithm starts over, so we may consider that the entire system starts over.
% and may be viewed as simplifications of \cite{ChernyakFC97_PhysRevE}.
%We also use an electrode measurement model that captures the electric waveforms captured by the \ac{ICD} electrodes.
%\subsection{Basic cardiac electrophysiology}
%The heart has two upper chambers called the \emph{atria} and two lower chambers called the \emph{ventricles} (Fig. \ref{fig:icd})
%The synchronized contractions of atria and ventricles deliver an adequate supply of oxygenated blood to the rest of the body.
%This contraction is driven by electrical activity in the heart.
%Under normal conditions, the SinoAtrial (SA) node (a tissue in the right atrium) spontaneously \emph{depolarizes}, producing an electrical wave that propagates to the atria and then down to the ventricles (Fig.\ref{fig:whole heart})
%This is referred to as the \ac{NSR}.
%Disturbances of \ac{NSR} are known as \emph{arrhythmias}, and can result in insufficient blood supply and even death of a patient. 
%\emph{\ac{VT}} is an example of an arrhythmia originating in the ventricles, in which the ventricles spontaneously beat at a very high rate.
%If the \ac{VT} is sustained, or degenerates into \ac{VF}, it is fatal within seconds.
%An abnormally fast heart rate that originates in the atria is referred to as a \emph{\ac{SVT}}.
%This is a diseased but non-fatal condition, and many arrhythmias fall under this heading.
%In what follows, we will refer to sustained \ac{VT} and \ac{VF} together as \ac{VT}.
%%
\subsection{Cellular automata model}
%\begin{figure}[t]
%	\centering
%	\vspace{-10pt}
%	\includegraphics[scale=0.35]{figures/wholeHeartMesh}
%	\vspace{-10pt}
%	\caption{\small Whole heart modeled as a 2D mesh of cells. The \ac{ICD} leads are shown in the right atrium and ventricle. \textbf{AV}: atrio-ventricular node, \textbf{RVA}: right ventricle apex, \textbf{SA}: sino-atrial node.}
%	\label{fig:whole heart}
%	\vspace{-10pt}
%\end{figure}
The heart has two upper chambers called the \emph{atria} and two lower chambers called the \emph{ventricles} (Fig. \ref{fig:icd})
The synchronized contractions of the heart are driven by electrical activity.
Under normal conditions, the SinoAtrial (SA) node (a tissue in the right atrium) spontaneously \emph{depolarizes}, producing an electrical wave that propagates to the atria and then down to the ventricles (Fig.\ref{fig:overview})
In this model, the myocardium (heart's muscle) is treated as a 2D surface (so it has no depth), and discretized into \emph{cells}, which are simply regions of the myocardium (Fig. \ref{fig:overview}). 
Thus we end up with $N^2$ cells in a square $N$-by-$N$ grid.
A cell's voltage changes in reaction to current flow from neighboring cells, and in response to its own ion movements across the cell membrane.
This results in an \emph{\ac{AP}}.

Fig. \ref{fig:cellaut} shows how the \ac{AP} is generated by a given cell \cite{Klabunde_CVEPconcepts}:
in its quiescent mode (Phase 4), a cell $(i,j)$ in the grid has a cross-membrane voltage $V(i,j,t)$ equal to $V_{min} < 0$.
As it gathers charge, $V(i,j,t)$  increases until it exceeds a threshold voltage $V_{th}$.
\yhl{In Phase 0}, the voltage then experiences a very fast increase (Phase 0), called the upstroke, to a level $V_{max} > 0$, after which it decreases \yhl{(Phase 1)} to a plateau \yhl{(Phase 2)}.
It stays at the plateau level for a certain amount of time \textbf{PD} then decreases linearly to below $V_{th}$ (Phase 3 - ERP).
Once below $V_{th}$ it is said to be in the Relative Refractory Period (Phase 3 - RRP) .
\yhl{In Phase 3 - RRP}, the cell can be depolarized a second time, albeit at a higher threshold $V_{th,2}$, slower and to a lower plateau level $V_{max,2} < V_{max}$ \yhl{(Upstroke 2)}.
Otherwise, when the voltage reaches $V_{min}$ again, the cell enters the quiescent stage again. 
This model is suitable for both pacemaker and non-pacemaker cells, the main differences being in the duration of the plateau (virtually non-existent for pacemaker cells), and the duration of phases 0 and 4 (both are shorter for pacemaker cells).

In Fig. \ref{fig:cellaut}, $V(i,j) \in \Re$ denotes the voltage in cell $(i,j)$ of the grid, and \yhl{$V =(V(1,1),\ldots, V(N^2,N^2))^T$} in $\Re^{N^2}$ groups the cross-membrane voltages of all cells in the heart.
The whole heart model $\Sys_{CA}$ is the parallel composition of these $N^2$ single-cell models. 
\begin{figure}[t]
	\centering
	\includegraphics[scale=0.26]{figures/cellaut1v2}
	\vspace{-10pt}
	\caption{Hybrid model $\Sys_c$ of one cell of the heart model. AP figure from \cite{eplab}. 
		$V_{th,2}>V_{th}$, $V_{max,2}<V_{max}$}
	\vspace{-20pt}
	\label{fig:cellaut}
\end{figure}
%
The $(i,j)^{th}$ cell's voltage at time $t$ in Phase 4 depends on that of its neighbors and its own as follows \cite{Spector11_Emergence}
\begin{eqnarray}
\dot{V}(i,j,t) &=& \frac{1}{R_h}[V(i-1,j,t)+V(i+1,j,t) - 2V(i,j,t)] 
\nonumber \\ 
&& +  \frac{1}{R_v}[V(i,j-1,t) +  V(i,j+1,t) - 2V(i,j,t)]  
\nonumber\\
&=& a(i,j)^TV(t), \; a(i,j) \in \Re^{N^2}
\;
\end{eqnarray}
where $R_h$, $R_v$ are conduction constants that can vary across the myocardium.
Thus $V$ evolves according to a linear ODE $\dot{V} = AV$ where $A$ is the matrix whose rows are the $a(i,j)$. 
The two states $t$ and $t_p$ are clocks.
Clock $t_{p}$ keeps track of the value of the last discrete jump. 
We will use this arrangement in all our models: it avoids resetting the clocks which preserves Reset Monotonicity.

\acp{ICD} observe the electrical activity through three channels (Fig.~\ref{fig:icd}).
Each signal is called an \acf{EGM} signal.
The signal read on a channel is given by \cite{CorreaEtAl11_EGMFractionation}:
\begin{equation}
	\label{eq:vegm}
	\egm(t) = \frac{1}{K} \sum_{i,j} \left(\frac{1}{||p_{i,j}-p_0|| } - \frac{1}{||p_{i,j}-p_1||}\right) \dot{V}(i,j,t)
\end{equation}
\yhl{where $\|\cdot\|$ is the Euclidian norm, $p_0$ and $p_1$ are the electrodes' positions and $p_{i,j}$ is the position of the $(i,j)^{th}$ cell on the 2D myocardium ($p_0,p_1,p_{i,j} \in \Re^2$). 
Positions $p_0,p_1$ should be chosen different from $p_{i,j}$ to avoid infinities.}
%This model was validated against real recordings in vitro \cite{StinnettDonnelly12_EGMresolution}.

\textbf{Extensions}. 
The Action Potential Duration (APD) restitution mechanism of heart cells as modeled in \cite{Spector11_Emergence} can be included in this model without changing its formal properties.
\yhl{More detailed APD restitution models exist~\cite{Grosu09_LearningEmergent}.
Also, note that cell topology (the way cells are connected to each other) is not a factor in determining the STORMED property, so other topologies than a rectangular mesh may be used.}

We now state and prove the main result of this section.
\begin{thm}
	Let $\Sys_{CA}$ be the whole heart cellular automaton model obtained by parallel composition of $N^2$ models $\Sys_c$ with state vector $x = [V, t,t_p,\egm ] \in \Re^{N^2}\times \Re^{3}$.
	Assume that all executions of the system have a duration of $D\geq 0$.
	Then $\Sys_{CA}$ is STORMED.
\end{thm}
\begin{prf}
	We verify each property of STORMED.
In this and all the proofs that follow, the approach is the same: $(ED)$ holds by Prop.~\ref{prop:ED} because our state spaces are bounded.
After establishing properties $(S), (T)$ and $(O)$, we draw up the constraints on $\phi$ and $\varepsilon$ imposed by reset and flow monotonicity (property (RM)). 
Then we argue that these constraints can be solved for $\phi$ and $\varepsilon$.
Often there is more than one solution and we just point to one.

\textbf{(S)} Separability holds because $V_{min} < V_{th}< V_{th,2} < V_{max,2} < V_{max}$ and $PD>0,D_{Ph_1}>0$. 
For example, on transition \textbf{Phase 4} $\rightarrow$ \textbf{Phase 0}, $V(i,j)=V_{th}$, which is separated from the next guard $\{V(i,j) > V_{max}\}$ by $|V_{max}-V_{th}|$.
%The only mode with two transitions is Phase 3 - RRP and its guards are $\{V(i,j) \leq V_{min}\}$ and $\{V(i,j) \geq V_{th,2}\}$ with separation $\sqrt{|V_{th,2}-V_{min}|}>0$.
\\
\textbf{(T)} All flows are linear or exponential and thus are TISG.
\\
\textbf{(O)} The flows, resets and guard sets are all definable in $\Lc_{\exp}$.
In particular the flow of $\dot{V} = AV$ is exponential with real exponent, and $\egm$ is a sum of exponentials and linear terms.
\\
\textbf{(RM)}
We seek a vector $\phi = (\phi_V,\phi_t,\phi_p,\phi_\egm)^T \in \Re^{N^2+3}$ such that resets and flows are monotonic along $\phi$.
Only transitions $p \rightarrow q \neq p$ are to be found in $\Sys_{CA}$, during which only $t_p$ is reset. 
Always, $t_p^+ = t \geq t_p^-$, thus the reset is indeed monotonic as can be seen by choosing any $\varepsilon >0$ and $\phi_p > \varepsilon$.

Monotonic flows: $\phi$ must also be such that in all modes:
\begin{equation*}
\phi \cdot (\theta_\ell(t+\tau;x) - \theta_\ell(t;x)) \geq \varepsilon ||\theta_\ell(t+\tau;x) - \theta_\ell(t;x)||
\end{equation*}
Decomposing, we want 
\begin{eqnarray}
\label{eq:monotonic flow ca}
&&\phi_V \cdot(V(t+\tau) - V(t)) + \phi_t \tau + \phi_p \cdot 0 
\\
&&\quad+ \phi_\egm \cdot (\egm(x,t+\tau) - \egm(x,t)) \geq \varepsilon ||\theta_\ell(x,t+\tau) - \theta_\ell(x,t)||
\nonumber 
\end{eqnarray}
Now note that all flows have bounded derivatives in every bounded duration of flow and are thus Lipschitz. 
Let $L_V$ be the Lipshitz constant of $V(t)$ and $L_{\egm}$ that of $\egm(t)$.
Then on the LHS of the above inequality we have 
$\phi_V \cdot(V(t+\tau) - V(t)) + \phi_\egm \cdot(\egm(t+\tau) - \egm(t)) \geq -\phi_V L_V \tau  -\phi_\egm L_{\egm} \tau$.
On the RHS we have  
$\varepsilon (L_V\tau + L_{\egm}\tau + \tau) \geq \varepsilon ( ||V(t+\tau) - V(t)|| + ||\egm(t+\tau) - \egm(t) ||  + \tau) \geq \varepsilon ( ||\theta_\ell(x,t+\tau) - \theta_\ell(x,t)||)$
Thus \eqref{eq:monotonic flow ca} is satisfied if the stronger inequality 
\[-\phi_V L_V \tau  -\phi_\egm L_{\egm} \tau + \phi_t \tau \geq \varepsilon (L_V\tau + L_{\egm}\tau + \tau) \]
is satisfied.
But this can be achieved by, for example, choosing $\phi_V = \phi_\egm = 0$ and $\phi_t \geq \varepsilon(L_V+L_{\egm}+1)$.
\\
\textbf{(ED)} Our system has bounded state spaces: $V$ and $\egm$ are voltages typically in the range $[-80,60]$ mV and $t_p \leq t \leq D$.
So (ED) holds by Lemma \ref{prop:ED}. 
\end{prf}
\section{ICD Sensing}
\label{sec:sensing}
\begin{figure}[t]
	\centering
	\includegraphics[scale=0.35]{figures/sensingModel}
	\vspace{-10pt}
	\caption{\small $\Sys_{Sense}$. States not shown in a mode have a 0 derivative, e.g., $\dot{eF}=0$ in all modes.}
	\vspace{-10pt}
	\label{fig:sensingModel}
\end{figure}
\begin{figure}[t]
	\centering
	\includegraphics[scale=0.3]{figures/sensingExample}
	\vspace{-10pt}
	\caption{\small Example of dynamic threshold adjustment in ICD sensing algorithm. The shown signal is rectified.}
	\vspace{-10pt}
	\label{fig:sensingExample}
\end{figure}

\emph{Sensing} is the process by which cardiac signals $\egm$ measured through the leads of the \ac{ICD} are converted to cardiac timing events.
The \ac{ICD} sensing algorithm is a threshold-based algorithm which declares events when the signal exceeds a dynamically-adjusted threshold $Th$.
%The threshold is dynamically adjusted in order to operate robustly in complex environments where cardiac events can vary greatly in signal amplitude and frequency, such as during \ac{VF}.

Fig. \ref{fig:sensingModel} shows the model $\Sys_{Sense}$ of the sensing algorithm, and Fig. \ref{fig:sensingExample} illustrates its operation. 
%In Fig. \ref{fig:overview} (ICD Sensing - $\Sys_{Sense}$), states not shown in a mode have a 0 derivative, e.g., $\dot{eF}=0$ in all modes. $y(t) = |\egm(t)|$.
The sensing takes place on the rectified \ac{EGM} signal $y = |\egm|$.
After an event is declared at the current threshold value ($y(t)\geq Th(t)$ in Fig. \ref{fig:sensingModel}), the algorithm tracks the signal in order to measure the next peak's amplitude \yhl{(Peak Tracking)}.
%During transition, the state to indicate peak discovery is reset ($f=0$). 
For a duration $MinTP$ (min tracking period) the latest peak is saved in $y_M$.
A variable $f$ indicates that a peak was found.
After a peak is found ($f==1$) and after the end of the tracking period, the algorithm enters a fixed \emph{Blanking Period} \yhl{(Blanking)}, during which additional events are ignored.
\yhl{On the transition to Blanking}, $Th$ and $Th_0$ are set to 3/4 the current value of $y_{M}$ and the exponential factor of decay is updated ($eF=(-1/3)*ln{\frac{minTh}{TH}}$). 
At the end of the blanking period, the algorithm then transitions to the Exponential Decay mode in which $Th$ decays exponentially from $Th_0$ to a minimum level \yhl{(Exponential Decay)}:
$Th(t) = \max(minTh, Th_0\cdot exp(-(eF/TC)t)) $.
The algorithm stays in the Exponential Decay mode for at least a sampling period of $MinDecP$.
Correspondingly, there is a de facto Maximum Decay Period $MaxDecP$ after which the system transitions again to PeakTracking since the signal $y$ is bound to exceed the minimum threshold $minTh$.
Different manufacturers may use a step-wise decay instead of exponential, but the principle is the same.
%
Local peak detection is modeled via the $\dot{y} = 0 \wedge \ddot{y}<0$ transition.
While $y=|\egm|$ is non-differentiable at 0, the peak will occur away from 0, as shown in Fig. \ref{fig:sensingExample}.
The other states in Fig. \ref{fig:sensingModel} are $t, t_p$ (clocks).
$minTh$ and $TC$ are constant parameters.
\begin{thm}
	\label{thm:sensing}
	$\Sys_{Sense}$ is STORMED.	
\end{thm}
\begin{prf}
	\textbf{(S)} By definition, we only need to consider transitions between different modes to establish separability. 
	For all such transitions, there is a minimum dwell time in the mode before taking the transition, namely $MinTP$ in PeakTracking, $BlankingPeriod$ in Blanking, and  $MinDecP$ in mode ExponentialDecay.
	So the system is separable since there is a uniform minimum flow before jumping.
	\\
	\textbf{(T)} Flows are either constant, (piece-wise) linear, or piece-wise linear and exponential (in the case of $y$ and its derivatives) and therefore are TISG.
	\\
	\textbf{(O)} All the flows, resets and guard sets are definable in $\Lc_{\exp}$.
	(The absolute value and $\max$ functions can be broken down into boolean disjunctions of definable functions, and $t \mapsto \ln(t)$ is o-minimal by o-minimality of $\exp$).
	\\
	\textbf{(RM)} The state is $x = (t, t_p, y, y_M, f, Th, Th_0,eF) \in \Re^8$, and let 
	 $\phi = (\phi_t, \phi_{p}, \phi_y, \phi_m, \phi_f, \phi_{Th},\phi_0,\phi_{eF})$ be the corresponding $\phi$ vector.
	Recall that the \ac{EGM} voltage $\egm$, and so $y=|s|$, is upper-bounded by $V_M$.	
	\\ 
	\textbf{ExponentialDecay $\rightarrow$ PeakTracking}.
	Only $t_p,y_M$ and $f$ are modified, so monotonicity produces the constraint
	 $\phi_p(t-t_p) +\phi_m(0-y_M) + \phi_f(0-1) \stackrel{Want}{\geq} \varepsilon (|t-t_p|+|y_M|+1)$.
	We require the stronger constraint to hold:
	\[\phi_t MinDecP - \phi_m V_M -\phi_f \stackrel{Want}{\geq} \varepsilon(MaxDecP + V_M+1)\]
	\\
	\textbf{PeakTracking $\rightarrow$ PeakTracking}. Only $y_M$ and $f$ are reset. 
	Algebraic manipulation yields $-2V_M\phi_m + \phi_f \stackrel{Want}{\geq} \zeta$
	\\
	\textbf{PeakTracking $\rightarrow$ Blanking}.
	$t_p,eF,Th$ and $Th_0$ are reset, so we get
	\begin{eqnarray*}
	&&\;\phi_p(t-t_p) + \phi_{eF}(-(1/3)\ln(minTh/Th)-eF) 
	\\
	&&+\phi_{Th}(3y_M/4-Th) +\phi_0(3y_M/4-Th_0)
	\\
	&&\geq \varepsilon(|t-t_p|+ |-\frac{1}{3}\ln(\frac{minTh}{Th})-eF|
	\\
	&&+|\frac{3y_M}{4}-Th|+|\frac{3y_M}{4}-Th_0|)
	\end{eqnarray*}
	
	$Th$ is lower-bounded by $minTh$ at all times, and it is naturally upper-bounded by $V_M$ as the threshold should never exceed the largest possible attainable voltage. 
	By the same token, $0\leq eF \leq (1/3)\ln(V_M/minTh)$.
	Then we want the stronger inequality
	\begin{eqnarray*}
	\phi_p MinTP &+& \phi_{eF}(0-(1/3)\ln(V_M/minTh)
	\\
	&+&\phi_{Th}(-V_M) +\phi_0(-V_M)
	\\
	&\geq& \varepsilon(MaxTP+ |\frac{1}{3}\ln(\frac{V_M}{Th})|+|V_M|+|V_M|)
	\end{eqnarray*}
	\\
	\textbf{Blanking $\rightarrow$ ExponentialDecay}. Only $t_p$ is reset and therefore we want, $\phi_p(t-t_p) \geq \varepsilon(|t-t_p|)$, thus the transition yields $\phi_p \geq \varepsilon$.
	
	The above equations can be simultaneously satisfied.
	The simplest thing would be to set all $\phi$ terms that appear above to 0 except for $\phi_t,\phi_p$ which are calculated accordingly.
	
	The flows can be shown to be monotonic along the same $\phi$ and with the same $\varepsilon$.
	For example, in mode ExponentialDecay, only $t,y$ and $Th$ flow.
	Making use of the $V_M$ bound on $y$, we get the constraint
	$\phi_t \tau - 2V_M\phi_y +\phi_{Th}(Th(t+\tau)-Th(t))\geq \varepsilon(\tau+2V_M + |Th(t+\tau)-Th(t)| )$, 
	which yields $\phi_t \geq \varepsilon$, $\phi_y \leq -\varepsilon$ and $\phi_{Th} \geq \varepsilon$. 
	Similarly for the rest.	
\end{prf}

%\begin{thm}
%\label{thm:sensing}
%$\Sys_{Sense}$ is STORMED.	
%\end{thm}
%\begin{prf}
%\textbf{(S)} The guards are separable since each mode has only one guard.
%\\
%\textbf{(T)} The flows are constant, linear or equal to $\pm \egm(t)$ (in the case of $y$) and so are TISG.
%\\
%\textbf{(O)} All the flows, resets and guard sets are definable in $\Lc_{\exp}$.
%(The absolute value and $\max$ functions can be broken down into boolean disjunctions of definable functions).
%In particular, $t \mapsto \ln(t)$ is o-minimal by o-minimality of $\exp$.
%\\
%\textbf{(RM)} The only reset happens on the PeakTracking $\rightarrow$ Blanking transition. 
%The state is $x = (t,Th, eF,Th_0,t_p,y) \in \Re^5$.
%We seek a vector $\phi = (\phi_t, \phi_{Th}, \phi_{eF},\phi_0,\phi_y)$ and $\varepsilon >0$ s.t. 
%\begin{equation}
%\label{eq:sense rm}
%\phi \cdot \left(\begin{matrix}
%t-t \\ (3/4)Th-Th\\ -(1/3)\ln(minTh/Th) - eF\\ (3/4)Th-Th_0\\ t-t_p \\ y-y
%\end{matrix}
%\right) \defeq \phi \cdot \delta \stackrel{Want}{\geq} \varepsilon ||\delta||
%\end{equation} 
%$Th$ is lower-bounded by $minTh$ at all times, and it naturally has an upper bound, since it doesn't make sense to set it above the largest possible voltage. 
%Let that maximum be $maxTh$.
%Then we want the stronger inequality
%\begin{eqnarray*}
%&&\phi_{Th}(-Th/4) + \phi_{eF}(-(1/3)ln(minTh/Th)-eF) 
%\\
%&+& \phi_0(3Th/4-Th_0) + \phi_t (t-t_p) \geq ||\delta||
%\\
%&\geq& -\frac{\phi_{Th}}{4}maxTh + \phi_{eF}(-\frac{2}{3}\ln(minTh/Th)) 
%\\
%&& -\phi_0(2maxTh) + \phi_p(t-t_p)
%\\
%&\stackrel{Want}{\geq}& \varepsilon \left(|\frac{maxTh}{4}| + |\frac{2}{3}\ln(\frac{minTh}{Th})| + |2maxTh| +  \phi_p|t-t_p|\right)
%\end{eqnarray*}
%Since the $\ln$ term is negative and $t\geq t_p$,this yields the constraints:
%$\phi_0,\phi_{Th} < -\varepsilon \text{  and  } \phi_{eF},\phi_t > \varepsilon$.
%%\begin{equation}
%%\label{eq:constraint sense rm}
%%\phi_0,\phi_{Th} < -\varepsilon \text{  and  } \phi_{eF},\phi_t > \varepsilon
%%\end{equation}
%
%The flows are also monotonic along the same $\phi$ and with the same $\varepsilon$.
%For any $t,\tau>0$ and $x\in \stSet$, flow monotonicity is implied by the stronger inequality
%\begin{equation}
%\label{eq:sense fm}
%\phi \cdot \left(\begin{matrix}
%t+\tau-t \\ Th-Th\\ eF - eF\\ Th_0-Th_0\\ t_p-t_p \\ |\egm(t+\tau)|-|\egm(t)|
%\end{matrix}
%\right) \stackrel{Want}{\geq} \varepsilon (\tau + \underbrace{||\egm(t+\tau)|-|\egm(t)||}_{\delta \egm})
%\end{equation} 
%$\implies \phi_t \tau + \phi_y (|\egm(t+\tau)|-|\egm(t)|) \geq \varepsilon (\tau + \left| |\egm(t+\tau)|-|\egm(t)| \right| )$
%Therefore we can choose $\phi_t > \varepsilon$ as before and $\phi_y < -\varepsilon$.
%%We show this for Mode 1 only, the other modes are dealt with similarly.
%%\underline{Mode 1.} For any $t,\tau>0$ and $x\in \stSet$, flow monotonicity 
%%$\phi\cdot(\theta(t+\tau;x)-\theta(t;x)) \geq \varepsilon ||\theta(t+\tau;x)-\theta(t;x)||$ is implied by the stronger inequality
%%\begin{equation}
%%\label{eq:sense fm}
%%\phi \cdot \left(\begin{matrix}
%%t+\tau-t \\ Th-Th\\ eF - eF\\ Th_0-Th_0\\ t_p-t_p \\ -\egm(t+\tau)+\egm(t)
%%\end{matrix}
%%\right) \stackrel{Want}{\geq} \varepsilon (\tau + \underbrace{|\egm(t)-\egm(t+\tau)|}_{\delta \egm})
%%\end{equation} 
%%Observing that the \ac{EGM} signal $\egm$ has naturally defined minimum $\egm_{min}$ and maximum $\egm_{max}$, \eqref{eq:sense fm} is further implied by 
%%\begin{equation}
%%\phi_t\tau + \phi_y(\delta \egm) \geq \phi_t \tau - \phi_y(s_{min} -  s_{max}) \stackrel{Want}{\geq} \varepsilon (\tau +|s_{min} -  s_{max}|)
%%\end{equation}
%%which yields the constraints $\phi_t > \varepsilon, \phi_y < -\varepsilon$, which are consistent with \eqref{eq:constraint sense rm}.
%%The flow monotonicity constraints from the other modes are similarly consistent with \eqref{eq:constraint sense rm}. 
%\end{prf}

\section{Arrhythmia detection}
\label{sec:discriminators}
Because a sustained \ac{VT} (or \ac{VF}) can be fatal whereas an \ac{SVT} is usually not fatal, 
\emph{the ICD's main task is to discriminate \ac{VT} from \ac{SVT} and deliver therapy to the former only} \cite{compass}.
%\emph{\ac{VT}} is an example of a tachycardia originating in the ventricles, in which the ventricles spontaneously beat at a very high rate.
%If the \ac{VT} is sustained, or degenerates into \ac{VF}, it can be fatal.
%A tachycardia that originates above the ventricles is referred to as a \emph{\ac{SVT}} and is a diseased but non-fatal condition.
%In what follows, we will refer to sustained \ac{VT} and \ac{VF} together as \ac{VT}.
%\emph{The ICD's main task is to discriminate \ac{VT} from \ac{SVT} and deliver therapy to the former only}.
Most \ac{VT}/\ac{SVT} detection algorithms found in ICDs today are composed of individual \emph{discriminators}. 
A discriminator is a software function whose task is to decide whether the current arrhythmia is \ac{SVT} or \ac{VT}.
No one discriminator can fully distinguish between SVT and VT.
Thus a detection algorithm is often a decision tree built using a number of discriminators \emph{running in parallel}.
We have modeled each discriminator in this detection algorithm as a STORMED hybrid system.
The algorithm itself is then a hybrid system.
\textbf{The ICD system is thus 
$\mathbf{\Sys_{ICD} = \Sys_{Sense}||\Sys_{Detection-Algo}}$ where $\mathbf{\Sys_{Detection-Algo}}$ is the parallel composition of the discriminator systems}.
In what follows, we present two of these discriminators we modeled, which are found in most ICDs and model them as hybrid systems, and prove they are STORMED.

%\subsection{Three Consecutive Fast Intervals}
\label{sec:tcfi}
\begin{figure}[t]
	\centering
	\includegraphics[scale=0.3]{figures/TCFI}
		\vspace{-10pt}
	\caption{Three Consecutive Fast Intervals $\Sys_{TCFI}$}
	\label{fig:tcfi}
\end{figure}
Our first module simply detects whether three consecutive fast intervals have occurred, where `fast' means the interval length, measured between 2 consecutive peaks on the \ac{EGM} signal, is shorter than some pre-set amount.
See Fig. \ref{fig:tcfi}.
States $t$ and $t_p$ are clocks as before.
The vector $L_3$ is three-dimensional, and stores the values of the last three intervals.
The event VEvent? is shorthand for the transition $y(t) \geq Th$ being taken by the $\Sys_{Sense}$ automaton.
In other words, it indicates a ventricular event.
Then $L_3$ gets reset to $L_3^+ = (z_1,z_2,z_3)^+ \defeq \text{Circulate}(L_3,t-t_p)$ where
\begin{equation}
L_3^+ = 
\left(\begin{matrix}
z_2\\z_3\\t-t_{p}\\
\end{matrix}
\right)
=
\left(\begin{matrix}
0 & 1 & 0\\0& 0& 1\\0& 0 &0
\end{matrix}
\right) L_3 + 
\left(\begin{matrix}
0\\0\\t-t_p
\end{matrix}
\right)
\end{equation}
%
\begin{lemma}
	$\Sys_{TCFI}$ is STORMED.
\end{lemma}


\subsection{Vector Timing Correlation}
\label{sec:VTC}
\begin{figure}[t]
	\centering
	\vspace{-15pt}
	\includegraphics[scale=0.3]{figures/VTCEGMCompare}
	\caption{\small \acp{EGM} of different origin have different morphologies.}
	\label{fig:egmmorphology}
	\vspace{-10pt}
\end{figure}
%\caption{\small \acp{EGM} of different origin have different morphologies, while \acp{EGM} of same origin have very similar morphologies.}
It has been clinically observed that a depolarization wave originating in the ventricles (as produced during \ac{VT} for example) will in general produce a different \ac{EGM} morphology than a wave originating in the atria (as produced during \ac{SVT}) \cite{compass}.
See Fig. \ref{fig:egmmorphology}.
%
A morphology discriminator measures the correlation between the morphology of the current \ac{EGM} and that of a stored \emph{template} \ac{EGM} acquired during normal sinus rhythm.
If the correlation is above a pre-set threshold for a minimum number of beats, then this is an indication that the current arrhythmia is supraventricular in origin.
Otherwise, it might be of ventricular origin.

Boston Scientific's implementation of a morphology discriminator is called Vector and Timing Correlation (VTC).
VTC first samples 8 \emph{fiducial} points $\egm_i,i=1,\ldots,8$ on the current \ac{EGM} $\egm$ at pre-defined time instants.
Let $\egm_{m,i}$ be the corresponding points on the template \ac{EGM}.
A simple 0-shift correlation $\rho_{new}$ is calculated between the two sequences. 
If 3 out of the last 10 calculated correlation values exceed the threshold, then \ac{SVT} is decided and therapy is withheld.

The system of Fig. \ref{fig:HVTC} implements the VTC discriminator.
As before, $t$ is a local clock.
$\mu$ accumulates the values of the current \ac{EGM}, $\alpha$ accumulates the product $\egm_i \egm_{m,i}$, 
$\beta$ accumulates $\egm_i^2$.
State $w$ is an auxiliary state we need to establish the STORMED property.
$\vec{\nu}$ is a 10D binary vector: $\nu_i = -1$ if the $i^{th}$ correlation value fell below the threshold, and is $+1$ otherwise.
$L_3$ is the state of $\Sys_{TCFI}$: the guard condition $L_3 \leq th$ indicates that all its entries have values less than the tachycardia threshold, which is when $\Sys_{VTC}$ starts computing.
$WindowEnds$ indicates the `end' of an \ac{EGM}, measured as a window around the peak sensed by $\Sys_{Sense}$.  
%
\begin{figure}[t]
\centering
\includegraphics[scale=0.325]{figures/VTC1v2}
\vspace{-10pt}
\caption{VTC calculation. $iT_s$ is the sampling time for the $i${th} fiducial point, $i=1,\ldots,8$. $R2_{1},\ldots,R2_{8}$ are the corresponding resets. For clarity of the figure, 8 transitions are represented on the same edge.}
\vspace{-10pt}
\label{fig:HVTC}
\end{figure}
%
\begin{lemma}
	\label{lemma:vtc}
	$\Sys_{VTC}$ is STORMED.
	\end{lemma}

\subsection{Stability discrimination}
\label{sec:stability}
%\begin{figure}[t]
%	\centering
%	\includegraphics[scale=0.35]{figures/EGMStability}
%	\caption{Examples of a unstable rhythm (top) and stable rhythm (bottom).}
%	\label{fig:stable unstable}
%\end{figure}
\emph{Stability} refers to the variability of the peak-to-peak cycle length.
A rhythm with large variability (above a pre-defined threshold) is said to be \emph{unstable}, and is called stable otherwise.
The Stability discriminator is used to distinguish between atrial fibrillation, which is usually unstable, and \ac{VT}, which is usually stable.
%Atrial fibrillation, which is usually unstable, might induce a high ventricular rate .
%A \ac{VT}, on the other hand, is usually stable.
%Therefore this is a useful discriminator.

The Stability discriminator shown in Fig. \ref{fig:Hstab} simply calculates the variance of the cycle length over a fixed period called a Duration (measured in seconds).
Let $DL \geq 0$ be the Duration length.
\yhl{The events $DurationBegins?$ and $DurationEnds?$ indicate the transitions} of a simple system that measures the lapse of one Duration (not shown here).
State $t$ is a clock, $L_1$ accumulates the sum of interval lengths (and will be used to compute the average length), 
$L_2$ accumulates the squares of interval lengths,
and $\kappa$ is a counter that counts the number of accumulated beats.
$\sigma_2$ is assigned the value of the variance given by $\frac{1}{\kappa}[L_2 - L_1^2/\kappa]$
\begin{figure}[t]
	\centering
	\includegraphics[scale=0.3]{figures/stability1v2}
	\vspace{-10pt}
	\caption{Stability discriminator.}
	\vspace{-10pt}
	\label{fig:Hstab}
\end{figure}

\begin{lemma}
	\label{lemma:stability}
	$\Sys_{Stab}$ is STORMED.	
\end{lemma}

Now that each system was shown to be STORMED, it remains to establish that their parallel composition is STORMED.
This result does not hold in general - Thm.~\ref{thm:SHS composition} gives conditions under which parallel composition respects the STORMED property.
Intuitively, we require that whenever a sub-collection of the systems jumps, the remaining systems that did not jump are separated from all of their respective guards by a uniform distance.
This is a requirement that can be shown to hold for our systems by modeling various minimal delays in the systems' operation. 
%For example, when a $VEvent?$ is issued by $\Sys_{Sense}$, $\Sys_{VTC}$ does not jump and will wait at least until the sampling time of the next fiducial point to make a transition.
%Or, when an atrial cell fires (in $\Sys_{CA}$), we model a minimal delay between it and all other cells that do not fire simultaneously.
We may now state:
\begin{thm}
Consider the collection of systems $\Sys_{CA}$, $\Sys_{ICD} = \Sys_{Sense} || \Sys_{Detection-Algo}$ where the latter is the parallel composition of the discriminator systems.
This collection satisfies the hypotheses of Thm. \ref{thm:SHS composition} (Section \ref{sec:compositionality}) and therefore the parallel system  $\Sys_{CA} || \Sys_{ICD}$ is STORMED and has a finite bisimulation.
\end{thm}


\section{Properties of interest}
\label{sec:properties}
The finite simulation we obtain by running (the approximate version of) Alg.~\ref{algo:bisimulation} abstracts away the duration of continuous transitions $\trans{\tau}$, so that only time-unbounded properties (like LTL) can be verified on the abstraction.
%A priori, this is a shortcoming of the approach because many properties of interest for ICDs involve bounds on when events happen. 
However, every component of $\Sys_{ICD}$ has a local clock in its state vector.
Thus these clocks can be used to express interesting time-bounded properties of $\Sys_{ICD}||\Sys_{CA}$.

For example, to expresses that a Sustained \ac{VT} event should be followed by a \ac{VT} determination within 30sec, we write:
\begin{equation}
\label{eq:fTh}
\formula_{Th} \defeq \formula_{VT} \implies \formula_{VT} \until_{[0,30]} \Sys_{ICD}.Mode = VT
\end{equation}
The VT decision ($\Sys_{ICD}.Mode = VT$) can be reached by one of three paths of execution (see Fig. \ref{fig:bsc detection}):
For example, path $P_1$ goes from the root to ``8/10 faster'' on the right and ends in VT.
Along each path, the component automata have local clocks that keep track of how long they are running in this execution.
Therefore, the total execution time of all automata on a given path must be less than 30sec.
So the time constraint may now be expressed as the disjunction $\lor_{P \in \{P_1,P_2,P_3\}}\sum_{c\in P} c \leq 30$.
The formula can be re-written
\begin{equation*}
\formula_{VT} \implies \formula_{VT}\until \lor_{k=1,2,3}(\Sys_{ICD}.Mode = VT_k \land \sum_{c \in P_k} c \leq 30)
\end{equation*}
where $VT_k$ is the VT decision reached along the $k^{th}$ path.
%

%Letting VD denote a spontaneous (non-conducted) Ventricular Depolarization, the following formula describes a sequence of ventricular depolarizations that are 200ms or less apart.
%\begin{equation}
%\formula_{VF} \defeq VD \land \eventually_{[0,200]} (VD \land \eventually_{[0,200]} (VD \ldots))
%\end{equation}

%In terms of our model, a ventricular depolarization can be described as follows.
%Let $\Cc$ be a set of ventricular cells in the heart model that are meant to be the source of the depolarization, and let $\Nc(\Cc)$ be a neighborhood of these cells.
%The following expresses that the neighborhood is quiescent until the cells in $\Cc$ become active, implying spontaneous activity.
%Incidentally, this is a formula that does \emph{not} require time bounds, and so can be directly expressed in LTL.
%\begin{equation}
%VD \defeq \land_{c \in \Nc} (c.Mode = Quiescent) \until \land_{c\in \Cc} (c.Mode = Upstroke)
%\end{equation}
%Other ways of describing this event may be possible or preferable based on other factors.
%
%\todo[inline]{cite GF, Donze?}
%Signals used in TFL properties are also o-minimal (if the underlying time-domain signal is o-minimal), thus TFL properties can also be verified.
%Indeed, TFL properties use the Short-Time Fourier Transform (STFT) coefficients of the signal $x(t)$ to define predicates ($i=\sqrt{-1}$):
%\begin{eqnarray*}
%c_\omega(\tau) &=& \int_{\tau-L/2}^{\tau+L/2}x(t)g_L(t-\tau)e^{-i2\pi\omega t}dt
%\\
%&=& \int_{\tau-L/2}^{\tau+L/2}x(t)g_L(t-\tau)cos{2\pi\omega t}dt 
%\\
%&\quad& - i\int_{\tau-L/2}^{\tau+L/2}x(t)g_L(t-\tau)sin{2\pi\omega t}dt 
% \\
% &=& C_r(\tau+L/2) - C_r(\tau-L/2) 
% \\
% &\;& - i[C_i(\tau+L/2) - C_i(\tau - L/2)]
%\end{eqnarray*}
%Each of the summands is an integral. 
%The integrand in each is the product of a definable $x(t)$ and an analytic bounded $cos$ or $sin$ (bounded because multiplied by the window function $g_L$ of width $L$), and therefore it is definable. 
%So the antiderivatives $C_r,C_i$ are definable by a corollary of Spessegger \cite{Speissegger99_Pfaffian}.
%%($I \supset [\tau-L/2,\tau+L/2], a=\tau-L/2$), 
%So the real and imaginary parts of $c_\omega(\tau)$ are definable.
%If we identify the complex field $\Ce$ with $\Re^2$ and map the field operations of $\Ce$ to those of $\Re^2$ in the usual way, then $\Ce$ and its field operations are definable in $\Re$ and every algebraic subset of $\Ce^n$ is definable in $\Re$ \cite{PeterzilS04_ComplexOminmal}.
%Thus the function $\tau \mapsto c_\omega(\tau)$ is o-minimal being the result of applying multiplication (by $i$) and addition to definable terms.
\section{Composing STORMED systems}
\label{sec:compositionality}
The results in this section and the next apply to STORMED systems in general, including those with time-unbounded operation.
We write $[m] = \{1,\ldots,m\}$.
Given hybrid systems $\Sys_1,\ldots,\Sys_m$ in this section, $x^i, \guard^i, \theta^i,\ldots$ etc refer to a state, guard, flow $\ldots$ of system $\Sys_i$, $i\leq m$.
%
We show that the parallel composition of SHS is still a SHS.
Recall that $\theta_{\ell}(t;x)$ is the flow starting at $(\ell,x)$.
Given hybrid systems $\Sys_1,\ldots,\Sys_m$, their parallel composition $\Sys = \Sys_1 || \ldots ||\Sys_m$ is defined in the usual way:
$\Sys.\stSet = \Pi_i \stSet^i$,
$\Sys.\modeSet = \Pi_i \modeSet^i$,
$\Sys.\hsSet_0 =\Pi_i \hsSet_0^i$,
$Inv(\ell) = \Pi_{i}Inv^i(\ell^i)$,
$\theta_{\ell}(x,t)= [\theta_{\ell^1}^1(x^1,t)(t),\ldots,\theta_{\ell^m}^m(x^m,t)(t) ]^T$.
The system jumps if any of its subsystems jumps.
%, so its guard sets are of the form 
%$A^1\times\ldots \times A^m$ where for at least one $i$, $A^i$ is a guard of $\Sys_i$, and for the rest $A^j =\stSet^j$.
When a guard of a subsystem is satisfied, the state of that subsystem is reset according to its reset map.
The guards are made disjoint to avoid non-determinism.
%\yhl{A system $\Sys$ is \emph{deterministic} if to every initial state $(\mode,\stPt)$, $\Sys$ produces a unique trajectory starting there.}

In general $\Sys$ is not separable: indeed for any candidate value of $d_{min}$, one could find a transition $(i,j)$ of $\Sys$ due to, say, a jump of $\Sys_1$, s.t. at that moment $x^2$ is closer than $d_{min}$ to one of its own guards, say $\guard^2_{(j^2,k^2)}$. 
This causes $\Sys$ to further jump $j \rightarrow k$ without having traveled the requisite minimum distance, thus violating the separability of $\reset_{ij}(\guard_{ij})$ and $\guard_{jk}$.
Therefore we need to impose an extra condition on minimum separability \emph{across} sub-systems.
\begin{thm}
	\label{thm:SHS composition}		
	Let $\SHS_i = (\Sys_i, \Ac, \phi^i,b^{i,-},b ^{i,+}, d_{min}^i, \varepsilon^i, \zeta^i)$, $i=1,\ldots,m$ be deterministic SHS 
	defined using the same underlying o-minimal structure, 
	and where each state space $\stSet^i$ is bounded by $B_{X^i}$.
	\\
	Define parallel composition $\SHS = (\Sys, \Ac, \phi,b^-,b^+, d_{min}, \varepsilon, \zeta)$ where
	$\Sys = \Sys_1 || \ldots ||\Sys_m$,	
	$\phi = (\phi^1,\ldots,\phi^m)^T \in \Re^{mn}$,
	$b^{i,-} = \inf_{x \in \stSet} \phi\cdot x$,
	$b^{i,+} = \sup_{x \in \stSet} \phi\cdot x$,
	$\varepsilon = \min(\min_i \varepsilon^i, \min_i \frac{\zeta^i}{B_{X^i}})$,
	$\zeta = \min_i \zeta^i$ and
	\[d_{min} = \min_{I\subset [m]} (\min_{i\in I}d_{min}^i, \min_{i\in I ,j \in [m]\setminus I }d_{min}^{ij})\]	
	Assume that the following \textbf{Collection Separability} condition holds: 	
	\yhl{for all $i,j \leq m, \neq j $ there exists $d_{min}^{ij}>0$ s.t. 
		\yhl{if $\stPt \in \stSet$ is in the reachable set of $\Sys$} and 
		$x^i \in G^i_e \land x^j \notin G^j_{e'} \; \forall e' \in E^j$ 
		then $d(x^j,G^j_{e'}))>d_{min}^{ij}$ for all $e'\in E^j$ 
		where $E^j$ is the edge set of $\SHS_j$ and $G^j_{e'}$ is a guard of $\SHS_j$ on edge $e' \in E^j$.}
	Then $\SHS$ is STORMED.
\end{thm}

\section{Finite simulation for STORMED systems}
\label{sec:simulationAprox}
In general it is not possible to compute the reach sets required by the iteration \eqref{eq:Ft,Fd} exactly unless the underlying theory is decidable.
The $\Sys_{ICD}||\Sys_{CA}$ closed loop is definable in $\Lc_{\exp}$, and the latter is not known to be decidable.
%The authors in \cite{PrabhakarVVD09_toklerant} proposed approximating the flows and resets by polynomial flows and resets in the decidable theory $\Lc_\Re$.
%However, the approximation process is typically iterative and requires manual intervention, or is restricted to subclasses of STORMED systems \cite{PrabhakarVVD09_toklerant}.
Here we show that if an approximate reachability tool with definable over-approximations is available for the continuous dynamics, it can be used in \eqref{eq:Ft,Fd} to yield a finite \emph{simulation}.
Since we only have a simulation, counter-examples on the abstraction should be validated in a CEGAR-like fashion.
\begin{lemma}
	\label{lemma:finite simu}
	Let $\SHS = (\Sys,\ldots)$ be a SHS and $\sim$ an equivalence relation on $\stSet$.
	For any mode $\mode$ of $\Sys$, the dynamical system $\Dc$ with state space $X = \Sys.\stSet$ and set-valued flow $\Theta(t;x) = \{y \in \Re^n \;|\; ||y-\theta(t;x)||^2 \leq \epsilon^2\}$ admits a finite simulation $\simu_\mode$ that respects $\sim$.
\end{lemma}
Let $\Ft^\epsilon(\partition) \defeq \cap_{\mode}\simu_{\mode \in \modeSet}$ where $\partition = \stSet/\sim$. $\Ft^\varepsilon$ refines all the $\simu_\mode$'s, and it is a finite simulation of $\Sys$ by itself w.r.t. the continuous transition $\trans{\tau}$.

\begin{thm}
	\label{thm:finite simulation}
	Let $\Sys$ be a STORMED hybrid system, 
	and $\partition$ be a finite definable partition of its state space.
	Define 
	\begin{equation}
	\label{eq:Fte,Fde}
W_0 = \Ft^\epsilon(\partition), \quad \forall i\geq 0, W_{i+1} = \Ft^\epsilon(\Fd(W_i))
	\end{equation}
Then there exists $U \in \Ne$ s.t. $W_{U+1} = W_U$ and $\Ft^\epsilon(W_U)$ is a simulation of $\Sys$ by itself.
\end{thm}

\section{Conclusion}
In this paper, we presented the first formalization of a hybrid system model of the human heart and the \ac{ICD} device and showed that the resulting closed-loop may be formally verified. 
We showed that the heart model, the \ac{ICD} measurement process, the modules of common \ac{ICD}, and the parallel composition of the entire system to be a STORMED hybrid system, which admits finite bisimulation.
In the process, we were able to show that approximate reachability yields finite simulation for STORMED systems and that certain composition respect the STORMED property.
Finally, we showed that the reach set computed by SpaceEx may be used to build the simulation.


\bibliographystyle{abbrv}
\bibliography{HSCC2015_CompositionalConf,houssam,fainekos_bibrefs,hscc2016,biblio2}  %zhihao
% ACM needs 'a single self-contained file'!
%
%\clearpage
\appendix
%
%vtcextension
% tfl
{\large \textbf{Proof of Lemma \ref{lemma:stability}.}}
\begin{prf}
We show the resets are monotonic - the other properties are immediate.
The state is $x = (t,L_2,L_1,\kappa,\sigma_2)^T$.
The self-transition ACCUMULATE $\rightarrow$ ACCUMULATE is initiated by VEvent (ventricular peak).
At reset time, $0 \leq t \leq DL$, we have that 
$\phi\cdot(0-t,t^2,t,1,0)^T \geq -\phi_1 DL + \phi_4 \stackrel{Want}{\geq} \zeta$.

The transition ACCUMULATE $\rightarrow$ FINALIZE, initiated at the end of Duration, saves the value of the variance in $\sigma_2$.
This reset produces the constraint
$\phi_5 ((L_2 -L_1^2/\kappa)/\kappa) \geq \varepsilon |((L_2 -L_1^2/\kappa)/\kappa)|$.
But the quantity in absolute value is itself a variance and so is positive, therefore the constraint is simply $\phi_5 \geq \varepsilon$, compatible with the previous inequality.
\end{prf}
%\yhl{{\large \textbf{An equivalent characterization of Collection Separability}.}
	\\
Collection separability (Thm. \ref{thm:SHS composition}) can be equivalently expressed in terms of set operations. 
Let $R$ be the reachable set of $\Sys$,
and let $R_{|ij}$ be the projection of $R$ onto the space $\stSet^i \times \stSet^j$.
Then Collection Separability is expressed as:
\\
for all $i,j \leq m, i\neq j$ there exists $d_{min}^{ij}>0$ s.t. for all guards $\guard^i$ of $\Sys_i$ and all guards $\guard^j$ of $\Sys_j$, \begin{equation*}
R_{|ij} \cap \guard^i \times (\stSet^j \setminus \guard^j) \subset \stSet^i \times \{x^2 \in \stSet^2 \;|\; d(x^2,\guard^2) > d_{min}^{ij}\}
\end{equation*}}
{\large \textbf{Proof of Lemma \ref{lemma:finite simu}}.}
\begin{prf}
	This follows the lines of the elegant proof of \cite{BrihayeM05_ominimal} as formulated in \cite{tabuada} and generalizes it to set-valued maps.
	(The fact that using an approximate $Post$ operator yields a simulation is a special case of a more general result on transition systems but we prove it here for completeness. 
	\yhl{Also note that this result holds for o-minimal systems \cite{LaFerrierePS00_Ominimal} generally, not just STORMED systems}).
	
	First observe that using approximate reachability on a system $\Sys$ is tantamount to replacing $\Sys$ with a system $\Sys^\varepsilon$ whose flows and reset maps are set-valued $\varepsilon$ over-approximations of the flows and resets of $\Sys$ (but is otherwise unchanged).
	Therefore define the dynamical system $\Dc^\varepsilon$ with state space $\stSet$ and whose flow $\Theta: \Re \times \Re^n \rightarrow 2^{\Re^n}$ is a set-valued $\varepsilon$ over-approximation of $\theta_\mode$:
	$\Theta(t;x) = \{y \in \Re^n \;|\; ||y-\theta(t;x)||^2 \leq \epsilon^2\}$.	
	Let $\partition \defeq \stSet/\sim$ be the partition induced by $\sim$.
	%
	It follows from the definability of $\theta$ and $||\cdot||^2$ that $\Theta$ is definable. 
	Given $P \in \partition$, let $Z(P) = \Theta^{-1}(P) \defeq \{(x,t) \;|\; \Theta(x,t) \cap P \neq \emptyset\}$.
	Then $Z(P)$ is definable because $P$ and $\Theta$ are definable.
	Let $Z_x(P) = \{t \;|\; (x,t) \in Z(P)\} \subset \Re$ be the \emph{fiber} of $Z$ over $x$.
	The number of connected components of $Z_x(P)$ equals the number of times that $\Theta(x,t)$ intersects $P$.
	Now it follows from \cite{tabuada} Thm.7.11 that there exists a uniform upper bound on the number of connected components of $Z_x(P)$, independent of $x$.
	Let that bound be $V_P$.
	Thus $\Theta(x,t)$ visits $P$ at the most $V_P$ times, regardless of $x$.
	Since there is a finite number of blocks $P \in \partition$, then $\Theta(x,t)$ visits any block $P$ a maximum of $V \defeq \max_P(V_P)$ times.%, independent of $x$ and $P$.
	
	Thus we can associate to each $x\in \stSet$ a finite number of finite strings $q(x) = (\ell_1,\ell_2,\ldots,\ell_{i-1},\widehat{\ell_i},\ell_{i+1},\ldots,\ell_s)$, where $\ell_i,\widehat{\ell}_i \in \partition$.
	Each $q(x)$ gives the sequence of blocks that $\Theta(x,t)$ visits (with repetition), and in which $\widehat{\ell_i}$ is the block containing $x$.
	There may be more than one such string because the set $\Theta(x,t)$ might intersect more than one block of $\partition$ at a time.		
	The length of $q(x)$ is thus uniformly upper-bounded by $V\cdot |\partition|$, so there's a finite number of different strings $q(x)$. 
	%
	Let $\Qc(x)$ be the set of such strings associated to $x$, and let $\Qc = \cup_x \Qc(x)$.
	Then $\Qc$ is the state space of the finite transition system $K = (\Qc,\{*\},\trans{},\Qc_0)$ whose transition relation is 
	\begin{compactitem}
		\item $\ell_1\ldots\widehat{\ell}_i\ldots\ell_s \trans{*} \ell_1\ldots\widehat{\ell}_{i+1}\ldots\ell_s$
		\item $\ell_1\ldots\ell_{s-1} \widehat{\ell_s} \trans{*} \ell_1\ldots\ell_{s-1} \widehat{\ell}_s$
	\end{compactitem}

	It is clear that $K$ is non-deterministic and simulates $\Dc$ but is not a bisimulation because of the over-approximation produced by $\Theta$.	 
\end{prf} 
%Proof of Prop.~\ref{prop:spaceex definable}.
\begin{prf}
	\newcommand{\EPS}{\mathcal{E}}
	We show that if $S \subset \Re^n$ is o-minimal, then $\boxdot S$ and $TH_\Vc(S)$ are o-minimal. 
	O-minimality of the other sets is a standard result.
	
	We start with $\boxdot S$.
	\begin{eqnarray*}
		\boxdot S &=& \{y \in \Re^n \;|\; -\overline{|x_i|}\leq y_i \leq \overline{|x_i|},i=1,\ldots,n\}
		\\
		&=& \{y \in \Re^n \;|\; \exists a_i \in \Re_+. (a_i = \sup \{|x_i|,x\in S\} 
		\\
		&& \land (-a_i\leq y_i \leq a_i)),i=1,\ldots,n\}
		\\
		&=&  \{y \in \Re^n \;|\; \exists a_i \in \Re_+. (\forall \varepsilon_i>0 \exists x_i \in S . a_i-\varepsilon < |x_i(i)|)
		\\
		&& \land (-a_i\leq y_i \leq a_i)),i=1,\ldots,n\}
		\\
		&=&  \{y \in \Re^n \;|\; \exists a_i \in \Re_+. (\forall \varepsilon_i>0 \exists x_i \in S . a_i-\varepsilon < x_i(i)
		\\
		&& \lor a_i - \varepsilon > -x_i(i)) \land (-a_i\leq y_i \leq a_i)),i=1,\ldots,n\}
\end{eqnarray*}

The template hull is given by
\[TH_\Vc(X) \defeq \{x\in\Re^n \;|\;\land_{a \in \Vc} a\cdot x \leq \rho(a,X)\}\]
where the support function is given by 
\begin{equation}
\label{eq:supportfnt}
\rho(a,S) \defeq \max_{x \in S} a\cdot x
\end{equation}
Therefore $TH_\Vc(X)$ is definable if $\rho$ is.
The graph of $a \mapsto \rho(a,S)$ is given by 
\begin{eqnarray}
\textbf{Gph}\rho = \{(a,r) \;|\; \forall x\in S.\; r\geq a\cdot x \land \exists x\in S. r=a\cdot s\}
\end{eqnarray}
which is a first-order formula on predicates that use $+,\times$.
Since $S$ is o-minimal, Gph$\rho$ is o-minimal.
%
%To do so we show that it can be computed using a finite number of operations from the o-minimal structure $\Lc_{\exp}$.
%As we assume that $S$ is a polytope (a closed bounded intersection of half-spaces), it has a finite number of extreme points. 
%Let $\mathcal{E} =  \{e_i,i=1,\ldots,k\}$ be the set of extreme points and any point $x\in S$ can be written as a convex combination of the extreme points: $x = \sum_{i}\alpha_i e_i$.
%Therefore we can write
%\[ \rho(a,S) = \max_{\alpha \in I^k} \sum_{i=1}^k \alpha_i a\cdot e_i \]
%where $I^k = \{(\alpha_1,\ldots,\alpha_k)\;|\; \alpha_i \geq 0, \sum \alpha_i = 1\}$
%
%The maximizer $x^*$ of \eqref{eq:supportfnt} is necessarily a boundary point of $S$.
%Indeed let $x^0$ be an interior point of $S$ ($x^0 \in S\setminus \partial S$).
%Then the line through $x^0$ intersects $\partial S$ twice, at points $\lambda_1x^0$ and $\lambda_2x^0$, with $\lambda_1 \geq 1$ and $\lambda_2 \leq 1$.
%This is the case because $S$ is closed and bounded.
%If $a\cdot x^0 \geq 0$, then $a\cdot \lambda_1 x^0 \geq a\cdot x^0$. 
%And if $a\cdot x^0<0$ then $0 \geq a\cdot \lambda_2 x^0 \geq a\cdot x^0$. 
%Thus to any interior point of $S$ there exists a boundary point with a larger projection on $a$.
%
%A boundary point can always be written as the convex combination of exactly $n$ extreme points, namely the extreme points that are the vertices of the face to which the boundary point belongs.
%
%Therefore the support function expression is further simplified to:
%\[ \rho(a,S) = \max_{\alpha \in [0,1], \mathcal{F} \subset \EPS:|\mathcal{F}|=n } \sum_{i=1}^n \alpha_i a\cdot e_i \]
	\end{prf}

\end{document}
