\section{Finite simulation for STORMED systems}
\label{sec:simulationAprox}
%In general it is not possible to compute the bisimulation exactly unless the underlying o-minimal theory is decidable
Whether the bisimulations of STORMED systems can be computed depends on whether the underlying o-minimal theory is decidable. 
The human heart+ICD closed loop is definable in $\Lc_{exp}$, and the latter is not known to be decidable.
The authors in \cite{PrabhakarVVD09_toklerant} proposed approximating the flows and resets by polynomial flows and resets.
However, the process of approximation is typically iterative and requires manual intervention, or is restricted to subclasses of STORMED systems \cite{PrabhakarVVD09_toklerant}.
%More generally, building the abstraction might require explicit knowledge of the flows (solutions of the ODEs), which is often not possible.
%Collocation methods are more automatic and work with the ODEs directly, but they are restricted to sub-classes of STORMED systems and might still require manual intervention \cite{PrabhakarVVD09_toklerant}.
Even if the o-minimal theory is decidable, the decision procedure first applies Quantifier Elimination, whose worst-case time complexity is doubly exponential in the number of quantifier blocks \cite{DavenportH88_QE}.
%
Here we show that if an approximate reachability tool is available, it can be used in Algo~\ref{algo:bisimulation} (instead of exact reachability) to yield a \emph{finite simulation} (rather than a bisimulation).
%This is in the spirit of tolerant systems \cite{PrabhakarVVD09_toklerant}, but does not need explicit approximation of the flows.
%We take as our starting point the operators $\Fd$ and $\Fc$, and simply replace the exact $\Ft$ operator by an approximate operator $\Ft^\epsilon$ based on approximate reachability $\Rc^\epsilon_I$, as shown in the following Algorithm \ref{algo:apx Ft}.
%\begin{algorithm}
%	\caption{Simulation refinement for continuous transitions}
%	\label{algo:apx Ft}
%	\begin{algorithmic}
%		\Require Hybrid system $\Sys$, interval $I \subset [0,+\infty)$, a partition $\partition$ of $\hsSet$, $\epsilon>0$.
%		\State Set $\simu = \partition$			
%		\While{$\exists P,P' \in \simu$ s.t. $\emptyset \neq P' \cap \Rc^\epsilon_I(P) \cap P' \neq P'$}
%			\State Set $\simu = \simu \setminus \{P'\} \cup \{P' \cap \Rc_I^\epsilon(P) , P' \setminus \Rc_I^\epsilon(P) \}$
%		\EndWhile	
%		\State Return $\simu^\epsilon$
%	\end{algorithmic}
%\end{algorithm}
%It is clear that $\simu^\epsilon$ is a partition that refines $\partition$. 
%\begin{lemma}
%$\simu^\epsilon$ is finite.
%\end{lemma}
\begin{lemma}
	\label{lemma:finite simu}
	Let $\SHS = (\Sys,\ldots)$ be a SHS and $\partition$ a partition of $\stSet$.
	For any mode $\mode$ of $\Sys$ with flow $\theta_\mode $, consider the dynamical system $\Dc$ with state space $X = \Sys.\stSet$ and whose flow $\Theta: \Re \times \Re^n \rightarrow 2^{\Re^n}$ is a set-valued $\varepsilon$ over-approximation of $\theta_\mode$:
	$\Theta(t;x) = \{y \in \Re^n \;|\; ||y-\theta(t;x)||^2 \leq \epsilon^2\}$.	
	Then $\Dc$ admits a finite simulation $\simu_\mode$ that refines $\partition$, returned by Alg.~\ref{algo:bisimulation}.
\end{lemma}
The proof is in the Appendix.
%====================================================================================
%This finite simulation is that returned by running Algorithm 1 with the approximate reachability. 
%The lemma can be repeated for all modes of $\Sys$ producing a simulation $\simu_\mode$ for each mode $\mode \in \modeSet$.
Let $\Ft^\epsilon(\partition) \defeq \cap_{\mode}\simu_{\mode \in \modeSet}$: then it refines all the $\simu_\mode$'s, and it is a finite simulation of $\Sys$ by itselt w.r.t. the continuous transition $\trans{\tau}$.
%Let $\simue$ be the finite simulation of $\Sys$ thus obtained, and let $\partition_{\simue}$ be the associated partition of $\hsSet$.
%$\Ft^\epsilon(\partition) \defeq \{(\stPt,P)  \in \hsSet \times \simu^\epsilon \;|\; \stPt \in P\}$.
It is clear that $\Ft^\epsilon(\cdot)$ is idempotent: $\Ft^\epsilon(\Ft^\epsilon(\partition)) = \Ft^\epsilon(\partition)$
%We also have:
%\begin{lemma}
%For any partition $\partition$, $\Ft^\epsilon(\partition)$ is a simulation of $\Sys$ by $T_\Sys/\simu^\epsilon$.
%\end{lemma}
%\begin{proof}
%Two states $x,x' \in \stSet$ are related by $\simu^\varepsilon$ iff $q(x) = q(x')$, that is, their flows visit the same sequence of blocks of $\partition$.
%Do they also visit the same sequence of blocks in $\partition_{\simue}$?
%Assume not, and let $B,B' \in \partition_{\simue}$ be the first blocks in which they differ, that it, $\theta(x,t) \in B$ and $\theta(x',t) \in B'$.
%Because $B,B'$ are different blocks, this implies that there exist $y\in B,y' \in B'$ such that $q(y) \neq q(y')$.
%But $x \in B \implies q(x) = q(y)$, and similarly $x' \in B' \implies q(x')=q(y')$, which contradicts $q(x) = q(x')$.
%Therefore the flows of $x$ and $x'$ visit the same sequence of blocks in $\partition_{\simue}$.
%Thus if we re-apply $\Ft^\varepsilon$ to $\simu^\varepsilon$, it will again yield $x$ and $x'$ will again have the same strings and so remain related by $\Ft^\varepsilon()$
%%Idempotency is immediate: Algorithm \ref{algo:apx Ft} terminates only when no further partitioning of any block $P$ is possible. Thus if we re-run it on $\simu^\epsilon$, it will return $\simu^\epsilon$ and the resulting relation $\Ft^\epsilon$ is unchanged.
%
%Let $\simu^\epsilon = \{P_1,\ldots,P_V\}$ be the final refinement of $\partition$ returned by Algorithm \ref{algo:apx Ft}.
%Write $F \defeq \Ft^\epsilon(\partition)$.
%Let $(x,P) \in F$, that is, $x \in P$. 
%Consider a continuous transition $x \trans{\tau} y$ such that $y = \theta(x,t)$.
%Then by definition $y \in \Theta(x,t) \subset P'$ for some $P' \in \simu^\epsilon$. 
%Moreover $P \trans{} P'$ by definition of the transition relation of $T_\Sys/\simu^\epsilon$, so $(y', P') \in F$.
%
%The converse is not necessarily true: if $P \trans{}P'$ then it might be that $P' = \Rc^\epsilon_I(P) \setminus \Rc^0_I(P)$, i.e. $P'$ is purely the result of the over-approximation.
%In this case, it can not be guaranteed that $x \in P$ will evolve into $P'$.
%This is why $\Ft^\epsilon$ is only a simulation.
%\end{proof}

%====================================================================================

\begin{thm}
	\label{thm:finite simulation}
	Let $\Sys$ be a STORMED hybrid system, 
	and $\partition$ be a finite definable partition of its state space.
	Define 
	\begin{compactitem}
		\item $W_0 = \Ft^\epsilon(\partition)$.
		\item For all $i\geq 0$, $W_{i+1} = \Ft^\epsilon(\Fd(W_i))$.
	\end{compactitem}
Then there exists $U \in \Ne$ s.t. $W_{U+1} = W_U$ and $\Ft^\epsilon(W_U)$ is a simulation of $\Sys$ by itself.
\end{thm}

\begin{prf}
By Lemma 10 of \cite{VladimerouPVD08_STORMED} there exists a uniform bound $U$ on the number of discrete transitions of any execution of the STORMED system $\Sys$, so $\Fd(W_k) = W_k$ for all $k\geq U$.
Moreover 
$W_{U+1} = \Ft^\epsilon(\Fc_d(W_U)) = \Ft^\epsilon(W_U)$
and $W_{U+2}= \Ft^\epsilon(\Fc_d(W_{U+1}))= \Ft^\epsilon(\Fc_t(W_U)) = \Ft^\epsilon(W_U) = W_{U+1}$, so the iterations reach a fixed point.
The fact that $\Ft^\epsilon(W_U)$ is a simulation then yields the desired result.
\end{prf}
%
%Let $AP$ be a set of atomic propositions, and let $\formula$ be an LTL formula over $AP$.
%If the initial partition $\partition$ we start with is given by $x \equiv_\partition y$ iff $(x \models p \leftrightarrow y \models p \; \forall p \in AP)$, then the fact that simulation $\Sys^\epsilon$ satisfies $\formula$ implies that the original hybrid system $\Sys$ satisfies it too. 
\subsection{Example: SpaceEx reachable sets}
\label{sec:spaceex}
Lemma \ref{thm:finite simulation} required that the over-approximation sets $\Rc^\epsilon_{t}(\{{x}\})$ be definable for every $x$ and $t$ (see proof).
In practice, we need to show that the over-approximation \emph{actually computed by the reachability tool} (which may not be the full ball $\Rc^\epsilon_{t}(x)$) is definable.
In this section we show that the over-approximations computed by SpaceEx \cite{FrehseCAV11} are definable.
Given the set $X\subset \Re^n$ and finite $\Vc \subset \Re^n$, parameter $\lambda \in [0,1]$ a time step $\delta>0$, and $(i,j) \in E$, 
SpaceEx over-approximates $\reset_{ij}(X)$ by $\Kc(\Vc,X) \defeq \reset_{ij}(TH_\Vc(X)\cap G_{ij})\cap Inv(j)$ and $\Rc_{\lambda \delta}^\epsilon(X)$ by \cite{FrehseCAV11}:
\begin{eqnarray}
\Omega_\lambda(X,\delta) &=& (1-\lambda)X \oplus e^{\delta A} X 
\nonumber \\
&\oplus&(\lambda E_\Omega^+(X,\delta) \cap (1-\lambda) E_\Omega^-(X,\delta))
\end{eqnarray}
where
$TH_\Vc(X) \defeq \{x\in\Re^n \;|\;\land_{\vec{a} \in \Vc} \vec{a}\cdot x \leq \rho(\vec{a},X)\}$ is the template hull of $X$ and $\rho$ its support function,
$E_\Omega^+ = \boxdot (\Phi_2 \boxdot(A^2 X)$,
$E_\Omega^- = \boxdot (\Phi_2 \boxdot(A^2 e^{\delta A}X))$,
$\oplus$ is the Minkowski sum,
$\boxdot S =  [-\overline{|x_1|}, \overline{|x_1|} ] \times \ldots \times [-\overline{|x_n|}, \overline{|x_n|} ]$ is the box hull
with $\overline{|x_i|} \defeq \max\{|x_i| \text{ s.t. } x=(x_1,\ldots,x_n) \in S\}$.

\begin{thm}
	\label{thm:spaceex definable}
	For all definable polytopes $X \subset \Re^n$, the sets $\Kc(\Vc,X)$ and $\Omega_\lambda(X,\delta)$ is definable are $\Lc_{\exp}$.
\end{thm}
	
\begin{prf}
Let $S, Y \subset \Re^n$ be two definable sets in some o-minimal structure $\Ac$.
Let $\lambda \in \Re$ and let $A$ be a real matrix.
Then the following sets are also o-minimal: $\lambda S$, $A S$, $S \cap Y$, $S \oplus Y$, $S \cap Y$, $TH_\Vc(S)$ and $\boxdot S$.
Now the result follows by noting that $\Kc(\Vc,X)$ and $\Omega_\lambda(X,\delta)$ are constructed by composing the above definability-preserving operations.
\end{prf}



