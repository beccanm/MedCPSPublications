\section{Hybrid systems and simulations}
\label{sec:preliminaries}

%\subsection{Hybrid systems}
%\label{sec:hybridSystems}
\begin{defn}
	\label{defn:hybrid system}	
	A \emph{hybrid automaton} is a tuple \[\Sys = (\stSet,\modeSet,\hsSet_0,\{f_\mode\}, Inv,E, \{\reset_{ij}\}_{(i,j)\in E}, \{\guard_{ij}\}_{(i,j)\in E})\] where 
		 $\stSet \subset \Re^n$ is the continuous state space equipped with the Euclidian norm $\|\cdot\|$, 
		$\modeSet \subset \Ne$ is a finite set of modes,
		 $\hsSet_0 \subset \stSet \times \modeSet$ is an initial set,
		 $\{f_\mode\}_{\mode \in \modeSet}$ determine the continuous evolutions with unique solutions,
		 $Inv: \modeSet \rightarrow 2^\stSet$ defines the invariants for every mode,
		 $E \subset \modeSet^2$ is a set of discrete transitions,
		 \yhl{$\guard_{ij} \subset \stSet$ is guard set for the transitions (so $\Sys$ transitions $i \rightarrow j$ when $\stPt \in \guard_{ij}$),
		 $\reset_{ij}: \stSet \rightarrow \stSet$ is an edge-specific reset function.}
		 \\
		 Set $\hsSet = \modeSet\times \stSet$.
		 Given $(\mode,\stPt_0) \in \hsSet$, the \emph{flow} $\theta_{\mode}(;\stPt_0):\Re_+ \rightarrow \Re^n$ is the solution to the IVP $\dot{x}(t) = f_\mode (x(t))$, $\stPt(0)=\stPt_0$.
\end{defn}
%
The associated \emph{transition system} is $T_\Sys = (\hsSet,  E \cup \{\tau\},\trans{},\hsSet_0)$ 
where $\hsSet$ is the state set, $E \cup \{\tau\}$ is the label set for transitions, $\hsSet_0$ is the set of initial states, 
and $\trans{} = (\bigcup_{e \in E} \trans{e}) \cup \trans{\tau}$ 
where $(i,\stPt) \trans{e} (j,y)$ iff $e = (i,j), \stPt \in \guard_{ij}, y = \reset_{ij}(\stPt)$ and $(i,\stPt) \trans{\tau} (j,y)$ iff $i = j$ and there exists 
a flow $\theta_i(\cdot;x)$ of $\Sys$ and $t\geq 0$ s.t. $\theta_i(t;x)=y$ and $\forall t' \leq t$, $\theta_i(t';x) \in Inv(i)$.
%Note that the transition $\trans{\tau}$ abstracts away time, i.e. it doesn't preserve information about the duration of continuous flow.
%For a set $P \subset \hsSet$,$P_{|\stSet}$ denotes its projection onto $\stSet$, 
%and $P_{|\modeSet}$ its projection onto $\modeSet$. 
%\begin{defn}
%	\label{defn:reachability operators}[Reachability]
%	Let $\Sys$ be a hybrid system with hybrid state space $\hsSet$, 	 
%	$I = [0,b) \subset [0,+\infty)$ be a (possibly unbounded) interval, 
%	$t \in I$, 
%	and $\epsilon >0$.
%	The \emph{$\epsilon$-approximate continuous reachability operator}, 
%	$\Rc^{\epsilon}_t : 2^\hsSet \rightarrow 2^\hsSet$ is given by
%	\begin{eqnarray*}
%		\Rc^{\epsilon}_t(P) = \{(i,\stPt) \in \stSet | \exists x_0 \in P_{|\stSet}, t \geq 0. 
%		||\theta_i(t;x_0) - \stPt|| \leq \epsilon\} 
%	\end{eqnarray*}
%	where $P = \{i\}\times W$, $W \subset Inv(i)$.
%	%
%	Define also $\Rc^{\epsilon}_I(P) = \cup_{t\in I} \Rc^{\epsilon}_t(P)$.
%	%
%	The (exact) \emph{discrete reachability operator} is:
%	\begin{eqnarray*}
%	\Rc_{d}(P) &=& \cup_{j: (i,j) \in E} \reset_{ij}(P \cap G_{ij})
%	\end{eqnarray*}
%\end{defn}
%%
%For a hybrid system, $Post_\slabel$ computes the forward reach sets, and is implemented by $\Rc^0_{[0,\infty)}$ and $\Rc_d$. 
Let $\sim$ be an equivalence relation on $\hsSet$ and $\hsSet/\sim$ the corresponding partition.
Let $\Fc_t(\hsSet/\sim)$ be the coarsest bisimulation with respect to $\trans{\tau}$\footnote{I.e., $\Ft$ only considers the continuous transition relation: it is a bisimulation of $T_\Sys^c \defeq (\hsSet/\sim,\{*\},\trans{\tau},\hsSet_0/\sim)$.} 
respecting the partition $\hsSet/\sim$, 
and 
$\Fc_d(\hsSet/\sim) \defeq \{(h_1,h_2)  \;|\; (h_1 \trans{e} h_1') \implies (\exists e' \in E, h_2' \; . h_2 \trans{e'} h_2' \land h_1' \sim h_2') \} \cap \hsSet/\sim$ \cite{VladimerouPVD08_STORMED}.
The iteration 
\begin{eqnarray}
\label{eq:Ft,Fd}
W_0 = \Ft(\hsSet/\sim), \quad\forall i\geq 0,\; W_{i+1} = \Ft(\Fd(W_i))
\end{eqnarray}
computes a bisimulation of $\Sys$.
However it does not necessarily terminate for hybrid systems because the system's reach set might intersect a given block of $\hsSet/\sim$ an infinite number of times (see \cite{LaFerrierePS00_Ominimal} for an example).
The class of systems introduced in the next section has the property that the iteration does terminate for it and returns a finite $\simu$.

Given a set of atomic propositions, if $\sim$ is s.t. $\hsPt \sim \hsPt'$ iff both states satisfy exactly the same atomic propositions, then model checking temporal logic properties can be done on the finite bisimulation instead of the possibly infinite $\Sys$.

\subsection{O-minimality and STORMED systems}
\label{sec:ominimality}
We give a very brief introduction to o-minimal structures.
A more detailed introduction can be found in \cite{LaFerrierePS00_Ominimal} and references therein.
We are interested in sets and functions in $\Re^n$ that enjoy certain finiteness properties, called order-minimal sets (o-minimal).
These are defined inside \emph{structures} $\Ac = (\Re,<, +,-,\cdot,\exp,\ldots)$.
The subsets $Y \subset \Re^n$ we are interested in are those that are \emph{definable} using first-order formulas $\formula$: $Y = \{(a_1,\ldots,a_n) \in \Re^n \;|\;  \formula(a_1,\ldots,a_n)\}$.
(First-order formulas use the boolean connectives and the quantifiers $\exists,\forall$).
The atomic propositions from which the formulas are recursively built allow only the operations of the structure $\Ac$ on the real variables and constants, and the relations of $\Ac$ and equality.
For example $2x-3.6y < 3z$ and $x=y$ are valid atomic propositions of the structure $\Lc_\Re=(\Re,<, +,-,\cdot)$, while $cosh(x) < 3z$ is not because $cosh$ is not in the structure.
These structures are already sufficient to describe a set of dynamics rich enough for our purposes and for various classes of linear systems.
%\textbf{Formal definitions}.
A \emph{language} is a set of relations, functions and constants.
E.g. $\Lc_\Re = (<,+,-, \cdot,\Re)$ is the language where the only relation is $<$, the functions are $+,-,\cdot$, and the constants are taken from $\Re$,
and  $\Lc_{exp} = (<,+,-, \cdot,\exp, \Re)$ adds the exponential symbol to it.
Let $\Vc = \{x,y,z,x_1,x_2,\ldots\}$ be a set of variables.
A \emph{term} of a language is either a variable, a constant, or $F(\theta_1,\ldots,\theta_m)$ where $F$ is an $m$-ary function expressible in the language and the $\theta_i$ are terms.
E.g. $2x-3.6y$ is a term of $\Lc_\Re$.
The \emph{atomic formulas} of the language are of the form $\theta_1=\theta_2$ or $R(\theta_1,\ldots,\theta_m)$ where $R$ is an $m$-ary relation on the terms $\theta_i$.
E.g., $2x-3.6y < z\cdot z$ is a an atomic formula in $\Lc_\Re$ which uses the terms $2x-3.5y$ and $z \cdot z$.
The set of (first-order) \emph{formulas} is defined recursively as follows: every atomic formula is a formula, and if $\formula_1,\formula_2$ are formulas, then so are $\formula_1 \land \formula_2, \neg \formula_1, \forall x: \formula_1, \exists x:\formula_1$.
Here the symbols are the usual boolean connectives ( conjunction $\land$ and negation $\neg$), and quantifiers (there exists $\exists$ and for all $\forall$).
A \emph{sentence} in the language is a formula where all variables are inside the scope of a quantifier, e.g. $\exists y . (x+y>3)$ is not a sentence.

So far, a language has been defined in a purely syntactic manner.
A \emph{model} of a language is a set $S$ along with an interpretation of the relations, functions and constants of the language.
We are interested in this paper in the model of $\Lc_{exp}$ where the set $S = \Re$, and the symbols $<,+,-, \cdot,exp$ have their usual meaning (less than, addition, etc).
A set $Y \subset \Re^n$ is \emph{definable} in $\Lc_\Re$ if there exists a formula of the model such that $Y$ can be expressed as $Y = \{(a_1,\ldots,a_n) \in \Re^n \;|\; \formula(x_1,\ldots,a_n)\}$.
E.g., the formula $x^2-1=0$ defines the set $\{-1,+1\}$.
Finally, a \emph{theory} is a subset of sentences in the model $\Lc_\Re$, i.e. it is a collection of sentences that are true in $\Lc_\Re$.
\begin{defn}
	\label{defn:ominimal struct}	
	A theory of $(\Re,\ldots)$ is \emph{o-minimal} if the only definable subsets of $\Re$ are finite unions of points and (possibly unbounded) intervals.	
	A function $f:x \mapsto f(x)$ is o-minimal if its graph $\{(x,y) \;|\; y=f(x)\}$ is a definable set.
\end{defn}
We use the terms o-minimal and definable interchangeably, and they refer to $\Lc_{\exp}= (\Re,<, +,-,\cdot,\exp)$ which is known to be o-minimal.
%
The dot product between $x,y\in \Re^n$ is denoted $x \cdot y$, and $d(Y,S)=\inf \{ \|y-s\| \;|\; (y,s) \in Y\times S \}$.
\begin{defn}\cite{VladimerouPVD08_STORMED}.
	\label{defn:stormed system}	
	A \emph{STORMED hybrid system} (SHS) $\SHS$ is a tuple $(\Sys,\Ac, \phi,b_-,b_+, d_{min}, \epsilon,\zeta)$ where $\Sys$ is a hybrid automaton, $\Ac$ is an o-minimal structure, $d_{min}, \epsilon, \zeta$ are positive reals, $b_-,b_+ \in \Re$ and $\phi \in \stSet$ such that:
	\\
	\textbf{(S)} The system is $d_{min}$-separable, meaning that for any $e=(\mode, \mode ')\in E$ and $\mode ''\neq \mode'$,$d(\reset_e(\guard_{(\mode ,\mode ')}), \guard_{(\mode' ,\mode '')})>d_{min}$
	\footnote{\yhl{The original definition of separability \cite{VladimerouPVD08_STORMED} required the guards themselves to be separated, which is insufficient to guarantee that if $\Sys$ flows, it flows a uniform minimum distance along $\phi$. Indeed assume the guards are separated. If $x \in \guard_{(\mode,\mode')}$ and $y = \reset_{(\mode,\mode')}(x)$, it can be that $y \in \guard_{(\mode',\mode'')}$ and thus a jump happens, even though $\guard_{(\mode,\mode')}$ and $\guard_{(\mode',\mode'')}$ are separated. Therefore we need $d(y,\guard_{\mode',\mode''}) > d_{min}$ for all $y \in \reset_e(\guard_e)$, which is the condition we use in Def.~\ref{defn:stormed system}. The properties of SHS, in particular the existence of finite bisimulation, are therefore preserved by this change.}}
	\\
	\textbf{(T)} The flows (i.e., the solutions of the ODEs) are Time-Independent with the Semi-Group property (TISG), meaning that for any $\mode \in \modeSet, \stPt \in \stSet$, the flow $\theta_\mode$ starting at $(\mode , x)$ satisfies: 1) $\theta_{\mode}(0;x) = x$, 2) for every $t,t' \geq 0$, $\theta_{\mode}(t+t';x) = \theta_{\mode}(t'; \theta_\mode(t;x))$
	\\
	\textbf{(O)} All the sets and functions of $\Sys$ are definable in the o-minimal structure $\Ac$
	\\
	\textbf{(RM)} The resets and flows are monotonic with respect to the same vector $\phi$, meaning that \\
	1) (Flow monotonicity) for all $\mode \in \modeSet$, $x \in \stSet$ and $t,\tau \geq 0$, $\phi \cdot (\theta_\mode (t+\tau;x) - \theta_\mode (t;x) ) \geq \epsilon ||\theta_\mode (t+\tau;x) - \theta_\mode (t;x) ||$, 
	and \\
	2) (Reset monotonicity) for any edge $(\mode,\mode') \in E$ and any $x^-,x^+ \in \stSet$ s.t. $x^+ = R_{\mode ,\mode'}(x^-)$, 
	\begin{compactenum}
		\item if $\mode = \mode'$, then either $x^-=x^+$ or $\phi \cdot (x^+-x^-)\geq \zeta$
		\item if $\mode \neq \mode'$, then $\phi\cdot (x^+-x^-) \geq \epsilon ||x^+-x^-||$
	\end{compactenum}

	\textbf{(ED)} Ends are Delimited: for all $e \in E$ we have $\phi \cdot x \in (b_- , b_+)$ for all $x \in G_{e}$
\end{defn}
Intuitively, the above conditions imply the trajectories of the system always move a minimum distance along $\phi$ whether flowing or jumping, which guarantees that no area of the state space will be visited infinitely often. 
This is at the root of the finiteness properties of STORMED systems.
\begin{thm}\cite{VladimerouPVD08_STORMED}
	\label{thm:stormed finite bisimu}	
	Let $\Sys$ be a STORMED hybrid system, and let $\partition$ be an o-minimal partition of its hybrid state space. 
	Then $\Sys$ admits a finite bisimulation that respects $\partition$.
\end{thm}
