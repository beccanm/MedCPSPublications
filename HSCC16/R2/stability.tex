\subsection{Stability discrimination}
\label{sec:stability}
%\begin{figure}[t]
%	\centering
%	\includegraphics[scale=0.35]{figures/EGMStability}
%	\caption{Examples of a unstable rhythm (top) and stable rhythm (bottom).}
%	\label{fig:stable unstable}
%\end{figure}
\emph{Stability} refers to the variability of the peak-to-peak cycle length.
A rhythm with large variability (above a pre-defined threshold) is said to be \emph{unstable}, and is called stable otherwise.
The Stability discriminator is used to distinguish between atrial fibrillation, which is usually unstable, and \ac{VT}, which is usually stable.
%Atrial fibrillation, which is usually unstable, might induce a high ventricular rate .
%A \ac{VT}, on the other hand, is usually stable.
%Therefore this is a useful discriminator.

The Stability discriminator shown in Fig. \ref{fig:Hstab} simply calculates the variance of the cycle length over a fixed period called a Duration (measured in seconds).
Let $DL \geq 0$ be the Duration length.
\yhl{The events $DurationBegins?$ and $DurationEnds?$ indicate the transitions} of a simple system that measures the lapse of one Duration (not shown here).
State $t$ is a clock, $L_1$ accumulates the sum of interval lengths (and will be used to compute the average length), 
$L_2$ accumulates the squares of interval lengths,
and $\kappa$ is a counter that counts the number of accumulated beats.
$\sigma_2$ is assigned the value of the variance given by $\frac{1}{\kappa}[L_2 - L_1^2/\kappa]$
\begin{figure}[t]
	\centering
	\includegraphics[scale=0.3]{figures/stability1v2}
	\vspace{-10pt}
	\caption{Stability discriminator.}
	\vspace{-10pt}
	\label{fig:Hstab}
\end{figure}

\begin{lemma}
	\label{lemma:stability}
	$\Sys_{Stab}$ is STORMED.	
\end{lemma}

Now that each system was shown to be STORMED, it remains to establish that their parallel composition is STORMED.
This result does not hold in general - Thm.~\ref{thm:SHS composition} gives conditions under which parallel composition respects the STORMED property.
Intuitively, we require that whenever a sub-collection of the systems jumps, the remaining systems that did not jump are separated from all of their respective guards by a uniform distance.
This is a requirement that can be shown to hold for our systems by modeling various minimal delays in the systems' operation. 
%For example, when a $VEvent?$ is issued by $\Sys_{Sense}$, $\Sys_{VTC}$ does not jump and will wait at least until the sampling time of the next fiducial point to make a transition.
%Or, when an atrial cell fires (in $\Sys_{CA}$), we model a minimal delay between it and all other cells that do not fire simultaneously.
We may now state:
\begin{thm}
Consider the collection of systems $\Sys_{CA}$, $\Sys_{ICD} = \Sys_{Sense} || \Sys_{Detection-Algo}$ where the latter is the parallel composition of the discriminator systems.
This collection satisfies the hypotheses of Thm. \ref{thm:SHS composition} (Section \ref{sec:compositionality}) and therefore the parallel system  $\Sys_{CA} || \Sys_{ICD}$ is STORMED and has a finite bisimulation.
\end{thm}