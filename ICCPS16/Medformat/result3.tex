 \subsection{Effect of Device Parameters on Discriminating Capability}
ICDs have a number of parameters which can be tuned to accommodate specific patient conditions by the physicians. 
%Currently there are very few clinical results on the effect of tuning parameters and their effect on sensitivity and specificity~\cite{maditrit}.
%One of the main causes of VT/SVT mis-classifications is inappropriate parameter settings~\cite{wrong_sensing}.
%In order for the physicians to set appropriate parameters, it is very important to understand how the change of one parameter can affect the discriminating capability of the device.
%With MBCT, one can use the same population across multiple devices with different parameter settings at virtually no cost. 
%The resulting trends can provide valuable insights to physicians.
In this section, we use MBCT to demonstrate the effects of changing two common parameters on SVT/VT discrimination specificity.
The first parameter is the \emph{\textbf{duration}} of arrhythmia before the ICD makes a therapy decision.
It is measured in seconds for Boston Scientific's devices, and in number of beats for Medtronic's devices. 
For Boston Scientific ICD the value can be set to 1 to 30 seconds.
%In this experiment we explore the values \{1,2,3,4,5,8,10\}.
%The equivalent parameter for Medtronic ICD is the number of consecutive fast ventricular intervals which can be set from 8 to 20 beats.
%In this experiment we explore the values \{8,10,12,16,18,24,30\} which roughly correspond to the parameters of Boston Scientific ICD.
%Intuitively, with a longer duration the device can examine a longer history of the arrhythmia episode, which can prevent inappropriate therapy, and thus increase SVT/VT discrimination specificity.
%Setting the duration too long can also cause missed therapies, thus affecting sensitivity. 
%These results are in agreement with the recently conducted ADVANCE-III RCT which showed that longer arrhythmia detection windows reduce shocks for Medtronic ICDs~\cite{advance3}.
%
The second parameter we vary is the \emph{\textbf{VF threshold}}.
For both devices, if the ventricular rate is faster than the VF threshold for a period of time the devices will deliver therapy without going into the SVT/VT detection algorithm. 
So a higher VF threshold means that more signals are passing through the discrimination algorithm.
%The value can be set to 150 to 200BPM.
%In this experiment we explore the value \{170,184,200\} for both devices.
%Intuitively the higher the threshold, the more episodes will be examined by the SVT/VT discrimination algorithm, which may increase specificity.
%However, VTs with rate less than the threshold may also be classified as SVT, causing missed therapies.\\

For each of the 21 parameter combinations described above, we ran a MBCT with 11,400 EGM episodes on both device models. 
Results are shown in Fig. \ref{fig:parameter}.
From the results we observe that for both devices the specificity increases monotonically with the length of the duration.
When the duration is longer than 5, sensitivities also dropped below 100\%, which is in line with the intuition.

\begin{figure}[t]
		\centering
		\vspace{-10pt}
		\includegraphics[width=0.45\textwidth]{figures/parameter.pdf}
		\caption{\small Effects of Duration and VF threshold parameters on Specificity}
		\vspace{-10pt}
		\label{fig:parameter}
\end{figure}

However, Boston Scientific algorithm and Medtronic algorithm displayed opposite trends for VF threshold.
%This agrees with the RIGHT finding that the rate of inappropriate therapy is highest for \acp{VT} with a rate $\leq 175$bpm \cite{GoldABBTB11_RIGHTresults}.
For Medtronic algorithm, the specificity increases when the VF threshold increases from 170BPM to 184BPM - i.e. a higher threshold admits more signals through the discrimination algorithm which performs better across all rates.
For Boston Scientific algorithm the specificity dropped when the VF threshold increases from 170BPM to 184BPM - i.e. the discrimination algorithm is less effective at higher rates. 
One possible interpretation of the result is that the Boston Scientific algorithm is more prone to inappropriate therapies for SVTs with ventricular rate between 170BPM to 184BPM, which is a very useful insight for the physicians to consider during parameter settings. 
%It should be noted that ICD settings are complex and studies across multiple parameter settings would need to be considered to provide conclusive guidance.







