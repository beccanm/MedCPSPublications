 \subsection{Patient Data Adjudication and EGM Template Extraction}
In order to obtain realistic morphologies for our simulations we utilize the Ann Arbor Electrogram Libraries (AAEL), a database of over 500 \ac{EGM} recordings made during clinical electrophysiology studies~\cite{AAEL}. 
The AAEL is used by all major \ac{ICD} manufacturers and is licensed by the US FDA. 
The AAEL provides descriptive annotations of records at a high level.
We performed additional detailed examination to precisely segment each record according to rhythm type.
123 records from 47 patients were manually examined and adjudicated into segments called \emph{episodes} containing one specific rhythm, e.g.\, \ac{NSR} or \ac{VF}. 
The adjudication was performed by a cardiologist.
Fig. \ref{fig:adjudication} (left) shows an example record (Record A185660) which has undergone this adjudication.
%It should be noted that only with a standard 12-lead \ac{ECG} in addition to ICD signals can all types of arrhythmia be accurately diagnosed.
From each episode, we developed an automated process which extracted \ac{EGM}s from a given episode. 
The \ac{EGM} are collected and organized by both patient record and by the type of rhythm which was annotated during the adjudication process.
These extracted rhythm \emph{signatures} provide the basis for the morphology information in the signal generated by our model.
Fig. \ref{fig:adjudication} (right) depicts an example of 10 signatures extracted from the record. 

\begin{figure*}[t]
	\centering
	\vspace{-10pt}
	\includegraphics[scale=0.35]{figures/figadjudication.pdf}
	\vspace{-10pt}
	\caption{\small  (Left) The \ac{EGM} record is segmented into episodes with distinct rhythms in each. (Right) From each episode, individual \acp{EGM} morphologies are extracted and stored.
	}
	\label{fig:adjudication}
\end{figure*}
%(Left) Adjudication of record A185660 
%(right) Examples of \ac{EGM} morphology signatures extracted from record: SA - Sinoatrial Node Event; Retrograde - Atrial event from Ventricular Retrograde; A2V - Sinus Atrial to Ventricular Conduction; A2V Shock - Shock signal of Sinus Atrial to Ventricular Conduction; PVC - Premature Ventricular Contraction Event; PVC Shock - Shock signal of Premature Ventricular Contraction Event; VT - \ac{VT} event; VT Shock - Shock signal of \ac{VT} event;  VF - \ac{VF} event; VF Shock - Shock signal of \ac{VF} event
%Our starting point is the AAEL database of \ac{EGM}s \cite{AAEL}.
%These are \ac{EGM} records collected from real patients during electrophysiologic testing.
%We worked with 123 records from 47 patients (a patient may have more than one recording session).
%We segmented each record into \emph{episodes}: an episode is a segment of a record with one main rhythm, e.g., Normal Sinus Rhythm or Ventricular Fibrillation.
%For a given rhythm, the database generally contains several episodes.
%Then from each episode of each rhythm, we extracted a number of electrograms, say, 10.
%The extracted \ac{EGM}s provide a \emph{signature} for what the \ac{EGM} looks like during that particular rhythm.
%Fig. \ref{fig:adjudication} illustrates the process of obtaining these signatures, as well as the extracted \ac{EGM}s and the rhythms of the episodes from which they were extracted.\begin{figure*}[t]
%		\centering
%		\includegraphics[scale=0.4]{figures/figadjudication.pdf}
%		\caption{\small Examples of \ac{EGM} morphology with different signatures.
%			 }
%		\label{fig:adjudication}
%\end{figure*}