\section{\acl{MBCT}}
\label{sec:mbct}

We demonstrate how a heart model we developed is used to generate a synthetic cohort and test whether Rhythm ID detection algorithm is indeed better than PRL+W as was assumed by the RIGHT investigators.
A negative answer to this question would cause the trial investigators to revise their assumptions before trial start.
% the costs of a trial had been sunk.%
Because an \ac{MBCT} is designed and conducted in support of a given \ac{RCT}, the details necessarily depend on the \ac{RCT} we consider.
In Fig. \ref{fig:mbct overview} we give an overview of the process of the \ac{MBCT} we conducted in support of RIGHT.

\circled{1} Modeling starts by adjudication of 380 individual episodes from a database of 123 \ac{EGM} records of real patients.
Each such episode provides \ac{EGM} morphologies that are annotated with the tachycardia that produced them (see Fig. \ref{fig:adjudication} for examples), resulting in the identification of 19 different rhythms.

\circled{2} An automata-based timing model is used to simulate the timing characteristics of various tachycardias.
Combined with the annotated EGM morphologies, we can now generate parametrized probabilistic heart models that simulate different tachycardias, and variations on each tachycardia.

\circled{3} A cohort of $>11,000$ models is generated by varying the parameters of the heart model. 
The parameter ranges depend on the tachycardia being simulated.

\circled{4} Every member of the cohort is then simulated to produce \ac{EGM} signals that are fed to both ICD algorithms
The rates of inappropriate detection from the two algorithms are analyzed.
We repeated this analysis for various distributions of arrhythmias in our cohort.

\subsection{Advantages of an \ac{MBCT}}
The use of computer models in MBCT gives us significantly more latitude since we don't have the ethical constraints of a clinical trial.
%For example, it is not ethical to implant a placebo device in a patient who needs arrhythmia treatment. 
%No such constraint exists when working with a heart model.
%As another example, a device trial typically can not be blinded, since the patients know whether they are being implanted with a device or not.
%An MBCT, on the other hand, can be easily double-blinded.
In particular, in an \ac{RCT}, a patient will typically be on either intervention (new device) or control (currently accepted device or other standard treatment). 
It is usually not possible or ethical to retrieve the first device from the patient just for the purposes of the trial, and implant them with the other device.
%Thus despite randomization, a question remains about the comparability of the two groups: are we really comparing two similar groups, so that any differences between them at trial's end can be reliably attributed to differences in treatment (rather than differences in the patients themselves)?
In an \ac{MBCT} on the other hand, the same model can be subjected to both intervention and control, so that perfect comparability between the two groups is assured.

A significant advantage of \acp{MBCT} is that we can generate very large cohorts, thus lending statistical strength to the results.
We control the variability in the cohort, so it is possible to test the effects of the new device on a particular sub-group, e.g., people with one dominant type of arrhythmia.
Moreover, we can test the outcome under varying device parameter values, something which is not feasible in a clinical trial.

\subsection{What an \ac{MBCT} is not}
An \ac{MBCT} does not seek to replace an \ac{RCT}: the latter provides data on the safety and efficacy of the new device \emph{in the clinical setting}.
That is, under conditions that closely resemble the conditions under which the device will be used in real life, in clinics and hospitals around the world.
Rather, an \ac{MBCT} is designed to improve key steps in the planning and execution of an \ac{RCT}, and to confirm early on assumptions about the effectiveness of the new device.

\headline{The difference between MBCT and current Model-Based Design (MBD) is one of goals and emphasis.
	The goal of an MBCT is to help better plan and conduct a subsequent \ac{RCT}.
	It looks at things like statistical significance and performance in a target population.
	The goal of MBD is to design and debug the device. 
	It looks at things like corner cases and implementation errors.
	The device (or software) that is evaluated in an MBCT has already been through the verification stage and is considered market-ready (pending trial results and regulatory approval).}
