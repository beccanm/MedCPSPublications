
%%Primary result: 
%What is: 
%Early feasibility of a clinical trial
%Model-based clinical trials provide empirical evidence to the feasiblity of a clinical trial (compared to extrapolating from literature)
%Provide insight into the feasiblity of a clinicial trial
%
% - Comparative analysis of ICD algorithms: 
%	 - By generating a large cohort (over 10,000 heart models) we show that the Medtronic algorithm delivers less inappropriate therapy than the Boston Scientific algorithm.
% 
% - Condition-specific analysis
%	 - The Medtronic algorithm has better specificity than the BSc algorithm and we isolated the arrhythmias where the effect is most pronounced
%	 
% - Effect of device parameters on discrimination ability
%	 - By running multiple model-based trials across a range of parameters, this provides insight to the physician on the most appropriate parameter values.
%	 
%While the goal of MBCT is not to replace RCT it provides the following benefits:
%
%- Generate large and varied population distributions
%- Ability to input the same signal to multiple devices
%- At low cost in a short time and with no ethical risk
%
%OUTLINE:
%1. Introduction (Rahul)
%- High-level discussion of MBCT
%- fig:bringDeviceToMarket
%- fig:ICD
%- Modeling as regulatory grade evidence
%- Contributions and Results
%
%2. Clinical trials and the RIGHT Trial (Houssam)
%- Clinical trial difficulties
%- Short paragraph explaining background EP
%- Summary of RIGHT Trial, including results
%	- high-level VT vs SVT; Bsc worse
%
%3. Model-based Clinical Trials (Sunday)
%
%- What is MBCT?
%- What are the advantages/benefits?
%- Pipeline of MBCT for RIGHT (Overview figure - previous fig. 3)
%	- fig:MBCToverview
%	- high-level VT vs. SVT
%- What MBCT is not.
%
%4. Heart Models (Hao)
%- Basic EP
%- Timing, morphology
%	-fig:timingModel
%	-fig:EGMgeneration
%- What we did: 
%	-Adjudication (Kuk)
%		-fig:Adjudication
%	-Mapping of parameters to model
%	-Generated 10,000 + models
%		
%
%5. Device Algorithms (Hao)
%- Discrimination algorithm - What? Why?
%	-VT vs SVT
%	
%- VT detection algorithm (Part a: sensing (Kuk) Part b: detection)
%	-fig:SVTVTDetection
%- What we did: 
%	-Implement complex algorithm
%		- reference for different components
%	- Validation (Kuk)
%		-fig:validationsetup
%		-fig:validation
%			(add text: Real device, Device model)
%			
%6. Results (Hao, Houssam)
%- Overview of results
%
% - Comparative analysis of ICD algorithms: 
%	 - By generating a large cohort (over 10,000 heart models) we show that the Medtronic algorithm delivers less inappropriate therapy than the Boston Scientific algorithm.
%	 -fig:population variation
% 
% - Condition-specific analysis
%	 - The Medtronic algorithm has better specificity than the BSc algorithm and we isolated the arrhythmias where the effect is most pronounced
%	 -table:condition-level specificity
% 
% - Effect of device parameters on discrimination ability
%	 - By running multiple model-based trials across a range of parameters, this provides insight to the physician on the most appropriate parameter values.
%	 -fig:device params
%	 
%7. Discussion (Houssam)
%- Limitations w.r.t. RIGHT Trial 
%	- Don't expect to get exact numbers, trends can match (overall outcome)
%		- Cannot get patient-level out
%		- We can get condition-specific trends 
%	- Only implemented components of sensing and detection (no onset, post-shock detection)
%- Differences in adjudication
%	- Different doctors, different adjudication
%	- Not reliable without EKG
%
%8. Conclusion
%
%
% 
% 
