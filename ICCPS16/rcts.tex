\section{Clinical trials and RIGHT}
\label{sec:rcts}

At the clinical trial stage (Fig.~\ref{fig:spectrum}), the objective is no longer to find bugs in the device: it is, rather, to evaluate the safety and efficacy of the validated device on humans. 
\acp{RCT} are the gold standard for evaluating the safety and efficacy of a new medical device \cite{FriedmanFD10_ClinicalTrials}.
They constitute the only time prior to market use where the effects of the device on humans are actually observed, and are legally mandated for new high-risk medical devices like \acp{ICD}.
%It is important to understand how a typical RCT proceeds to appreciate where CPS modeling can help in that process.
%Very broadly, in an \ac{RCT}, a new treatment also known as the \emph{intervention}, is compared to the current standard of care, commonly known as the \emph{control}.
%For example, two competing ICDs are compared.
%Each patient recruited to the trial is randomly assigned to either the intervention or the control group and undergoes the corresponding treatment.
%At the end of the trial, any clinically relevant differences between the two groups are evaluated to determine if they are statistically significant.
The planning and execution of an \ac{RCT} requires carefully navigating a number of technical, logistical and ethical issues to obtain reliable and statistically significant results.

Because of the very high cost of \acp{RCT} in terms of money, time, and the risk of harm they present to enrolled patients, our focus in this paper is on the use of CPS models, formalized in an \ac{MBCT}, \emph{to validate the assumptions made by the investigators and thus increase the chances of success of an \ac{RCT}}.
We illustrate our approach by applying it to the Rhythm ID Going Head-to-Head Trial (RIGHT) \cite{GoldABBTB11_RIGHTresults}, which we present next.
\subsection{The RIGHT trial}
\label{sec:right}
We first provide a brief background to better understand RIGHT (see Fig.\ref{fig:icd}). Tachycardias (abnormally elevated heart rates) can be divided into \acp{VT}, which originate in the heart's ventricles, 
and \acp{SVT}, which originate above the ventricles.
A sustained \ac{VT} can be fatal, while an \ac{SVT} is typically non-fatal.
The therapy applied by the ICD often takes the form of a high-energy electric shock.
The shock can be pro-arrhythmic
\mynote{SD}{replaced pro-arrhythmic with 'quite uncomfortable'. We got pro-arrhythmic effect from Pinksy et al. "The Proarrhythmic Potential of Implantable Cardioverter-Defibrillators", 1995. Is it out-dated? It was cited in RIGHT 2006. }, and was even linked to increased morbidity \cite{shock_mortality}.
Therefore, one of the biggest challenges for ICDs is to guarantee shock delivery for \acp{VT}, and simultaneously reduce inappropriate shocks during \acp{SVT} \cite{Ellenbogen11_Pacingbook}.

RIGHT is a trial that sought to compare the VT/SVT discrimination abilities of two algorithms \cite{GoldABBTB11_RIGHTresults}: 
the Rhythm ID detection algorithm found in Boston Scientific's Vitality II ICDs~\cite{compass},
and the PR Logic + Wavelet (PRL+W) detection algorithm found in a number of Medtronic's ICDs (Medtronic Maximo,
Marquis, Intrinsic, Virtuoso, or Entrust ICD).
%\mynote{SD}{make sure to list all the Boston Scientific and Medtronics ICD models that use each of these algorithims. This is very important}
%The investigators chose the following primary question: is there a difference in time-to-first inappropriate therapy between the two ICDs?
\emph{Inappropriate therapy} was defined as therapy applied to an arrhythmia other than \ac{VT} or \ac{VF} (\ac{VF} is a type of \ac{VT}).
RIGHT enrolled 1962 patients and ran for approximately five years.
It was fully sponsored by Boston Scientific. 

One of the trial's assumptions was that Rhythm ID would reduce the risk of inappropriate therapy by 25\% over PRL+W~\cite{Berger06_RIGHT}.
The outcome of the trial~\cite{GoldABBTB11_RIGHTresults}, however, was that patients implanted with ICDs running Rhythm ID had a \emph{\textbf{34\% risk increase}} of inappropriate therapy as compared to patients implanted with \acp{ICD} running PRL+W. 
This result  is the opposite of the effect hypothesized by the trial investigators. 
In this paper, we design an \ac{MBCT} to test early and quickly whether the hypothesized effect holds by comparing the two ICDs on a large \emph{synthetic} cohort.\\\\
\textbf{\emph{Organization:}} In the following sections we describe the building blocks of the MBCT: modeling the heart, processing 100's of real patients' data, mapping the timing and morphology components of the signal to a heart model we developed, generating a population of 10,000+ synthetic heart models, implementing the device algorithms and conducting multiple trials for the comparative rate of inappropriate therapy, condition-level rates and evaluating the effect of device parameters on discrimination rates.

%Note that these differences in inappropriate therapy were confined to single-chamber ICDs.

%The patients were randomized to two groups: the intervention group were implanted with Vitality II ICDs which ran the Rhythm ID discriminators, and the control group were implanted with Medtronic ICDs.
%In the rest of this paper, for conciseness, we will refer to these as Vitality II and Medtronic groups.
%, but the reader should keep in mind that this wasn't a true algorithm to algorithm comparison. 
%Rather, the intervention group all received one type of device (Vitality II) while the control group received a variety of Medtronic devices, all running PRL+W, depending on the recommendation of the patient's treating physician.


