\paragraph{Modeling as regulatory grade evidence}
\label{sec:related work}
Most medical models today are aimed at either better understanding the phenomenon under study \cite{vfiborganization_Tusscher07} or at device debugging and verification \cite{VHM_proc}. 
There is only one case in which a computer model has been used to intervene in the regulatory process of medical devices, namely the T1 Diabetes Model (T1DM) of UVA/PADOVA \cite{T1DM}.
T1DM models glucose kinetics in hypoglycemia, and has been accepted by the FDA as a substitute for animal trials.
The T1DM has a fixed virtual cohort with 300 patients.
Its objective is to test the efficacy of new glucose control algorithms by simulating them on the virtual cohort.
While our models can be used in this way, our objective here is to target specific clinical trials steps and improve how they are conducted.
This dictates the experimental setup and the cohort generation considerations.

The Avicenna consortium \cite{Avicenna} lays out a vision for `In-Silico Clinical Trials' similar to our approach.
However, the emphasis in Avicenna is on individualized patient models, as they propose to customize the model to each patient enrolled in a trial.
In the present work, we propose a usage of MBCT \emph{prior} to recruitment.
Thus our models need not be fitted to a given patient's data, which might be impossible, invasive, or burdensome for the conduct of the trial.
