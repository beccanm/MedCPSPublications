\subsection{The RIGHT trial}
\label{sec:right}
We first provide a brief background to better understand RIGHT (see Fig.\ref{fig:icd}). Tachycardias (abnormally elevated heart rates) can be divided into \acp{VT}, which originate in the heart's ventricles, 
and \acp{SVT}, which originate above the ventricles.
A sustained \ac{VT} can be fatal, while an \ac{SVT} is typically non-fatal.
The therapy applied by the ICD often takes the form of a high-energy electric shock.
The shock can be pro-arrhythmic
\mynote{SD}{replaced pro-arrhythmic with 'quite uncomfortable'. We got pro-arrhythmic effect from Pinksy et al. "The Proarrhythmic Potential of Implantable Cardioverter-Defibrillators", 1995. Is it out-dated? It was cited in RIGHT 2006. }, and was even linked to increased morbidity \cite{shock_mortality}.
Therefore, one of the biggest challenges for ICDs is to guarantee shock delivery for \acp{VT}, and simultaneously reduce inappropriate shocks during \acp{SVT} \cite{Ellenbogen11_Pacingbook}.

RIGHT is a trial that sought to compare the VT/SVT discrimination abilities of two algorithms \cite{GoldABBTB11_RIGHTresults}: 
the Rhythm ID detection algorithm found in Boston Scientific's Vitality II ICDs~\cite{compass},
and the PR Logic + Wavelet (PRL+W) detection algorithm found in a number of Medtronic's ICDs (Medtronic Maximo,
Marquis, Intrinsic, Virtuoso, or Entrust ICD).
%\mynote{SD}{make sure to list all the Boston Scientific and Medtronics ICD models that use each of these algorithims. This is very important}
%The investigators chose the following primary question: is there a difference in time-to-first inappropriate therapy between the two ICDs?
\emph{Inappropriate therapy} was defined as therapy applied to an arrhythmia other than \ac{VT} or \ac{VF} (\ac{VF} is a type of \ac{VT}).
RIGHT enrolled 1962 patients and ran for approximately five years.
It was fully sponsored by Boston Scientific. 

One of the trial's assumptions was that Rhythm ID would reduce the risk of inappropriate therapy by 25\% over PRL+W~\cite{Berger06_RIGHT}.
The outcome of the trial~\cite{GoldABBTB11_RIGHTresults}, however, was that patients implanted with ICDs running Rhythm ID had a \emph{\textbf{34\% risk increase}} of inappropriate therapy as compared to patients implanted with \acp{ICD} running PRL+W. 
This result  is the opposite of the effect hypothesized by the trial investigators. 
In this paper, we design an \ac{MBCT} to test early and quickly whether the hypothesized effect holds by comparing the two ICDs on a large \emph{synthetic} cohort.\\\\
\textbf{\emph{Organization:}} In the following sections we describe the building blocks of the MBCT: modeling the heart, processing 100's of real patients' data, mapping the timing and morphology components of the signal to a heart model we developed, generating a population of 10,000+ synthetic heart models, implementing the device algorithms and conducting multiple trials for the comparative rate of inappropriate therapy, condition-level rates and evaluating the effect of device parameters on discrimination rates.

%Note that these differences in inappropriate therapy were confined to single-chamber ICDs.

%The patients were randomized to two groups: the intervention group were implanted with Vitality II ICDs which ran the Rhythm ID discriminators, and the control group were implanted with Medtronic ICDs.
%In the rest of this paper, for conciseness, we will refer to these as Vitality II and Medtronic groups.
%, but the reader should keep in mind that this wasn't a true algorithm to algorithm comparison. 
%Rather, the intervention group all received one type of device (Vitality II) while the control group received a variety of Medtronic devices, all running PRL+W, depending on the recommendation of the patient's treating physician.


