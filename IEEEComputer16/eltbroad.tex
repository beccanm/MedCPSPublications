\section{Life-critical Closed-loop Software}
\label{sec:eltbroad}

The heart is a specialized muscle that pumps oxygenated blood to the rest of the body.
It is composed of four chambers: two upper chambers called the left and right atrium, and two lower chambers called the left and right ventricle, which contract synchronously.
%The blood depleted of oxygen flows into the right atrium and passes into the right ventricle. 
%The latter contracts and pumps the blood to the lungs.
%Once oxygenated in the lungs, the blood flows into the left atrium and thereafter into the left ventricle. 
%The latter contracts and pumps the oxygenated blood to the rest of the body.
%Ventricular contractions are synchronized, as are atrial contractions.
In a healthy resting adult, the heart rate is 60 to 100 beats per minute (each ventricular contraction is a beat). The contractions of the heart are controlled by the waves of spontaneous electric depolarization that traverse it regularly.
A spontaneous electric current originates in the \emph{Sino-Atrial (SA) node} in the right atrium and propagates throughout the atria, causing them to contract. 
It then propagates down to the ventricles along well-defined conduction pathways, causing the ventricles to contract in turn.
The SA node is thus termed the natural pacemaker of the heart.

Under certain diseased conditions, the heart rate drops below what is needed to maintain adequate blood flow to the body. 
This clinical condition is called \emph{bradycardia}.
When such a heart rate drop is due to abnormalities in the electrical conduction system, an implanted pacemaker might be recommended as treatment. A pacemaker is implanted near the left collar of the patient as shown in Fig.\ref{fig:pacemaker},
and has two leads: one connects to the right atrium, the other to the right ventricle. 

The leads act as both sensors and effectors: if the pacemaker fails to sense electric activity on either lead within certain time constraints, indicative of a delayed/missed contraction, it will send an electric pulse to the corresponding chamber to provoke contraction, thus acting as an artificial pacemaker.
The algorithms for detecting missed beats are complex and implemented in software which runs within the pacemaker itself. 
Part of the difficulty of performing that detection comes from the great variability in heart rates between patients and even within a single patient across time.
Moreover, because the pacemaker is limited to sensing electrical activity through its two leads, different phenomena can manifest themselves identically to the pacemaker, thus making detection even harder.

For example, in Endless Loop Tachycardia (ELT), this ambiguity causes the pacemaker to actually \emph{induce} dangerously elevated heart rates (tachycardia), which would not have arisen had the heart been operating on its own.
This is an example of a \emph{adverse closed-loop condition}: a dangerous situation that arises as a result of the \emph{interaction} between device and heart.
No amount of open-loop device testing and verification can reveal this condition - hence the need for closed-loop validation of medical devices, and for physiological heart models that enable early and affordable closed-loop validation.
