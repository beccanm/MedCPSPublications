\section{Heart model validation}
\label{sec:heart model validation}

Model description: What is the purpose for our model? Generating arrhythmias - so show the 19 different arrhythmias, how they map to the model (nodes/paths), parameters and internal state of each model. Show 19 examples with timing only and timing and morphology. 

The purpose of the model is to generate exemplars of a pre-determined set of tachyarrhythmias.
Each arrhythmia must be either a VT or an SVT.
Each arrhythmia episode must be led by an interval of NSR, to allow the ICD algorithms to acquire an NSR template.

The 19 arrhythmias are:
'AF'    'AF-PVC'    'AF-VF'    'AF-VT'    'AF-ashman'    'Aft'    'Aft-VT'    'NSVT'    'PAC'    'PAC-VF'    'PVC'    'SVT'    'SVT-Wenkebach', 'SVT-Wenkebach-PVC'    'SickSinus'    'VF'    'VT'    'VT-PAC'    'VT-retrograde'.
These have been extracted from the annotations that Jackson provided to the database signals.
\todo[inline]{show example of each}

Most exemplars should be physiologically valid, i.e. could've been generated by a human heart.
\gap{We haven't quantified `most'}. 
Moreover, there should be some amount of variability between the exemplars of the same arrhythmia.
\gap{We haven't defined or quantified `variability'}.

\gap{The only validation that was done was to show 19 episodes to Sanjay, one of each arrhythmia.
He said they looked OK.}

Briefly, the UVA/PADOVA T1DM model is obtained by: fitting the model's parameters to the data from 204 normal subjects;
scaling the parameters of the 204 synthetic subjects by various clinically obtained quantities to obtain 204 T1DM patients;
fitting a log-normal distribution to the 204 parameters;
sampling from that distribution.

Validation strategies:
\begin{enumerate}
	\item The validity of the [UVA/PADOVA] cohort was tested by experiments “aiming to assess its capability to reflect a variety of clinical situations as closely as possible: 1) Reproducing the distribution of insulin correction factors in the T1DM population  2) Reproducing glucose traces in children with T1DM observed in clinical trials. 3) Reproducing glucose traces of induced moderate hypoglycemia”
	\item Hypothesis test
\end{enumerate}
