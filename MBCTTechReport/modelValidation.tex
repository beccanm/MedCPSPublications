\section{Heart model validation}
\label{sec:heart model validation}

Model description: What is the purpose for our model? Generating arrhythmias - so show the 19 different arrhythmias, how they map to the model (nodes/paths), parameters and internal state of each model. Show 19 examples with timing only and timing and morphology. 

The purpose of the model is to generate exemplars of a pre-determined set of tachyarrhythmias.
Each arrhythmia must be either a VT or an SVT.
Each arrhythmia episode must be led by an interval of NSR, to allow the ICD algorithms to acquire an NSR template.

The 19 arrhythmias are:
'AF'    'AF-PVC'    'AF-VF'    'AF-VT'    'AF-ashman'    'Aft'    'Aft-VT'    'NSVT'    'PAC'    'PAC-VF'    'PVC'    'SVT'    'SVT-Wenkebach', 'SVT-Wenkebach-PVC'    'SickSinus'    'VF'    'VT'    'VT-PAC'    'VT-retrograde'.
These have been extracted from the annotations that Jackson provided to the database signals.
\todo[inline]{show example of each}

Most exemplars should be physiologically valid, i.e. could've been generated by a human heart.
We haven't quantified `most'. 
Moreover, there should be some amount of variability between the exemplars of the same arrhythmia.
We haven't defined or quantified `variability'.

The only validation that was done was to show 19 episodes to Sanjay, one of each arrhythmia.
He said they looked OK.

Validation: actually validate the model? Or adopt a sampling-based approach to manually adjudicating the set of 11400 signals we actually used? [Discuss with Jackson]

